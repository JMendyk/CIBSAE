
\newcommand{\file}[1]{\texttt{#1}}

\chapter{Notes on the attached source code}

\section{Division of implementation into files}

\begin{itemize}
\item \file{Common.he} and \file{Signature.he} -- definitions of basic types and signatures,
\item \file{Store.he} -- Store effect and handlers for it,
\item \file{Schedulers.he} -- implementations of schedulers,
\item \file{Rebuilders.he} -- implementations of rebuilders,
\item \file{Traces.he} -- functions for interacting with traces,
\item \file{Track.he} -- implementations of \helinl{run} and \helinl{track} functions,
\item \file{Systems.he} -- example definitions of described systems,
\item \file{Spreadsheet.he} -- implementations of entities related to spreadsheets,
\item \file{scratch.he} -- (demo) usage of build systems to build exemplary tasks,
\item \file{Utils.he} -- helper functions,
\item \file{PatchedWriter.he} -- extension of \helinl{Writer} module from standard library,
\item \file{State.he} -- implementation of mutable state effect along with handlers,
\item \file{Logger.he} -- utility module for tracking programs' behaviour,
\item \file{SchedulersWithLogging.he} -- implementation of schedulers as in \file{Schedulers.he} enriched with runtime diagnostic messages about their behaviour
\end{itemize}

\section{Implementations of traces}

Description of traces in chapter 5 was omitted, since they didn't seem like an interesting part of implementation. However, for orderliness it's implementations are provided here.

\subsection{Trace types}

For traces we define a type analogous to the one from original implementation in Haskell.

\begin{minipage}{\textwidth}
  \lstinputlisting[style=Haleff-long, linerange={13-14}]{../src/Traces.he}
\end{minipage}

It is worth noting, that building of tasks using \helinl{fetch} in verifying and constructive traces happens in a specialised handler, which returns hashes instead of plain values.

\subsection{Verifying traces}

\begin{minipage}{\textwidth}
  \lstinputlisting[style=Haleff-long, linerange={18-28}]{../src/Traces.he}
\end{minipage}

\subsection{Constructive traces}

\begin{minipage}{\textwidth}
  \lstinputlisting[style=Haleff-long, linerange={37-49}]{../src/Traces.he}
\end{minipage}

In functions \helinl{constructCT} and \helinl{deepDependencies} we are using \helinl{Writer} effect. In \BSaLC{} type \haskinl{Maybe}, function \haskinl{catMaybes} and list comprehensions were used instead. The author, encouraged by lack of infrastructure around \helinl{Some} (Helium's counterpart of \haskinl{Maybe}), decided to test \helinl{Writer} module and see what results it gives.

\subsection{Deep constructive traces}

\begin{minipage}{\textwidth}
  \lstinputlisting[style=Haleff-long, linerange={53-73}]{../src/Traces.he}
\end{minipage}
