
\newcommand{\haskinl}[1]{\lstinline[style=haskell-inl]{#1}}
\newcommand{\helinl}[1]{\lstinline[style=Haleff-inl]{#1}}

\chapter{Systemy kompilacji z użyciem efektów algebraicznych i uchwytów}
\chaptermark{Systemy z użyciem efektów i uchwytów}

W tym rozdziale powtórzymy implementację systemów kompilacji przedstawioną w \BSaLC\cite{mokhov2018build}, jednak dokonamy jej w Helium używając efektów i uchwytów. Na początku wymyślimy własne odpowiedniki abstrakcyjnych struktur z Haskella związanych z systemami, następnie zaimplementujemy wszystkich rekompilatorów oraz wszystkich prócz jednego planistów. Na koniec przyglądniemy się co problematycznej jest w pominiętym planiście, że jego implementacja została pominięta.

\section{Pomysł, typy i idea}

Przypomnijmy sobie Haskellowe reprezentacje składowych implementacji oraz wprowadźmy ich Heliumowe odpowiedniki.

\subsection{Zasób (Store)}

\begin{lstlisting}[style=haskell-style]
data Store i k v
initialise :: i -> (k -> v) -> Store i k v
getInfo :: Store i k v -> i
putInfo :: i -> Store i k v -> Store i k v
getValue :: k -> Store i k v -> v
putValue :: Eq k => k -> v -> Store i k v -> Store i k v
\end{lstlisting}

Autorzy \BSaLC\cite{mokhov2018build} reprezentowali Store jako typ z operacjami odczytu i zapisu trwałem informacji dla systemu oraz wartości wynikowych. Każdorazowo jednak, wartość Store była przechowywana w Haskellowym typie modyfikowalnego stanu \haskinl{Store}. Możemy więc uprościć implementację przez scalenie zasobu z modyfikowalnym stanem przez uczynienie \haskinl{Store} efektem, a działania na nim operacjami powodującymi ten efekt.

\lstinputlisting[language=Haleff, firstline=5, lastline=9]{../Store.he}

Podobnie jak \haskinl{Store} w oryginalnej implementacji, \helinl{StoreEff} jest polimorficzny ze względu na typy informacji, kluczy oraz wartości wynikowych kompilacji. Równania dla niego są analogiczne jak dla zwykłego modyfikowanego stanu z dokładnością do ustalenia klucza w operacjach na wartościach wynikowych. Definiujemy ponadto uchwyt \helinl{funStoreHandler} dla obliczeń używających zasobu reprezentowanego przez funkcję -- jak w przykładach w \BSaLC.

\lstinputlisting[language=Haleff, firstline=38, lastline=52]{../Store.he}

Implementacja jest zbliżona do przykładu modyfikowalnego stanu z rozdziału 4. Dla porządku wartość początkowa trwałej informacji oraz słownika wartości jest opakowana w typ \helinl{FunStoreType I K V}.

Jako że Helium, podobnie jak inne języki używające ML-owego systemu modułów nie posiadają klas typów znanych z Haskella, definiujemy kilka sygnatur odpowiadających klasom typów użytym w oryginalnej implementacji. Moduł o sygnaturze \helinl{Comparable K} jest więc używany w \helinl{funStoreHandler} do porównywania kluczy identyfikujących zadania.

\begin{minipage}[t]{.45\textwidth}

  \lstinputlisting[language=Haleff, firstline=4, lastline=16]{../Signatures.he}

\end{minipage}\hfill
\begin{minipage}[t]{.45\textwidth}

  \lstinputlisting[language=Haleff, firstline=18, lastline=27]{../Signatures.he}

\end{minipage}

\subsection{Modyfikowalny stan}

Implementację modyfikowalnego stanu zobaczyliśmy w przykładach w rozdziale 4 i wykorzystamy ją konstruując systemy kompilacji. Nazwy uchwytom dla stanu, w zależności od zwracanych wartości, nadajemy zgodnie z ich Haskellowymi odpowiednikami -- \haskinl{runState}, \haskinl{evalState}, \haskinl{execState}. Definiujemy także prostą funkcję \helinl{modiy} przekształcającą stan używając podanej funkcji oraz nieco bardziej skomplikowaną funkcję \helinl{embedState}.

\lstinputlisting[language=Haleff, linerange={7-7,33-37}, float=h, title={Definicje \helinl{modify} oraz \helinl{embedState}}]{../State.he}

Funkcja \helinl{embedState} tworzy uchwyt dla efektu modyfikowalnego stanu, w którym modifykacje -- zamiast być wykonywane przez uchwyt -- są przekazywanego podanym funkcjom \helinl{getter} oraz \helinl{setter}, które w czasie swojego działania mogą powodować jakiś efekt uboczny. Z takiego zanurzenia modyfikowalnego stanu w innym efekcie będziemy korzystać podczas implementacji planistów, którzy trwałą informację z zasobu będą przekazywać do rekompilatorów jako właśnie modyfikowalny stan.

\begin{lstlisting}[language=Haleff, float=h, caption={Wykorzystanie \helinl{embedState}}]
handle `store in
    (* ... *)
    handle `state in
        (* ... *)
    with embedState (getInfo `store) (putInfo `store)
    (* ... *)
with (* ... *)
\end{lstlisting}

\subsection{Zadanie i efekt budowania}

W oryginalnej implementacji, zadanie było funkcją przyjmującą procedurę budowania wskazanego zadania, a wynik był zwracany w jakimś typie \haskinl{f} ograniczonym przez klasę typów \haskinl{c}.

\begin{lstlisting}[style=haskell-style]
newtype Task c k v = Task (forall f. c f => (k -> f v) -> f v)
type Tasks c k v = k -> Maybe (Task c k v)
\end{lstlisting}

Możemy jednak zauważyć, że zbudowanie zadania jest oczywistym efektem ubocznym działania systemu kompilacji, stąd w naszej implementacji zamiast przekazywać funkcję, która była przez autorów zazwyczaj nazywana \haskinl{fetch}, zdefiniujemy efekt \helinl{BuildEff}, który będzie występował w czasie kompilacji zadań. Efekt ten będzie miał związaną jedną operację \helinl{fetch}.

\lstinputlisting[language=Haleff, linerange={33-38}]{../Common.he}

Zadanie będzie funkcją wymagającą informacji o instancji efektu budowania i będzie polimorficzna ze względu na typ kluczy i wartości oraz ewentualnych efektów ubocznych nie będących efektem budowania (przyda się to przy implementacji rebuilderów). Zwróćmy uwagę, że definicja typu zadania nie zawiera informacji o klasie typów \haskinl{c} której element \haskinl{f} ,,opakowuje'' wynik w oryginalnej implementacji -- do tej różnicy powrócimy w dalszej części rozdziału.

\subsection{Kompilacja, planista, rekompilator}

Pozostaje zdefiniować trzy ostatnie typy związane ze wspomnianymi w podtytule obiektami. Kompilacja tak jak w oryginalnej implementacji wymagać będzie wskazania zbioru zadań oraz klucza który ma być zbudowany. Ponadto, w  naszej implementacji kompilacja powoduje efekt uboczny zmiany zasobu.

\begin{lstlisting}
type Build c i k v = Tasks c k v -> k -> Store i k v -> Store i k v
type Scheduler c i ir k v = Rebuilder c ir k v -> Build c i k v
type Rebuilder c ir k v = k -> v -> Task c k v -> Task (MonadState ir) k v
\end{lstlisting}

Nasi planiści także będą mieli sygnatury zbliżone do swoich Haskellowych odpowiedników, wzbogacone oczywiście o efekt uboczny zasobu, a także moduł definiujący podstawowe działania na kluczach i wartościach opisane wcześniej.

\lstinputlisting[language=Haleff, linerange={18-20}]{../Common.he}

Rebuilder także przypomina swój odpowiednik z oryginalnej implementacji. Jednakże, zamiast zwracać zadanie ze zmienionym \textit{constraint'em}, zadanie jest wzbogacone o dodatkowy efekt stanu mogący występować w czasie kompilacji zadania, a związany z przerobieniem go przez rekompilator.

\section{Przykład: system busy}

Skoro ustaliliśmy jak Haskellowa abstrakcja systemów kompilacji przenosi się na naszą w Helium, możemy spróbować zaimplementować prosty system budowania \haskinl{busy} przedstawiony przez autorów.

\lstinputlisting[language=Haleff, linerange={16-28}, float=h, title={System kompilacji \haskinl{busy}}]{../Schedulers.he}

Rdzeniem tej implementacji, podobnie jak oryginalnej w Haskellu, jest definicja uchwytu (tak: funkcji) dla \helinl{fetch}. Jego ciało to przetłumaczenie oryginalnej implementacji z tą różnicą, że zamiast kontynuować obliczenie niejawnie -- przez zwracanie wyniku -- jest ono kontynuowane jawnie przez wywołanie \helinl{resume} w ciele uchwytu.

\section{Implementacje systemów kompilacji}

\subsection{Excel}

\lstinputlisting[style=Haleff-long, linerange={13-14}]{../Systems.he}

\lstinputlisting[style=Haleff-long, linerange={31-34}]{../Rebuilders.he}

\lstinputlisting[style=Haleff-long, linerange={56-102}]{../Schedulers.he}

\subsection{Shake}

\lstinputlisting[style=Haleff-long, linerange={16-17}]{../Systems.he}

\lstinputlisting[style=Haleff-long, linerange={60-66}]{../Rebuilders.he}

\lstinputlisting[style=Haleff-long, linerange={30-54}]{../Schedulers.he}

\subsection{CloudShake}

\lstinputlisting[style=Haleff-long, linerange={19-20}]{../Systems.he}

\lstinputlisting[style=Haleff-long, linerange={95-105}]{../Rebuilders.he}

%% \lstinputlisting[style=Haleff-long, linerange={30-54}]{../Schedulers.he}

\subsection{Nix}

\lstinputlisting[style=Haleff-long, linerange={22-23}]{../Systems.he}

\lstinputlisting[style=Haleff-long, linerange={131-141}]{../Rebuilders.he}

%% \lstinputlisting[style=Haleff-long, linerange={30-54}]{../Schedulers.he}

\section{Istniejące podejścia do implementacji w innych językach}

Choć opisane wyżej wyniki są pierwszą -- według wiedzy autora -- próbą implementacji systemów kompilacji inspirowanych \BSaLC używając języka z efektami oraz uchwytami, w \BSaLCTP autorzy wspominają o dwóch znanych im próbach implementacji w innych językach: Rust\cite{translation_rust} oraz Kotlin\cite{translation_kotlin}. Jak jednak zauważają autorzy, w obu przypadkach ograniczenia użytych języków doprowadziły do utracenia precyzji i schludności rozwiązań w prównaniu z oryginalną implementacją w Haskellu.
