% Opcje klasy 'iithesis' opisane sa w komentarzach w pliku klasy. Za ich pomoca
% ustawia sie przede wszystkim jezyk i rodzaj (lic/inz/mgr) pracy, oraz czy na
% drugiej stronie pracy ma byc skladany wzor oswiadczenia o autorskim wykonaniu.
\documentclass[lic,declaration,shortabstract]{iithesis}

\usepackage[utf8]{inputenc}

%%%%% DANE DO STRONY TYTUŁOWEJ
% Niezaleznie od jezyka pracy wybranego w opcjach klasy, tytul i streszczenie
% pracy nalezy podac zarowno w jezyku polskim, jak i angielskim.
% Pamietaj o madrym (zgodnym z logicznym rozbiorem zdania oraz estetyka) recznym
% zlamaniu wierszy w temacie pracy, zwlaszcza tego w jezyku pracy. Uzyj do tego
% polecenia \fmlinebreak.
\polishtitle    {Kwalifikacja i implementacja\fmlinebreak systemów kompilacji z użyciem\fmlinebreak efektów algebraicznych}
\englishtitle   {Categorization and implementation of Build Systems using algebraic effects}
\polishabstract {\ldots}
\englishabstract{\ldots}
% w pracach wielu autorow nazwiska mozna oddzielic poleceniem \and
\author         {Jakub Mendyk}
% w przypadku kilku promotorow, lub koniecznosci podania ich afiliacji, linie
% w ponizszym poleceniu mozna zlamac poleceniem \fmlinebreak
\advisor        {dr Filip Sieczkowski}
\date          {4 września 2020}                     % Data zlozenia pracy
% Dane do oswiadczenia o autorskim wykonaniu
\transcriptnum {301111}                     % Numer indeksu
\advisorgen    {dr. Filipa Sieczkowskiego} % Nazwisko promotora w dopelniaczu
%%%%%

%%%%% WLASNE DODATKOWE PAKIETY
%
%\usepackage{graphicx,listings,amsmath,amssymb,amsthm,amsfonts,tikz}
\usepackage{cite}
\usepackage{amsmath, amsfonts, amsthm}
\usepackage{stmaryrd} % double brackets
\usepackage{listings}
\usepackage[usenames,dvipsnames]{xcolor}

%%%%% WŁASNE DEFINICJE I POLECENIA
%
%\theoremstyle{definition} \newtheorem{definition}{Definition}[chapter]
%\theoremstyle{remark} \newtheorem{remark}[definition]{Observation}
%\theoremstyle{plain} \newtheorem{theorem}[definition]{Theorem}
%\theoremstyle{plain} \newtheorem{lemma}[definition]{Lemma}
%\renewcommand \qedsymbol {\ensuremath{\square}}
% ...
%%%%%

\begin{document}

%%%%% POCZĄTEK ZASADNICZEGO TEKSTU PRACY

\chapter{Wprowadzenie}

\section{Problemy z efektami ubocznymi}

Programy komputerowe, dzięki możliwości interakcji z zewnętrznymi zasobami takimi jak nośniki pamięci, sieci komputerowe czy użytkownicy oprogramowania, mogą robić istotnie więcej niż tylko zadane wcześniej obliczenia. W ten sposób przebieg programu i jego wynik staje się jednak zależny od tegoż świata zewnętrznego, a sam program nie jest tylko serią czystych obliczeń ale także towarzyszących im efektów ubocznych.

%% v: ,,jednak'' trochę nie pasuje

Efekty uboczne powodują jednak, że rozumowanie i wnioskowanie o sposobie oraz prawidłowości działania programów staje się znacznie trudniejsze, a w konsekwencji ogranicza ich modularność i prowadzi do częstszych pomyłek ze strony autorów. Chcąc tego uniknąć, dąży się do wydzielania w programie jak największej części, która składa się z czystych obliczeń. Jednak to, czy jakiś moduł oprogramowania wykonuje obliczenia bez efektów ubocznych nie koniecznie jest jasne i często musimy zaufać autorowi, że w istocie tak jest.

\section{Radzenie sobie z efektami ubocznymi}

Jednym z rozwiązań tego problemu, jest zawarcie informacji o posiadaniu efektów ubocznych w systemie typów. Możemy skorzystać wtedy z mechanizmów inferencji i weryfikacji typów do automatycznej identyfikacji funkcji, które nie są czyste -- dzięki temu programista może łatwo wyczytać z sygnatury funkcji, które z efektów występują w czasie jej działania. Znanym przykładem umieszczenia efektów w typach jest wykorzystanie monad w języku programowania Haskell. Niestety, jednoczesne użytkowanie dwóch niezależnych zasobów reprezentowanych przez różne monady nie jest łatwe i wymaga dodatkowych struktur, takich jak transformery monad, które niosą ze sobą dodatkowe wyzwania -- problem modularności został jedynie przesunięty w inny obszar.

Nowym, konkurencyjnym podejściem do ujarzmienia efektów ubocznych przez wykorzystanie systemu typów są efekty algebraiczne z uchwytami. Powierzchownie, zdają się być podobne do konstrukcji obsługi wyjątków w językach programowania lub wywołań systemowych w systemach operacyjnych. Dzięki rozdziałowi między definicjami operacji związanych z efektami ubocznymi, a ich semantyką oraz interesującemu zastosowaniu kontynuacji, dają łatwość myślenia i wnioskowania o programach ich używających. Ponadto, w przeciwieństwie do monad, można bezproblemowo korzystać z wielu z nich jednocześnie.

\section{Systemy kompilacji}

Przykładem programów, których głównym zadaniem jest interakcja z zewnętrznymi zasobami są systemy kompilacji, w których użytkownik opisuje proces wytwarzania wyniku jako zbiór wzajemnie-zależnych zadań wraz z informacją jak mają być one wykonywane w oparciu o wyniki innych zadań, zaś system jest odpowiedzialny za ich poprawne uporządkowanie i wykonanie. Ponadto, od systemu kompilacji oczekujemy, że będzie śledził zmiany w danych wejściowych i -- gdy poproszony o aktualizację wyników -- obliczał ponownie jedynie zadania, których wartości ulegną zmianie. Przykładami systemów kompilacji są Make oraz -- co może wydawać się zaskakujące -- programy biurowe służące do edycji arkuszy kalkulacyjnych (np. popularny Excel).

W publikacjach pod tytułem \BSaLC{} \cite{mokhov2018build, mokhov2020build}, autorzy przedstawiają sposób klasyfikacji systemów kompilacji w oparciu o to jak determinują one kolejność w jakiej zadania zostaną obliczone oraz jak wyznaczają, które z zadań wymagają ponownego obliczenia. Uzyskana klasyfikacja prowadzi autorów do skonstruowania platformy umożliwiającej definiowanie systemów kompilacji o oczekiwanych właściwościach. Platforma ta okazuje się być łatwa w implementacji w języku Haskell, a klasy typów Applicative oraz Monad odpowiadać mocy języka opisywania zależności między zadaniami do obliczenia.

\section{O tej pracy}

Celem tej pracy jest zapoznanie czytelnika, który miał dotychczas kontakt z językiem Haskell oraz podstawami języków funkcyjny, z nowatorskim rozwiązaniem jakim są efekty algebraiczne oraz zademonstrowanie -- idąc śladami Mokhov'a i innych -- implementacji systemów kompilacji z wykorzystaniem efektów algebraicznych i uchwytów w języku programowania Helium. W konsekwencji możliwe jest porównanie obu implementacji oraz zaobserwowanie jak wygląda programowanie z efektami algebraicznymi i uchwytami.

W rozdziale drugim wprowadzony zostaje prosty i nieformalny model obliczeń wykorzystujący efekty algebraiczne i uchwyty. Zostaje przedstawionych kilka przykładów reprezentacji standardowych efektów ubocznych w opisanym modelu.

Celem rozdziału trzeciego jest wprowadzenie do \BSaLC{}, opisanie obserwacji poczynionych przez autorów i przedstawienie abstrakcji systemów kompilacji oraz ich konsekwencji. Treść źródłowego artykułu jest opisana w sposób dostateczny aby zrozumieć implementacje systemów z wykorzystaniem efektów i uchwytów przedstawione w rozdziale piątym. Czytelnik jest gorąco zachęcany do samodzielnego zapoznania się z całą treścią publikacji Mokhov'a i innych gdyż jest to bardzo dobra i łatwa w konsumpcji lektura.

Rozdział czwarty rozpoczyna się zapoznaniem czytelnika z istniejącymi językami oraz bibliotekami umożliwiającymi programowanie z efektami i uchwytami. Następnie omówiony jest język Helium oraz przykładowe problemy, wraz z programami je rozwiązującymi z użyciem efektów i uchwytów. Zademonstrowana jest ponadto łatwość wykorzystywania wielu efektów jednocześnie -- w bardziej przystępnej formie niż w przypadku monad w Haskellu.

Zwieńczeniem pracy jest rozdział piąty, w którym przedstawiona jest implementacja planistów, rekompilatorów oraz systemów kompilacji w sposób inspirowany wynikiami \BSaLC{}, jednak używając języka z efektami algebraicznymi i uchwytami. Przedstawione są różnice między abstrakcyjnymi typami od których wyprowadza się implementację oraz w jaki sposób efekty i uchwyty wpływają na formę wyniku. Ponadto, pominięta zostaje implementacja jednego z planistów z wytłumaczeniem dlaczego ma to miejsce.



\chapter{O efektach algebraicznych teoretycznie}

Wprowadzimy notację służącą opisowi prostych obliczeń, która pomoże nam -- bez zanurzania się głęboko w ich rodowód matematyczny -- zrozumieć jak prostym, a jednocześnie fascynującym tworem są efekty algebraiczne i uchwyty. Przedstawiona notacja jest intencjonalnie nieformalna, gdyż ma w dostępny sposób przedstawić abstrakcyjny opis obliczeń z efektami bez prezentowania konkretnego języka programowania.

Następnie przyjrzymy się, jak możemy zapisać popularne przykłady efektów ubocznych używając naszej notacji. Na koniec, czytelnikowi zostaną polecone zasoby do dalszej lektury, które rozszerzają opis z tego rozdziału.

\section{Notacja}

\newcommand{\return}[1]{\mathbf{return}\ #1}
\newcommand{\op}[3]{#1(#2, #3)}
\newcommand{\opi}[3]{\op{op_{#1}}{#2}{#3}}
\newcommand{\handle}[2]{\mathbf{handle}\ #1\ \mathbf{with}\ #2}
\newcommand{\hcase}[3]{#1\ #2\ \Rightarrow\ #3}
\newcommand{\fun}[2]{\lambda #1.\ #2}
\newcommand{\eval}[1]{\llbracket\, #1\, \rrbracket}
\newcommand{\cond}[3]{\mathbf{if}\ #1\ \mathbf{then}\ #2\ \mathbf{else}\ #3}

Będziemy rozważać obliczenia nad wartościami następujących trzech typów:
\begin{itemize}
\item boolowskim \(B\) -- z wartościami \(T\) i \(F\) oraz standardowymi spójnikami logicznymi,
\item liczb całkowitych \(\mathbb{Z}\) -- wraz z ich relacją równości oraz podstawowymi działaniami arytmetycznymi,
\item typem jednostkowym \(U\) -- zamieszkałym przez pojedynczą wartość \(u\),
\item oraz pary tychże typów.
\end{itemize}

% 
% Przemyślenia:
% 1. Pozbyć się "return v"
% 2. Dodać przypadek "return x" do zbioru w uchwycie, zdefiniować jego działanie
%    Można by wtedy rozszerzyć przykłady o zmianę wartości wynikowej
% 

Nasz model składać się będzie z wyrażeń:
\begin{itemize}
\item \(\return{v}\) -- gdzie \(v\) jest wyrażeniem boolowskim lub arytmetycznym,
\item \(\cond{v_1 = v_2}{e_t}{e_f}\) -- wyrażenie warunkowe, gdzie \(v_1 = v_2\) jest pytaniem o równość wartości dwóch wyrażeń arytmetycznych,
\item abstrakcyjnych operacji oznaczanych \(\{op_i\}_{i \in I}\) -- powodujących wystąpienie efektów ubocznych -- których działanie nie jest nam znane, zaś ich sygnatury to \(op_i: A \rightarrow (B \rightarrow C) \rightarrow D\), gdzie \(A\), \(B\), \(C\) oraz \(D\) to pewne typy w naszym modelu. Wyrażenie~\(\opi{i}{n}{\kappa}\) opisuje operację z argumentem \(n\) oraz dalszą częścią obliczenia \(\kappa\) parametryzowaną wynikiem operacji, które \textit{może (nie musi)} zostać wykonane po wykonaniu operacji,
\item uchwytów, czyli wyrażeń postaci \(\handle{e}{\{\ \hcase{op_i}{n\ \kappa}{h_i}\ \}_{i \in I}}\), gdzie \(e\) to inne wyrażenie; uchwyt definiuje działanie (dotychczas abstrakcyjnych) operacji. 
\end{itemize}

Przykładowymi obliczeniami w naszej notacji są więc:
\begin{equation}
\begin{gathered}
  \return{0},\quad\return{2 + 2},\quad \opi{1}{2}{\fun{x}{\return{x + 1}}} \\
  \handle{\opi{1}{2}{\fun{x}{\return{x + 1}}}}{\{\ \hcase{op_1}{n\ \kappa}{\kappa\ (2 \cdot n)} \ \}}
\end{gathered}
\end{equation}

Dla czytelności, pisząc w uchwycie zbiór który nie przebiega wszystkich operacji, przyjmujemy że uchwyt nie definiuje działania operacji; równoważnie, zbiór wzbogacamy o element: \(\hcase{op_i}{n\ \kappa}{op_i(n, \kappa)}\).

Obliczanie wartości wyrażenia przebiega następująco:
\begin{itemize}
\item \(\eval{\return v} = v\) -- wartością \(\mathbf{return}\) jest wartość wyrażenia arytmetycznego,
\item \(\eval{(\fun{x}{e})\ y} = \eval{e \left[x / \eval{y}\right]}\) -- aplikacja argumentu do funkcji,
\item
  \(\begin{aligned}[t]
    \eval{\cond{v_1 = v_2}{e_t}{e_f}} = \left\{\begin{matrix}
    \eval{e_t} & \text{gdy }\eval{v_1} = \eval{v_2} \\ 
    \eval{e_f} & \text{wpp}
    \end{matrix}\right.
  \end{aligned}\)
%% \item \(\eval{\opi{i}{a}{f}} = \opi{i}{a}{f}\) -- obliczenie z efektem ubocznym nie może poczynić postępu póki nie ma określonego działania,
\item \(\eval{\handle{\return v}{H}} = \eval{\return v}\) -- uchwyt nie wpływa na wartość obliczenia, które nie zawiera efektów ubocznych,
\item \(\eval{\handle{\opi{i}{a}{f}}{H}} = \eval{\handle{h_i \left[n / \eval{a},\, \kappa / f\right]}{H}} \), gdzie \(H~=~\{\ \hcase{op_i}{n\ \kappa}{h_i} \ \}\), a \(h_i\) nie ma wystąpień \(op_i\).
  
\end{itemize}

Zobaczmy jak zatem wygląda obliczenie ostatniego z powyższych przykładów:
\begin{equation}\begin{split}
  \eval{\handle{\opi{1}{2}{\fun{x}{\return{x + 1}}}}{\{\ \hcase{op_1}{n\ \kappa}{\kappa\ (2 \cdot n)} \ \}}} &= \\
  \eval{\handle{(\fun{x}{\return{x+1}}) (2 \cdot 2)}{\{\ \hcase{op_1}{n\ \kappa}{\kappa\ (2 \cdot n)} \ \}}} &= \\
  \eval{\handle{\return{4+1}}{\{\ \hcase{op_1}{n\ \kappa}{\kappa\ (2 \cdot n)} \ \}}} &= \\
  \eval{\return{4 + 1}} &= 5
\end{split}\end{equation}


\section{Równania, efekt porażki i modyfikowalny stan}

Do tego momentu, nie przyjmowaliśmy żadnych założeń na temat operacji powodujących efekty uboczne. Uchwyty mogły w związku z tym działać w sposób całkowicie dowolny. Ograniczymy się w tej dowolności i nałożymy warunki na uchwyty wybranych operacji. Przykładowo, ustalmy że dla operacji \(op_r\), uchwyty muszą być takie aby następujący warunek był spełniony:
\begin{align}
  \forall n\ \forall e.\ \eval{\handle{op_r(n, \fun{x}{e})}{H}} = n
\end{align}

%% \pagebreak

Zauważmy, że istnieje tylko jeden naturalny uchwyt spełniający tej warunek, jest nim \(H = \{\ \hcase{op_r}{n\ \kappa}{n} \ \}\). Co więcej, jego działanie łudząco przypomina konstrukcję wyjątków w popularnych językach programowania:

\begin{lstlisting}
  try {
    raise 5;
    // ...
  } catch (int n) {
    return n;
  }
\end{lstlisting}

Podobieństwo to jest w pełni zamierzone. Okazuje się że nasz język z jedną operacją oraz równaniem ma już moc wystarczającą do opisu konstrukcji, która w większości popularnych języków nie może zaistnieć z woli programisty, a zamiast tego musi być dostarczona przez twórcę języka.

Rozważmy kolejny przykład. Dla poprawienia czytelności, zrezygnujemy z oznaczeń \(op_i\) na operacje powodujące efekty, zamiast tego nadamy im znaczące nazwy: \(get\) oraz \(put\). Operacje te mają sygnatury \(get: U \rightarrow (\mathbb{Z} \rightarrow \mathbb{Z}) \rightarrow \mathbb{Z}\), \(put: \mathbb{Z} \rightarrow (U \rightarrow \mathbb{Z}) \rightarrow \mathbb{Z}\). Spróbujemy wyrazić działanie tych dwóch operacji by otrzymać modyfikowalną komórkę pamięci. Ustalamy równania:

\begin{itemize}
\item \(\forall e.\ \eval{get(u, \fun{\_}{get(u, \fun{x}{e})})} = \eval{get(u, \fun{x}{e})}\)

  kolejne odczyty z komórki bez jej modyfikowania dają takie same wyniki,
\item \(\forall e.\ \eval{get(u, \fun{n}{put(n, \fun{u}{e})})} = \eval{e}\)

  umieszczenie w komórce wartości, która już tam się znajduje, nie wpływa na wynik obliczenia,
\item \(\forall n.\ \forall f.\ \eval{put(n, \fun{u}{get(u, \fun{x}{f\ x})})} = \eval{f\ n}\)

  obliczenie które odczytuje wartość z komórki daje taki sam wyniki, jak gdyby miało wartość komórki podaną wprost jako argument,
\item \(\forall n_1.\ \forall n_2.\ \forall e.\ \eval{put(n_1, \fun{u}{put(n_2, \fun{u}{e})})} = \eval{put(n_2, \fun{u}{e})}\)

  komórka zachowuje się, jak gdyby pamiętała jedynie najnowszą włożoną do niej wartość.
\end{itemize}

Zauważmy, że choć nakładamy warunki na zewnętrzne skutki działania operacji \(get\) oraz \(put\), to w żaden sposób nie ograniczyliśmy swobody autora w implementacji uchwytów dla tych operacji. % W rozdziale 4 przyglądniemy się kilku przykładom uchwytów realizujących te operacje.

% Chyba jednak nie. Chociaż można dodać przykład, który agreguje historyczne wartości komórki i zwraca je w parze z wynikiem obliczenia.

\section{Poszukiwanie sukcesu}

%% sygnatura amb powinna mieć B zamiast Bool

Kolejnym rodzajem efektu ubocznego, który rozważymy w tym rozdziale jest niedeterminizm. Chcielibyśmy wyrażać obliczenia, w których pewne parametry mogą przyjmować wiele wartości, a ich dobór ma zostać dokonany tak by spełnić pewien określony warunek. Przykładowo, mamy trzy zmienne \(x,\, y\) oraz \(z\) i chcemy napisać program sprawdzający czy formuła \(\phi(x, y, z)\) jest spełnialna. W tym celu zdefiniujemy operację \(amb: U \rightarrow (Bool \rightarrow Bool) \rightarrow \mathit{Bool}\) związaną z efektem niedeterminizmu. Napiszmy obliczenie rozwiązujące nasz problem:
\begin{equation}\begin{split}
  \handle{
    &\op{amb}{u}{\fun{x}{
        \op{amb}{u}{\fun{y}{
            \op{amb}{u}{\fun{z}{
                \phi(x, y, z)
            }}
        }}
    }}\\
  }{ \{ \ &\hcase{amb}{u\ \kappa}{ \kappa\ (T) \ \mathbf{or} \ \kappa\ (F) } \ \} }
\end{split}\end{equation}

Gdy definiowaliśmy efekt wyjątku, obliczenie nie było kontynuowane. W przypadku niedeterminizmu kontynuujemy obliczenie dwukrotnie -- podstawiając za niedeterministycznie określoną zmienną wartości raz prawdy, raz fałszu -- w czytelny sposób sprawdzamy wszystkie możliwe wartościowania, a w konsekwencji określamy czy formuła jest spełnialna.

Możemy zauważyć, że gdybyśmy chcieli zamiast sprawdzania spełnialności, weryfikować czy formuła jest tautologią, wystarczy zmienić tylko jedno słowo -- zastąpić spójnik \(\mathbf{or}\) spójnikiem \(\mathbf{and}\) otrzymując nowy uchwyt:
\begin{equation}\begin{split}
  \handle{
    &\op{amb}{u}{\fun{x}{
        \op{amb}{u}{\fun{y}{
            \op{amb}{u}{\fun{z}{
                \phi(x, y, z)
            }}
        }}
    }}\\
  }{ \{ \ &\hcase{amb}{u\ \kappa}{ \kappa\ (T) \ \mathbf{and} \ \kappa\ (F) } \ \} }
\end{split}\end{equation}

Przedstawiona konstrukcja efektów, operacji i uchwytów tworzy dualny mechanizm w którym operacje są producentami efektów, a uchwyty ich konsumentami. Zabierając źródłom efektów ubocznych ich konkretne znaczenia semantyczne, lub nakładając na nie jedynie proste warunki wyrażone równaniami, otrzymaliśmy niezwykle silne narzędzie umożliwiające proste, deklaratywne oraz -- co najważniejsze, w kontraście do popularnych języków programowania -- samodzielne konstruowanie zaawansowanych efektów ubocznych.

\section{Dalsza lektura}

Rozdział ten miał na celu w lekki sposób wprowadzić idee, definicje i konstrukcje związane z efektami algebraicznymi i uchwytami, które będą fundamentem do zrozumienia ich wykorzystania w praktycznych przykładach oraz implementacji systemów kompilacji w dalszych rzodziałach. Czytelnicy zainteresowani głębszym poznaniem historii oraz rodowodu efektów algebraicznych i uchwytów mogą zapoznać się z następującymi materiałami:

\begin{itemize}
\item ,,An Introduction to Algebraic Effects and Handlers'' autorstwa Matija Pretnara \cite{pretnar2015introduction},
\item notatki oraz seria wykładów Andreja Bauera pt. ,,What is algebraic about algebraic effects and handlers?'' \cite{bauer2018algebraic} dostępne w formie tekstowej oraz nagrań wideo w serwisie YouTube,
\item prace Plotkina i Powera \cite{plotkin2001semantics, plotkin2002computational} oraz Plotkina i Pretnara \cite{plotkin2013handling} -- jeśli czytelnik chce poznać jedne z pierwszych wyników prowadzących do efektów algebraicznych oraz wykorzystania uchwytów,
\item społeczność skupiona wokół tematu efektów algebraicznych agreguje zasoby z nimi związane w repozytorium \cite{effectsbibliography} w serwisie GitHub.

\end{itemize}


\chapter{O systemach kompilacji (i ich klasyfikacji)}


\chapter{Efekty algebraiczne i uchwyty w~praktyce}

\section{Języki programowania z efektami algebraicznymi}

Zainteresowanie efektami algebraicznymi oraz uchwytami doprowadziło do powstania w ostatnich latach wielu bibliotek dla języków popularnych w środowisku akademickim i pasjonatów języków funkcyjnych -- Haskella (extensible-effects\footnote{\url{https://hackage.haskell.org/package/extensible-effects}},
fused-effects\footnote{\url{https://hackage.haskell.org/package/fused-effects}},
polysemy\footnote{\url{http://hackage.haskell.org/package/polysemy}}), Scali
(Effekt\footnote{\url{https://github.com/b-studios/scala-effekt}},
atnos-org/eff\footnote{\url{https://github.com/atnos-org/eff}})
i Idris (Effects \footnote{\url{https://www.idris-lang.org/docs/current/effects_doc/}}).

Związana z językiem OCaml jest inicjatywa ocaml-multicore\footnote{\url{https://github.com/ocaml-multicore/ocaml-multicore/wiki}}, której celem jest stworzenie implementacji OCamla ze wsparciem dla współbierzności oraz współdzielonej pamięci, a cel ten jest realizowane przez wykorzystanie konceptu efektów i uchwytów.

Badania nad efektami i uchwytami przyczyniły się także do powstania kilku eksperymentalnych języków programowania w których efekty i uchwyty są obywatelami pierwszej kategorii. Do języków tych należą:
\begin{itemize}
\item Eff\footnote{\url{https://www.eff-lang.org/}} -- powstający z inicjatywy Andreja Bauera and Matija Pretnara język o ML-podobnej składni,
\item Frank\footnote{\url{https://github.com/frank-lang/frank}} \cite{DBLP:journals/corr/LindleyMM16} -- pod przewodnictwem Sama Lindley'a, Conora McBride'a oraz Craiga McLaughlin'a projektowany z tęsknoty do ML'a, a jednocześnie upodobania do Haskell-owej dyscypliny,
\item Koka\footnote{\url{https://github.com/koka-lang/koka}} -- kierowany przez Daana Leijena z Microsoft projekt badawczy; Koka ma składnię inspirowaną JavaScriptem,
\item Helium\footnote{\url{https://bitbucket.org/pl-uwr/helium/src/master/}} \cite{biernacki2019abstracting} -- powstały w Instytucie Informatyki Uniwersytetu Wrocławskiego, z ML-podobnym systemem modułów i lekkimi naleciałościami z Haskella.
\end{itemize}

\section{Helium}

\lstset{style=helium}

Używając właśnie języka Helium, w tym rozdziale zobaczymy jak w praktyce wygląda programowanie z efektami algebraicznymi oraz uchwytami, zaś w następnym spróbujemy zaimplementować wyniki uzyskane w ,,Build systems {\`a} la carte'' \cite{mokhov2018build, mokhov2020build}. Po raz pierwszy Helium pojawia się w \cite{biernacki2019abstracting}, służąc za narzędzie do eksperymentowania i umożliwienia konstrukcji bardziej skomplikowanych przykładów oraz projektów w celu przetestowania efektów i uchwytów w praktyce.

Rozważmy przykład prostego programu napisanego w Helium, w którym definiujemy pomocniczą funkcję \textit{is\_negative} ustalającą czy liczba jest ujemna oraz {\textit question}, która pyta użytkownika o liczbę i informuje czy liczba ta jest ujemna:

\lstinputlisting{code_examples/syntax.he}

Sygnatura funkcji \textit{is\_negative} wyznaczona przez system typów Helium, to jak łatwo się domyśleć \(\mathit{Int \rightarrow Bool}\). Gdy jednak zapytamy środowisko uruchomieniowe o typ funkcji \textit{question} otrzymamy interesującą sygnaturę \(\mathit{Unit \rightarrow[IO] Unit}\). W Helium, informacje o efektach występujących w trakcie obliczania funkcji są umieszczona w sygnaturach funkcji w kwadratowych nawiasach. W przypadku funkcji \textit{question}, jej obliczenie powoduje wystąpienie efektu ubocznego związanego z mechanizmem wejścia/wyjścia. 

\begin{lstlisting}
printStr: String ->[IO] Unit
readInt: Unit ->[IO] Int
\end{lstlisting}

System inferencji typów wiedząc, że operacje we/wy są zadeklarowne z powyższymi sygnaturami wnioskuje, że skoro wystąpienia tychże operacji w kodzie \textit{question} nie są obsługiwane przez uchwyt, to efekt \textit{IO} wyjdzie poza tą funkcję.

Efekty \textit{IO} oraz \textit{RE} (runtime error) są szczególne, gdyż są dla nich zadeklarowane globalne uchwyty w bibliotece standardowej, jeśli efekt nie zostanie obsłużony i dotrze do poziomu środowiska uruchomieniowego, to ono zajmie się jego obsługą. W przypadku efektów wejścia/wyjścia środowisko skorzysta ze standardowego wejścia/wyjścia, zaś w przypadku wystąpienia efektu \textit{RE}, obliczenie zostanie przerwane ze stosownym komunikatem błędu.

\section{Przykłady implementacji uchwytów}

\subsection{Błąd}

Zaimplementujemy kilka efektów ubocznych, zaczynając od efektu błędu, wraz z uchwytamy dla nich. zaczniemy od efektu błędu. W Helium, efekt oraz powodujące go operacje definiuje się następująco:

\lstinputlisting{code_examples/error1__signature.he}

Stwórzmy funkcję podobną do \textit{question}, z tym że nie będzie ona lubić wartości ujemnych:

\lstinputlisting[firstline=7, lastline=18]{code_examples/error2__inline_abort.he}

Zdefiniowaliśmy efekt uboczny \textit{Error} wraz z operacją \textit{error}, która go powoduje. Operacja ta jest parametryzowana wartością typu \textit{Unit} oraz jej (możliwy) wynik to także wartość z \textit{Unit}. Definiujemy też funkcję \textit{main} w której wywołujemy \textit{no\_negatives\_question}, jednakże obliczenie wykonujemy w uchwycie w którym definiujemy co ma się wydarzyć, gdy w czasie obliczenia wystąpi efekt błędu spowodowany operacją \textit{error}. W tym przypadku mówimy, że skutkuje ono wypisaniem wiadomości na standardowe wyjście. Nie wznawiamy obliczenia, stąd wystąpienie błędu skutkuje zakończeniem nadzorowanego obliczenia. Jeśli uruchomimy teraz program i podamy ujemną liczbę, zakończy się on komunikatem zdefiniowanym w uchwycie, a tekst ,,Question finished'' nie zostanie wypisany -- zgodnie z oczekiwaniami, obliczenie \textit{no\_negatives\_question} nie zostało kontynuowane po wystąpieniu błędu.

Jeśli pewnego uchwytu zamierzamy używać wiele razy, możemy przypisać mu identyfikator -- uchwyty są wartościami w Helium:

\lstinputlisting[firstline=7, lastline=10]{code_examples/error3__reused_handler.he}

i zmodyfikować funkcję \textit{main} by z niego korzystać:

\lstinputlisting[firstline=12, lastline=13]{code_examples/error3__reused_handler.he}

Na potrzeby przykładu, możemy rozważyć spokojniejszy uchwyt dla wystąpień \textit{error}, który wypisze ostrzeżenie o wystąpieniu błędu ale będzie kontynuował obliczenie:

\lstinputlisting[firstline=7, lastline=10]{code_examples/error4__warn_not_abort.he}

Jeśli skorzystamy z tego uchwytu w programie, po wyświetleniu ostrzeżenia obliczenie \textit{no\_negatives\_question} zostanie wznowione i na ekranie zobaczymy komunikat ,,Question finished''. Specjalna funkcja \textit{resume}, dostępna w uchwycie reprezentuje kontynuację obliczenia, które zostało przerwane wystąpieniem operacji powodującej efekt uboczny.

\subsection{Niedeterminizm}

Powróćmy do problemu, który w rozdziale drugim był inspiracją do rozważania niedeterminizmu -- sprawdzanie czy formuła jest spełnialna oraz czy jest tautologią. Przedstawiliśmy wtedy uchwyty dla obu tych problemów w naszej notacji. Implementacja efektu niedeterminizmu, operacji \textit{amb} oraz uchwytów wraz z wykorzystaniem ich wygląda następująco:

\lstinputlisting[lastline=22]{code_examples/nondet1__simple.he}

Będziemy sprawdzać czy formuła wyrażona za pomocą funkcji \textit{formula1} jest spełnialna. W tym celu w funkcji \textit{main}, wewnątrz uchwytu, niedeterministycznie ustalamy wartości zmiennych \textit{x}, \textit{y}, \textit{z} po czym obliczamy wartość funkcji \textit{formula1}. Wartość obsługiwanego wyrażenia, którą przypisujemy do zmiennej \textit{ret}, jest następnie wykorzystana do wypisania komunikatu. Ponadto -- w celu demonstracji możliwości języka -- w uchwytach zamiast kontynuować obliczenie używając \textit{resume}, przypisujemy kontynuacji nazwę \textit{r}.

W Helium, uchwyty mogą posiadać przypadki nie tylko dla operacji związanych z jakimś efektem ale także dwa specjalne przypadki: \textit{return} oraz \textit{finally}. Pierwszy jest wykonywany, gdy obliczenie pod kontrolą uchwytu kończy się zwracając wynik, przypadek jako argument otrzymuje wynik obliczenia. Zaś \textit{finally} otrzymuje jako argument obliczenie obsługiwane przez uchwyt i jest uruchamiane na początku działania uchwytu. Domyślnie przypadki te są zaimplementowane jako:

\begin{lstlisting}
handler
| return x => x
| finally f => f
end
\end{lstlisting}

Możemy je jednak sprytnie wykorzystać. Przykładowo, zamiast tylko sprawdzać czy formuła jest spełnialna, możemy sprawdzić przy ilu wartościowaniach jest prawdziwa:

\lstinputlisting[firstline=5, lastline=16]{code_examples/nondet2__count_sats.he}

Gdy obliczenie się kończy, zamiast zwracać czy formuła jest spełniona zwracamy 1 albo 0, w zależności czy formuła przy aktualnym wartościowaniu jest spełniona. Gdy obsługujemy niedeterministyczny wybór, kontynuujemy obliczenie dla obu możliwych wartości boolowskich po czym dodajemy wyniki. Wykorzystując \textit{finally} możemy włączyć komunikat o liczbie wartościowań do uchwytu:

\lstinputlisting[firstline=5, lastline=17]{code_examples/nondet3__count_and_write_sats.he}

Tutaj wykorzystanie \textit{finally} jest lekkim nadużyciem, jak jednak za chwilę zobaczymy, konstrukcja ta jest bardzo przydatna.

\subsection{Modyfikowalny stan}

Rozważmy następujący przypadek dla \textit{return} w uchwycie:

\begin{lstlisting}
handler
(* ... *)
| return x => fn s => x
end
\end{lstlisting}

Wartość obliczenia, zamiast być jego wynikiem, jest funkcją. Co za tym idzie, w tym uchwycie kontynuacje nie będą funkcjami zwracającymi wartości lecz funkcje. W ten sposób możemy parametryzować dalsze obliczenia nie tylko wartościami zwracanymi przez operacje (zgodnie z ich sygnaturą) ale także wymyślonymi przez nas -- autorów uchwytu. Zauważmy jednak, że parametr ten nie jest widoczny w obsługiwanym obliczeniu, a jedynie w uchwycie. Co więcej, skoro wyniki obsługiwanego obliczenia jest teraz funkcją, a nie wartością to by użytkownik uchwytu nie zauważył niezgodności typów musimy funkcję tą uruchomić z jakimś parametrem -- tutaj właśnie przychodzi naturalny moment na wykorzystanie konstrukcji \textit{finally}.

Definiujemy efekt stanu z operacją jego oczytu oraz modyfikacji:

\lstinputlisting[lastline=3]{code_examples/state1__basics.he}

Efekt jak i operacje są polimorficzne ze względu na typ wartości stanu. Zdefiniujemy teraz standardowy uchwyt dla efektu stanu. Skorzystamy z faktu, że uchwyty są w Helium wartościami, stąd w szczególności mogą być wynikiem funkcji. Funkcja ta będzie u nas parametryzowana wartością początkową stanu:

\lstinputlisting[firstline=5, lastline=11]{code_examples/state1__basics.he}

Gdy obliczenie się kończy, zamiast wartość, zwracamy funkcję która ignoruje argument, a zwraca właściwy wynik obliczenia. Ten argument będzie bieżącą wartością stanu. W konsekwencji przypadki dla operacji też muszą być funkcjami. Dla \textit{put} nie musimy odczytywać aktualnej wartości stanu, stąd wartość tą ignorujemy. Obliczenie wznawiamy z wartością jednostkową. Jak jednak wiemy, wynikiem będzie nie zwykład wartość lecz funkcja, której u nas dajemy wartość stanu. Stąd podajemy jej nową wartość stanu, którą parametryzowana była operacja \textit{put}. W przypadku \textit{get} postępujemy podobnie -- jednak tym razem odczytamy argument funkcji i przekażemy go do kontynuacji. Niezmiennie kontynuacja zwraca funkcję, której przekażemy aktualną wartość stanu. Pozostaje rozstrzygnąć co zrobić w przypadku \textit{finally}. Skoro jednak przerobiliśmy obliczenie ze zwracającego wartość do takiego, które zwraca funkcję oczekującą wartości stanu, to możemy podać mu wartość początkową -- określoną przez użytkownika uchwytu.

Jeśli chcemy aby obliczenie zwracało nie tylko wartość wynikową ale także końcowy stan, wystarczy że zmodyfikujemy przypadek dla \textit{return}:

\lstinputlisting[firstline=5, lastline=11]{code_examples/state2__run_state.he}

Dzięki zdefiniowanemu efektowi ubocznemu, operacjom oraz uchwytom możemy teraz łatwo wykonywać obliczenia ze stanem:

\lstinputlisting[firstline=7, lastline=24]{code_examples/state3__example.he}

\subsection{Efekt rekursji}

W niektórych językach ML-podobnych (jak na przykład OCaml czy Helium) chcąc by w ciele definicji funkcji był widoczny jej identyfikator, trzeba zadeklarować ją używając słów kluczowych \textit{let rec}:

\lstinputlisting[firstline=2, lastline=5]{code_examples/rec1__rec.he}

Co ciekawe, dzięki własnym efektom i operacjom możemy tworzyć funkcje rekurencyjne, które nie używają jawnie rekursji:

\lstinputlisting[firstline=2, lastline=13]{code_examples/rec2__effect.he}

Konstrukcja \textit{\textbf{handle} `a \textbf{in}} służy doprecyzowaniu który efekt ma być obsłużony przez uchwyt -- jest przydatna w przypadku niejednoznaczności gdy używamy wielu instancji tego samego efektu lub dla ułatwienia rozumienia kodu.

Możemy w ten sposób definiować także funkcje wzajemnie rekurencyjne:

\lstinputlisting[firstline=5, lastline=23]{code_examples/rec3__mutual.he}

Utrzymujemy informację, która funkcja jest aktualnie wykonywana i gdy prosi o wywołanie rekurencyjne uruchamiamy obliczanie drugiej funkcji po czym wynik przekazujemy do kontynuacji.

% 
% Może jeszcze współbieżność kooperatywna?
% Byłaby naturalnym przedłużeniem wzajemnej rekursji
% ale zawierałaby trochę szumu w postaci kolejkowania zadań.
% 

\subsection{Wiele efektów na raz -- porażka i niedeterminizm}

Na koniec rozdziału, zobaczymy jak łatwo w Helium komponuje się efekty. Definiujemy efekty niedeterminzimu oraz porażki:

\lstinputlisting[firstline=1, lastline=5]{code_examples/fail_and_amb.he}

oraz bardzo proste uchwyty dla tych efektów:

\lstinputlisting[firstline=7, lastline=15]{code_examples/fail_and_amb.he}

Definiujemy teraz funkcję sprawdzającą czy otrzymana formuła z trzema zmiennymi wolnymi jest spełnialna:

\lstinputlisting[firstline=19, lastline=25]{code_examples/fail_and_amb.he}

Jeśli formuła przy ustalonym wartościowaniu nie jest spełniona powoduje efekt porażki. Zwróćmy uwagę w jakiej kolejności są umieszczone uchwyty -- niedeterminizmu na zewnątrz, zaś porażki wewnątrz. W ten sposób gdy wystąpi porażka, jej uchwyt zwróci fałsz, w wyniki czego nastąpi powrót do ostatniego punktu niedeterminizmu w którym jest jeszcze wybór. Dzięki temu wartość \textit{is\_sat f} jest równa fałszowi tylko gdy przy każdym wartościowaniu nastąpi porażka. Zobaczmy teraz funkcję sprawdzającą czy otrzymana formuła jest tautologią:

\lstinputlisting[firstline=27, lastline=33]{code_examples/fail_and_amb.he}

Tutaj uchwyt dla porażki znajduje się na zewnątrz -- wystąpienie porażki oznacza, że istnieje wartościowanie przy którym formuła nie jest prawdziwa, a w konsekwencji nie może być tautoligą. Możemy teraz napisać zgrabną funkcję, która wypisze nam czy \textit{formula1} jest spełnialna oraz czy jest tautologią:

\lstinputlisting[firstline=35, lastline=46]{code_examples/fail_and_amb.he}

Z łatwością napisaliśmy program, który korzysta z wielu efektów uboczonych jednocześnie, mimo że żaden z nich (ani uchwyty) nie wiedzą o istnieniu drugiego. Łączenie efektów jest bardzo proste, a kolejność w jakiej umieszczamy uchwyty umożliwia nam łatwe i czytelne definiowanie zachowania programu w przypadku wystąpienie któregokolwiek z efektów.

Dzięki językowi Helium, przyjrzeliśmy się z bliska efektom algebraicznym oraz uchwytom, zobaczyliśmy przykłady implementacji uchytów oraz rozwiązań prostych problemów. Jesteśmy gotowi do podjęcia próby zaimplementowania systemów kompilacji z użyciem efektów i uchwytów -- czego dokonamy w następnym rozdziale.


\chapter{Systemy kompilacji z użyciem efektów algebraicznych i~uchwytów}

\chapter{Podsumowanie i wnioski}

\ldots

%%%%% BIBLIOGRAFIA

\nocite{biernacki2017handle}
\nocite{biernacki2019abstracting}
\nocite{biernacki2019binders}
%% \nocite{pretnar2015introduction}
\nocite{pietersgeneralized}
\nocite{mcbride2012frank}

\bibliographystyle{abbrv}
\bibliography{bibliography}

\end{document}
