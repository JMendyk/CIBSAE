% Opcje klasy 'iithesis' opisane sa w komentarzach w pliku klasy. Za ich pomoca
% ustawia sie przede wszystkim jezyk i rodzaj (lic/inz/mgr) pracy, oraz czy na
% drugiej stronie pracy ma byc skladany wzor oswiadczenia o autorskim wykonaniu.
\documentclass[shortabstract]{iithesis}

\usepackage[utf8]{inputenc}

%%%%% DANE DO STRONY TYTUŁOWEJ
% Niezaleznie od jezyka pracy wybranego w opcjach klasy, tytul i streszczenie
% pracy nalezy podac zarowno w jezyku polskim, jak i angielskim.
% Pamietaj o madrym (zgodnym z logicznym rozbiorem zdania oraz estetyka) recznym
% zlamaniu wierszy w temacie pracy, zwlaszcza tego w jezyku pracy. Uzyj do tego
% polecenia \fmlinebreak.
\polishtitle    {Kwalifikacja i implementacja\fmlinebreak systemów kompilacji z użyciem\fmlinebreak efektów algebraicznych}
\englishtitle   {Categorization and implementation of Build Systems using algebraic effects}
\polishabstract {\ldots}
\englishabstract{\ldots}
% w pracach wielu autorow nazwiska mozna oddzielic poleceniem \and
\author         {Jakub Mendyk}
% w przypadku kilku promotorow, lub koniecznosci podania ich afiliacji, linie
% w ponizszym poleceniu mozna zlamac poleceniem \fmlinebreak
\advisor        {dr Filip Sieczkowski}
\date          {4 września 2020}                     % Data zlozenia pracy
% Dane do oswiadczenia o autorskim wykonaniu
\transcriptnum {301111}                     % Numer indeksu
\advisorgen    {dr. Filipa Sieczkowskiego} % Nazwisko promotora w dopelniaczu
%%%%%

%%%%% WLASNE DODATKOWE PAKIETY
%
%\usepackage{graphicx,listings,amsmath,amssymb,amsthm,amsfonts,tikz}
\usepackage{cite}
\usepackage{amsmath, amsfonts}
\usepackage{stmaryrd} % double brackets
\usepackage{listings}
%
%%%%% WŁASNE DEFINICJE I POLECENIA
%
%\theoremstyle{definition} \newtheorem{definition}{Definition}[chapter]
%\theoremstyle{remark} \newtheorem{remark}[definition]{Observation}
%\theoremstyle{plain} \newtheorem{theorem}[definition]{Theorem}
%\theoremstyle{plain} \newtheorem{lemma}[definition]{Lemma}
%\renewcommand \qedsymbol {\ensuremath{\square}}
% ...
%%%%%

\begin{document}

%%%%% POCZĄTEK ZASADNICZEGO TEKSTU PRACY

\chapter{Wprowadzenie}

\section{Problemy z efektami ubocznymi}

Programy komputerowe, dzięki możliwości interakcji z zewnętrznymi zasobami takimi jak nośniki pamięci, sieci komputerowe czy użytkownicy oprogramowania mogą robić istotnie więcej niż tylko zadane wcześniej obliczenia. W ten sposób przebieg programu i jego wynik staje się jednak zależny od tegoż świata zewnętrznego, a sam program nie jest tylko serią czystych obliczeń ale także towarzyszących im efektów ubocznych.

Efekty uboczne powodują jednak, że rozumowanie i wnioskowanie o sposobie oraz prawidłowości działania programów staje się znacznie trudniejsze, a w konsekwencji ogranicza ich modularność i prowadzi do częstszych pomyłek ze strony autorów. Chcąc tego uniknąć, dąży się do wydzielania w programie jak największej części, która składa się z czystych obliczeń. Jednak to, czy jakiś moduł oprogramowania wykonuje obliczenia z efektami ubocznymi nie koniecznie jest jasne i często musimy zaufać autorowi, że w istocie tak jest.

\section{Radzenie sobie z efektami ubocznymi}

Jednym z rozwiązań tego problemu, jest zawarcie informacji o posiadaniu efektów ubocznych w systemie typów. Możemy skorzystać wtedy z inferencji i weryfikacji typów do automatycznej identyfikacji modułów zawierających efekty uboczne. Programista może łatwo wyczytać z sygnatury funkcji, że w czasie jej działania występują efekty uboczne. Znanym przykładem takiego rozwiązania jest wykorzystanie monad w języku programowania Haskell. Niestety, jednoczesne użytkowanie dwóch niezależnych zasób reprezentowanych przez różne monady nie jest łatwe i wymaga dodatkowych struktur, takich jak transformery monad, które niosą ze sobą dodatkowe wyzwania. Problem modularności został jedynie przesunięty w inny obszar.

Nowym, konkurencyjnym podejściem do ujarzmienia efektów ubocznych przez wykorzystanie systemu typów są efekty algebraiczne z uchwytami. Powierzchownie, zdają się być podobne do konstrukcji obsługi wyjątków w językach programowania lub wywołań systemowych w systemach operacyjnych. Dzięki rozdziałowi między definicjami operacji związanych z efektami ubocznymi, a ich semantyką oraz interesującemu zastosowaniu kontynuacji, dają łatwość myślenia i wnioskowania o programach ich używających. Ponadto, w przeciwieństwie do monad, można je bezproblemowo składać.

\section{Systemy kompilacji}

Przykładem programów, których głównym zadaniem jest interakcja z zewnętrznymi zasobami są systemy kompilacji, w których użytkownik opisuje proces wytwarzania wyniku jako zbiór wzajemnie-zależnych zadań wraz z informacją jak zadania mają być wykonywane w oparciu o wyniki innych zadań, zaś system jest odpowiedzialny za poprawne uporządkowanie i wykonanie otrzymanych zadań. Ponadto, od systemu kompilacji oczekujemy, że będzie śledził zmiany w danych wejściowych i -- gdy poproszony o aktualizację wyników -- obliczał ponownie jedynie zadania, których wartości ulegną zmianie. Przykładami takich systemów są Make oraz -- co może wydawać się zaskakujące -- programu biurowe służące do edycji arkuszy kalkulacyjnych (np. popularny Excel).

%% Finalnie, w czasie działania system agreguje wyniki obliczeń (np. na dysku lub w pamięci) i decyduje, która zadania powinny być obliczone ponownie -- np. system Make lub popularne narzędzie biurowe Excel.

W publikacjach pod tytułem ,,Build systems {\`a} la carte'' \cite{mokhov2018build, mokhov2020build}, autorzy przedstawiają sposób klasyfikacji systemów kompilacji w oparciu o to jak determinują one kolejność w jakiej zadania zostaną obliczone oraz jak wyznaczają, które z zadań wymagają ponownego obliczenia. Uzyskana klasyfikacja prowadzi autorów do skonstruowania platformy umożliwiającej definiowanie systemów kompilacji o oczekiwanych właściwościach. Platforma ta okazuje się być łatwa w implementacji w języku Haskell, a klasy typów Applicative oraz Monad odpowiadać mocy języka opisywania zależności między zadaniami do obliczenia.

\section{O tej pracy}

Celem tej pracy jest zapoznanie czytelnika, który miał dotychczas kontakt z językiem Haskell oraz podstawami języków funkcyjny, z nowatorskim rozwiązaniem jakim są efekty algebraiczne oraz zademonstrowanie -- idąc śladami Mokhov i innych \cite{mokhov2018build} -- implementacji systemów kompilacji z wykorzystaniem efektów algebraicznych i uchwytów w języku programowania Helium. Jak się okazuje, wykorzystanie tych narzędzi daje schludną implementację ale także prowadzi do problemów w implementacji systemów o pewnym sposobie determinowania zależności między zadaniami.

W rozdziale drugim wprowadzony zostaje prosty model obliczeń wykorzystujący efekty algebraiczne i uchwyty. Zostaje przedstawionych kilka przykładów reprezentacji standardowych efektów ubocznych w opisanym modelu.



\chapter{O efektach algebraicznych teoretycznie}

Wprowadzimy notację służącą opisowi prostych obliczeń, która pomoże nam -- bez zanurzania się głęboko w ich rodowód matematyczny -- zrozumieć jak prostym, a jednocześnie fascynującym tworem są efekty algebraiczne i uchwyty. Następnie przyjrzymy się, jak możemy zapisać popularne przykłady efektów ubocznych używając naszej notacji. Na koniec, czytelnikowi zostaną polecone zasoby do dalszej lektury, które rozszerzają opis z tego rozdziału.

\section{Notacja}

\newcommand{\return}[1]{\mathbf{return}\ #1}
\newcommand{\op}[3]{#1(#2, #3)}
\newcommand{\opi}[3]{\op{op_{#1}}{#2}{#3}}
\newcommand{\handle}[2]{\mathbf{handle}\ #1\ \mathbf{with}\ #2}
\newcommand{\hcase}[3]{#1\ #2\ \Rightarrow\ #3}
\newcommand{\fun}[2]{\lambda #1.\ #2}
\newcommand{\eval}[1]{\llbracket\, #1\, \rrbracket}
\newcommand{\cond}[3]{\mathbf{if}\ #1\ \mathbf{then}\ #2\ \mathbf{else}\ #3}

Będziemy rozważać obliczenia nad wartościami następujących trzech typów:
\begin{itemize}
\item boolowskim \(B\) -- z wartościami \(T\) i \(F\) oraz standardowymi spójnikami logicznymi,
\item liczb całkowitych \(\mathbb{Z}\) -- wraz z ich relacją równości oraz podstawowymi działaniami arytmetycznymi,
\item typem jednostkowym \(U\) -- zamieszkałym przez pojedynczą wartość \(u\),
\item oraz pary tychże typów.
\end{itemize}

% 
% Przemyślenia:
% 1. Pozbyć się "return v"
% 2. Dodać przypadek "return x" do zbioru w uchwycie, zdefiniować jego działanie
%    Można by wtedy rozszerzyć przykłady o zmianę wartości wynikowej
% 

Nasz model składać się będzie z wyrażeń:
\begin{itemize}
\item \(\return{v}\) -- gdzie \(v\) jest wyrażeniem boolowskim lub arytmetycznym,
\item \(\cond{v_1 = v_2}{e_t}{e_f}\) -- wyrażenie warunkowe, gdzie \(v_1 = v_2\) jest pytaniem o równość wartości dwóch wyrażeń arytmetycznych,
\item abstrakcyjnych operacji oznaczanych \(\{op_i\}_{i \in I}\) -- powodujących wystąpienie efektów ubocznych -- których działanie nie jest nam znane, zaś ich sygnatury to \(op_i: \mathbb{Z} \rightarrow (\mathbb{Z} \rightarrow \mathbb{Z}) \rightarrow \mathbb{Z}\). Wyrażenie~\(\opi{i}{n}{\kappa}\) opisuje operację z argumentami \(n\) oraz dalszą częścią obliczenia \(\kappa\) parametryzowaną wynikiem operacji, które \textit{może (nie musi)} zostać wykonane po jej wystąpieniu,
\item uchwytów, czyli wyrażeń postaci \(\handle{e}{\{\ \hcase{op_i}{n\ \kappa}{h_i}\ \}_{i \in I}}\), gdzie \(e\) to inne wyrażenie; uchwyt definiuje działanie (dotychczas abstrakcyjnych) efektów ubocznych. 
\end{itemize}

Przykładowymi obliczeniami w naszej notacji są więc:
\begin{equation}
\begin{gathered}
  \return{0},\quad\return{2 + 2},\quad \opi{1}{2}{\fun{x}{\return{x + 1}}} \\
  \handle{\opi{1}{2}{\fun{x}{\return{x + 1}}}}{\{\ \hcase{op_1}{n\ \kappa}{\kappa\ (2 \cdot n)} \ \}}
\end{gathered}
\end{equation}

Dla czytelności, pisząc w uchwycie zbiór który nie przebiega wszystkich operacji, przyjmujemy że uchwyt nie definiuje działania operacji; równoważnie, zbiór wzbogacamy o element: \(\hcase{op_i}{n\ \kappa}{op_i(n, \kappa)}\).

Obliczanie wartości wyrażenia przebiega następująco:
\begin{itemize}
\item \(\eval{\return v} = v\) -- wartością \(\mathbf{return}\) jest wartość wyrażenia arytmetycznego,
\item \(\eval{(\fun{x}{e})\ y} = \eval{e \left[x / \eval{y}\right]}\) -- aplikacja argumentu do funkcji,
\item
  \(\begin{aligned}[t]
    \eval{\cond{v_1 = v_2}{e_t}{e_f}} = \left\{\begin{matrix}
    \eval{e_t} & \text{gdy }\eval{v_1} = \eval{v_2} \\ 
    \eval{e_f} & \text{wpp}
    \end{matrix}\right.
  \end{aligned}\)
%% \item \(\eval{\opi{i}{a}{f}} = \opi{i}{a}{f}\) -- obliczenie z efektem ubocznym nie może poczynić postępu póki nie ma określonego działania,
\item \(\eval{\handle{\return v}{H}} = \eval{\return v}\) -- uchwyt nie wpływa na wartość obliczenia, które nie zawiera efektów ubocznych,
\item \(\eval{\handle{\opi{i}{a}{f}}{H}} = \eval{\handle{h_i \left[n / \eval{a},\, \kappa / f\right]}{H}} \), gdzie \(H~=~\{\ \hcase{op_i}{n\ \kappa}{h_i} \ \}\), a \(h_i\) nie ma wystąpień \(op_i\).
  
\end{itemize}

Zobaczmy jak zatem wygląda obliczenie ostatniego z powyższych przykładów:
\begin{equation}\begin{split}
  \eval{\handle{\opi{1}{2}{\fun{x}{\return{x + 1}}}}{\{\ \hcase{op_1}{n\ \kappa}{\kappa\ (2 \cdot n)} \ \}}} &= \\
  \eval{\handle{(\fun{x}{\return{x+1}}) (2 \cdot 2)}{\{\ \hcase{op_1}{n\ \kappa}{\kappa\ (2 \cdot n)} \ \}}} &= \\
  \eval{\handle{\return{4+1}}{\{\ \hcase{op_1}{n\ \kappa}{\kappa\ (2 \cdot n)} \ \}}} &= \\
  \eval{\return{4 + 1}} &= 5
\end{split}\end{equation}


\section{Równania, efekt porażki i modyfikowalny stan}

Do tego momentu, nie przyjmowaliśmy żadnych założeń na temat operacji powodujących efekty uboczne. Uchwyty mogły w związku z tym działać w sposób całkowicie dowolny. Ograniczymy się w tej dowolności i nałożymy warunki na uchwyty wybranych operacji. Przykładowo, ustalmy że dla operacji \(op_r\), uchwyty muszą być takie aby następujący warunek był spełniony:
\begin{align}
  \forall n\ \forall e.\ \eval{\handle{op_r(n, \fun{x}{e})}{H}} = n
\end{align}

%% \pagebreak

Zauważmy, że istnieje tylko jeden naturalny uchwyt spełniający tej warunek, jest nim \(H = \{\ \hcase{op_r}{n\ \kappa}{n} \ \}\). Co więcej, jego działanie łudząco przypomina konstrukcję wyjątków w popularnych językach programowania:

\begin{lstlisting}
  try {
    raise 5;
    // ...
  } catch (int n) {
    return n;
  }
\end{lstlisting}

Podobieństwo to jest w pełni zamierzone. Okazuje się że nasz język z jedną operacją oraz równaniem ma już moc wystarczającą do opisu konstrukcji, która w większości popularnych języków nie może zaistnieć z woli programisty, a zamiast tego musi być dostarczona przez twórcę języka.

Rozważmy kolejny przykład. Dla poprawienia czytelności, zrezygnujemy z oznaczeń \(op_i\) na operacje powodujące efekty, zamiast tego nadamy im znaczące nazwy: \(get\) oraz \(put\). Spróbujemy wyrazić działanie tych dwóch operacji by otrzymać modyfikowalną komórkę pamięci. Ustalmy też bardziej naturalny sygnatury operacji -- \(get: U \rightarrow \mathbb{Z}\), \(put: \mathbb{Z} \rightarrow U\). Ustalamy równania:

\begin{itemize}
\item \(\forall e.\ \eval{get(u, \fun{\_}{get(u, \fun{x}{e})})} = \eval{get(u, \fun{x}{e})}\)

  kolejne odczyty z komórki bez jej modyfikowania dają takie same wyniki,
\item \(\forall e.\ \eval{get(u, \fun{n}{put(n, \fun{u}{e})})} = \eval{e}\)

  umieszczenie w komórce wartości która już tam się znajduje nie wpływa na wynik obliczenia,
\item \(\forall n.\ \forall f.\ \eval{put(n, \fun{u}{get(u, \fun{x}{f\ x})})} = \eval{f\ n}\)

  obliczenie które odczytuje wartość z komórki daje taki sam wyniki, jak gdyby miało wartość komórki podaną wprost jako argument,
\item \(\forall n_1.\ \forall n_2.\ \forall e.\ \eval{put(n_1, \fun{u}{put(n_2, \fun{u}{e})})} = \eval{put(n_2, \fun{u}{e})}\)

  komórka zachowuje się, jak gdyby pamiętała jedynie najnowszą włożoną do niej wartość.
\end{itemize}

Zauważmy, że choć nakładamy warunki na zewnętrzne skutki działania operacji \(get\) oraz \(put\), to w żaden sposób nie ograniczyliśmy swobody autora w implementacji uchwytów dla tych operacji. W rozdziale 4 przyglądniemy się kilku przykładom uchwytów realizujących te operacje.

\section{Poszukiwanie sukcesu}

Kolejnym rodzajem efektu ubocznego, który rozważymy w tym rozdziale jest niedeterminizm. Chcielibyśmy wyrażać obliczenia, w których pewne parametry mogą przyjmować wiele wartości i chcemy określić dobór wartości spełniający jakiś określony warunek. Przykładowo, mamy trzy zmienne \(x,\, y\) oraz \(z\) i chcemy napisać program sprawdzający czy formuła \(\phi(x, y, z)\) jest spełnialna. W tym celu zdefiniujemy operację \(amb: U \rightarrow \mathit{Bool}\) związaną z efektem niedeterminizmu. Napiszmy obliczenie rozwiązujące nasz problem:
\begin{equation}\begin{split}
  \handle{
    &\op{amb}{u}{\fun{x}{
        \op{amb}{u}{\fun{y}{
            \op{amb}{u}{\fun{z}{
                \phi(x, y, z)
            }}
        }}
    }}\\
  }{ \{ \ &\hcase{amb}{u\ \kappa}{ \kappa\ (T) \ \mathbf{or} \ \kappa\ (F) } \ \} }
\end{split}\end{equation}

Gdy definiowaliśmy efekt wyjątku, obliczenie nie było kontynuowane. W przypadku niedeterminizmu kontynuujemy obliczenie dwukrotnie -- podstawiając za niedeterministycznie określoną zmienną wartości raz prawdy, raz fałszu -- w ten sposób możemy w czytelny sposób sprawdzić wszystkie możliwe wartościowania, a w konsekwencji określić czy formuła jest spełnialna.

Możemy zauważyć, że gdybyśmy chcieli zamiast sprawdzania spełnialności, weryfikować czy formuła jest tautologią, wystarczy zmienić tylko jedno słowo -- spójnik \(\mathbf{or}\) na \(\mathbf{and}\) otrzymując nowy uchwyt:
\begin{equation}\begin{split}
  \handle{
    &\op{amb}{u}{\fun{x}{
        \op{amb}{u}{\fun{y}{
            \op{amb}{u}{\fun{z}{
                \phi(x, y, z)
            }}
        }}
    }}\\
  }{ \{ \ &\hcase{amb}{u\ \kappa}{ \kappa\ (T) \ \mathbf{and} \ \kappa\ (F) } \ \} }
\end{split}\end{equation}

Przedstawiona konstrukcja tworzy dualny mechanizm w którym operacje są producentami efektów, a uchwyty ich konsumentami. Zabierając operacjom powodującym efekty uboczne ich konkretne znaczenia semantyczne, lub nakładając na nie jedynie proste warunki wyrażone równaniami, otrzymaliśmy niezwykle silne narzędzie umożliwiające proste, deklaratywne oraz -- co najważniejsze, w kontraście do popularnych języków programowania -- samodzielne konstruowanie zaawansowanych efektów ubocznych.

\section{Dalsza lektura}

Rozdział ten miał na celu w lekki sposób wprowadzić idee, definicje i konstrukcje związane z efektami algebraicznymi i uchwytami które będą fundamentem do zrozumienia ich wykorzystania w praktycznych przykładach oraz implementacji systemów kompilacji. Czytelnicy zainteresowani głębszym poznaniem historii oraz rodowodu efektów algebraicznych i uchwytów mogą zapoznać się z następującymi materiałami:

\begin{itemize}
\item ,,An Introduction to Algebraic Effects and Handlers'' autorstwa Matija Pretnara \cite{pretnar2015introduction},
\item notatki oraz seria wykładów Andreja Bauera pt. ,,What is algebraic about algebraic effects and handlers?'' \cite{bauer2018algebraic} dostępne w formie tekstowej oraz nagrań wideo w serwisie YouTube,
\item Prace Plotkina i Powera \cite{plotkin2001semantics, plotkin2002computational} oraz Plotkina i Pretnara \cite{plotkin2013handling} jeśli czytelnik chce poznać jedne z pierwszych wyników prowadzących do odkrycia efektów algebraicznych oraz wykorzystania uchwytów,
\item społeczność skupiona wokół tematu efektów algebraicznych agreguje zasoby z nimi związane w repozytorium \cite{effectsbibliography} w serwisie GitHub.

\end{itemize}

\chapter{O systemach kompilacji (i ich klasyfikacji)}

\chapter{Efekty algebraiczne i uchwyty w~praktyce}

\chapter{Systemy kompilacji z użyciem efektów algebraicznych i~uchwytów}

\chapter{Podsumowanie i wnioski}

\ldots

%%%%% BIBLIOGRAFIA

\nocite{biernacki2017handle}
\nocite{biernacki2019abstracting}
\nocite{biernacki2019binders}
\nocite{pretnar2015introduction}

\bibliographystyle{abbrv}
\bibliography{bibliography}

\end{document}
