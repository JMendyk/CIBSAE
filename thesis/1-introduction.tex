\chapter{Wprowadzenie}

\section{Problemy z efektami ubocznymi}

Programy komputerowe, dzięki możliwości interakcji z zewnętrznymi zasobami takimi jak nośniki pamięci, sieci komputerowe czy użytkownicy oprogramowania, mogą robić istotnie więcej, niż tylko zadane wcześniej obliczenia. W ten sposób przebieg programu i jego wynik staje się jednak zależny od tegoż świata zewnętrznego, a sam program nie jest tylko serią czystych obliczeń, ale także towarzyszących im efektów ubocznych.

Efekty uboczne powodują też, że rozumowanie i wnioskowanie o sposobie oraz prawidłowości działania programów staje się znacznie trudniejsze, a w konsekwencji ogranicza ich modularność i prowadzi do częstszych pomyłek ze strony autorów. Chcąc tego uniknąć, dąży się do wydzielania w programie jak największej części, która składa się z czystych obliczeń. Jednak to, czy jakiś moduł oprogramowania wykonuje obliczenia bez efektów ubocznych niekoniecznie jest jasne i często musimy zaufać autorowi, że w istocie tak jest.

\section{Radzenie sobie z efektami ubocznymi}

Jednym z rozwiązań tego problemu jest zawarcie informacji o posiadaniu efektów ubocznych w systemie typów. Możemy skorzystać wtedy z mechanizmów inferencji i weryfikacji typów do automatycznej identyfikacji funkcji, które nie są czyste -- dzięki temu programista może łatwo wyczytać z sygnatury funkcji, które z efektów występują w czasie jej działania. Znanym przykładem umieszczenia efektów w typach jest wykorzystanie monad w języku programowania Haskell. Niestety, jednoczesne użytkowanie dwóch niezależnych zasobów reprezentowanych przez różne monady nie jest łatwe i wymaga dodatkowych struktur, takich jak transformery monad, które niosą ze sobą dodatkowe wyzwania -- problem modularności został jedynie przesunięty w inny obszar.

Nowym, konkurencyjnym podejściem do ujarzmienia efektów ubocznych przez wykorzystanie systemu typów są efekty algebraiczne z uchwytami. Powierzchownie, zdają się być podobne do konstrukcji obsługi wyjątków w językach programowania lub wywołań systemowych w systemach operacyjnych. Dzięki rozdziałowi między definicjami operacji związanych z efektami ubocznymi a ich semantyką oraz interesującemu zastosowaniu kontynuacji, dają łatwość myślenia i wnioskowania o programach ich używających. Ponadto, w przeciwieństwie do monad, można bezproblemowo korzystać z wielu z nich jednocześnie.

\section{Systemy kompilacji}

Przykładem programów, których głównym zadaniem jest interakcja z zewnętrznymi zasobami są systemy kompilacji, w których użytkownik opisuje proces wytwarzania wyniku jako zbiór wzajemnie zależnych zadań wraz z informacją jak mają być one wykonywane w oparciu o wyniki innych zadań, zaś system jest odpowiedzialny za ich poprawne uporządkowanie i wykonanie. Ponadto, od systemu kompilacji oczekujemy, że będzie śledził zmiany w danych wejściowych i -- gdy poproszony o aktualizację wyników -- obliczał ponownie jedynie zadania, których wartości ulegną zmianie. Przykładami systemów kompilacji są Make oraz -- co może wydawać się zaskakujące -- programy biurowe służące do edycji arkuszy kalkulacyjnych (np. popularny Excel).

W publikacjach pod tytułem \BSaLC{} \cite{mokhov2018build, mokhov2020build}, autorzy przedstawiają sposób klasyfikacji systemów kompilacji w oparciu o to, jak determinują one kolejność w jakiej zadania zostaną obliczone oraz jak wyznaczają, które z zadań wymagają ponownego obliczenia. Uzyskana klasyfikacja prowadzi autorów do skonstruowania platformy umożliwiającej definiowanie systemów kompilacji o oczekiwanych właściwościach. Platforma ta okazuje się być łatwa w implementacji w języku Haskell, a klasy typów Applicative oraz Monad odpowiadać mocy języka opisywania zależności między zadaniami do obliczenia.

\section{O tej pracy}

Celem tej pracy jest zapoznanie czytelnika, który miał dotychczas kontakt z językiem Haskell oraz podstawami języków funkcyjnych, z nowatorskim rozwiązaniem jakim są efekty algebraiczne oraz zademonstrowanie -- idąc śladami Mokhov'a i innych -- implementacji systemów kompilacji z wykorzystaniem efektów algebraicznych i uchwytów w języku programowania Helium. W konsekwencji możliwe jest porównanie obu implementacji oraz zaobserwowanie jak wygląda programowanie z efektami algebraicznymi i uchwytami.

W rozdziale drugim wprowadzony zostaje prosty i nieformalny model obliczeń wykorzystujący efekty algebraiczne i uchwyty. Zostaje przedstawionych kilka przykładów reprezentacji standardowych efektów ubocznych w opisanym modelu.

Celem rozdziału trzeciego jest wprowadzenie do \BSaLC{}, opisanie obserwacji poczynionych przez autorów i przedstawienie abstrakcji systemów kompilacji oraz ich konsekwencji. Treść źródłowego artykułu jest opisana w sposób dostateczny, aby zrozumieć implementacje systemów z wykorzystaniem efektów i uchwytów przedstawione w rozdziale piątym. Zachęca się przy tym czytelnika do samodzielnego zapoznania się z całą treścią publikacji Mokhov'a i innych. Jest to pozycja interesująca i łatwa w lekturze.

Rozdział czwarty rozpoczyna się zapoznaniem czytelnika z istniejącymi językami oraz bibliotekami umożliwiającymi programowanie z efektami i uchwytami. Następnie omówiony jest język Helium oraz przykładowe problemy wraz z programami je rozwiązującymi z użyciem efektów i uchwytów. Zademonstrowana jest ponadto łatwość wykorzystywania wielu efektów jednocześnie -- w bardziej przystępnej formie niż w przypadku monad w Haskellu.

Zwieńczeniem pracy jest rozdział piąty, w którym przedstawiona jest implementacja planistów, rekompilatorów oraz systemów kompilacji w sposób inspirowany wynikami \BSaLC{}, jednak używając języka z efektami algebraicznymi i uchwytami. Przedstawione są różnice między abstrakcyjnymi typami od których wyprowadza się implementację oraz w jaki sposób efekty i uchwyty wpływają na formę wyniku. Ponadto, pominięta zostaje implementacja jednego z planistów z wytłumaczeniem dlaczego ma to miejsce.
