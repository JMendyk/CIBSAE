
\newcommand{\inl}[1]{\lstinline[style=Haleff-inl]{#1}}

\chapter{Algebraic effects and handlers in practise}

\section{Programming languages with algebraic effects}

Interest in the subject of algebraic effects and handlers lead in the recent years to creation of multiple libraries for languages popular in the academic and functional language enthusiast communities -- for Haskell (extensible-effects\footnote{\url{https://hackage.haskell.org/package/extensible-effects}},
fused-effects\footnote{\url{https://hackage.haskell.org/package/fused-effects}},
polysemy\footnote{\url{http://hackage.haskell.org/package/polysemy}}), Scala
(Effekt\footnote{\url{https://github.com/b-studios/scala-effekt}},
atnos-org/eff\footnote{\url{https://github.com/atnos-org/eff}})
and Idris (Effects \footnote{\url{https://www.idris-lang.org/docs/current/effects_doc/}}).

Related to the OCaml language, is an initiative ocaml-multicore\footnote{\url{https://github.com/ocaml-multicore/ocaml-multicore/wiki}}, whose members aim to create an implementation of OCaml that supports concurrency and shared memory by expressing these concepts using effects and handlers.

Research regarding effects and handlers also lead to introduction of a few experimental programming languages in which effects and handlers are first-class citizens. To name a few examples:
\begin{itemize}
\item Eff\footnote{\url{https://www.eff-lang.org/}} -- created by Andrej Bauer and Matija Pretnar language with ML-like syntax,
\item Frank\footnote{\url{https://github.com/frank-lang/frank}} \cite{DBLP:journals/corr/LindleyMM16} -- initiated by Sam Lindley, Conor McBride and Craig McLaughlin, motivated by longing for ML and penchant for Haskell's discipline,
\item Koka\footnote{\url{https://github.com/koka-lang/koka}} -- lead by Daan Leijen from Microsoft research project; Koka has syntax inspired by JavaScript,
\item Helium\footnote{\url{https://bitbucket.org/pl-uwr/helium/src/master/}} \cite{biernacki2019abstracting} -- created at University of Wroc\l{}aw's Institue of Computer Science, with ML-like module system and small traces of Haskell.
\end{itemize}

\section{Helium}

\lstset{language=Haleff, showstringspaces=false}

Using the Helium language we will see in practise how programming with algebraic effects and handlers looks like, and in the next chapter we will try to implement results from \BSaLC{} \cite{mokhov2018build, mokhov2020build}. The first time Helium appears in \cite{biernacki2019abstracting}, by serving as a tool for experimenting and constructing more complicated examples and projects for testing effects and handlers in practise.

Let us consider an example of a simple program written in Helium, in which we define a helper function \inl{is_negative} telling if a given number is negative and a function \inl{question}, which asks the user for a number and informs if that number is negative:

\lstinputlisting{code_examples/syntax.he}

It is easy to guess that signature of the function \inl{is_negative} as determined by the Helium's type system is \inl{Int -> Bool}. However, when we ask the runtime environment about the type of function \inl{question} we will receive an interesting signature \inl{Unit ->[IO] Unit}. In Helium information about effects occurring during computation of a function are included in function signatures in square brackets. In the case of \inl{question}, its execution causes occurrence of a side effect related to input/output.

\begin{lstlisting}
printStr: String ->[IO] Unit
readInt: Unit ->[IO] Int
\end{lstlisting}

Type inference system knowing that i/o-operations are declared with above signatures and they are not wrapped by any handler concludes that \inl{IO} effect will leak from the \inl{question} function.

Effects \inl{IO} and \inl{RE} (runtime error) are special, since for them there global handlers declared in the standard library. If the effect instance is not handled and surfaces to the runtime environment, the global handler will take care of it. For effect \inl{IO} environment uses standard input/output, while for \inl{RE} computation will be halted with appropriate error message.

\section{Examples of handler implementations}

\subsection{Error}

Let us implement a couple of side effects, starting from error effect, including handlers for them. In Helium effect and operations that cause it are defined like this:

\lstinputlisting{code_examples/error1__signature.he}

We will create function similar to \inl{question}, however it won't ,,like'' negative numbers:

\lstinputlisting[firstline=7, lastline=18]{code_examples/error2__inline_abort.he}

We defined an \inl{Error} side effect with effectful operation \inl{error}. This operation is parameterised by type \inl{Unit} and its (possible) value is also of type \inl{Unit}. Moreover, we define function \inl{main}, which calls \inl{no\_negatives\_question}. However, the computation is performed inside a handler, which defines what should happen if error effect will occur causes by \inl{error} operation. In this case, we will write a message to standard output. We will not be resuming the computation, thus the error will stop the handled computation. If we execute our program and input a negative number, the program will halt with message as defined in handler, and ,,Question finished'' will not be displayed. As expected \inl{no\_negatives\_question} was not resumed after error occurred.

If we want to use the same handler multiple times, we can assign it to an identifier -- in Helium handlers are values:

\lstinputlisting[firstline=7, lastline=10]{code_examples/error3__reused_handler.he}

let us modify function \inl{main} to use the defined handler:

\lstinputlisting[firstline=12, lastline=13]{code_examples/error3__reused_handler.he}

For example's sake, let us consider more ,,peaceful'' handler for \inl{error}, that will display a warning but will not stop the computation:

\lstinputlisting[firstline=7, lastline=10]{code_examples/error4__warn_not_abort.he}

If we use this handler in our program, after displaying a warning the execution of \inl{no\_negatives\_question} will resume and message ,,Question finished'' will be outputted. Special function \inl{resume}, available in handler corresponds to continuation of computation which has been paused when effectful operation was encountered.

\subsection{Nondeterminism}

Let us go back to the problem from chapter 2 which was an inspiration for considering nondeterminism -- checking if logical formula is satisfiable and if is a tautology. We introduced two handlers for both instances of our problems. Implementation of nondeterminism's effect, operation \inl{amb} and handlers along with their usages is as follows:

\lstinputlisting[linerange={1-7}]{code_examples/nondet1__simple.he}
\pagebreak
\lstinputlisting[linerange={8-22}]{code_examples/nondet1__simple.he}

We will be verifying if formula represented by a function \inl{formula1} is satisfied. To do so, in \inl{main} -- inside the handler -- we nondeterministically set values of variables \inl{x}, \inl{y}, \inl{z}, and compute \inl{formula1}. Value of handled expression, which we assign to \inl{ret} is used to display the message. Moreover, to demonstrate capabilities of the language, instead of using \inl{resume} we name the continuation function as \inl{r}.

In Helium handlers can have cases not only for effectful operations but also for two special cases: \inl{return} and \inl{finally}. The first one is executed when computation which is wrapped by the handler finishes with some value and its argument is exactly this value. For the case of \inl{finally}, it receives as argument whole computation inside the handler and is run at the beginning of handler's execution. By default both handlers plainly return received values.

%% \begin{lstlisting}
%% handler
%% | return x => x
%% | finally f => f
%% end
%% \end{lstlisting}

%% ^ usunięte gdyż powodowało problemy dla LaTeX'a, a nic nie wnosiło (prócz nieprawidłowego wyrażenia w Helium)

We can cleverly use them. For instance, instead of just checking if formula is satisfiable, we can count the number of valuations that make it true:

\lstinputlisting[linerange={5-16}]{code_examples/nondet2__count_sats.he}

When computation is finished, instead of returning boolean value telling if formula was satisfied, we return 1 or 0. While handling nondeterministic choice we resume computation with both boolean values and add the results. By using \inl{finally} we can include the message about number of valuations into the handler:

\lstinputlisting[firstline=5, lastline=17]{code_examples/nondet3__count_and_write_sats.he}

Here we abuse \inl{finally} for example's sake, however we will soon see that this construct is really useful.

\subsection{Mutable state}

Let us consider handler's case for \inl{return}:

\begin{lstlisting}
handler
(* ... *)
| return x => fn s => x
end
\end{lstlisting}

Final value of the computation, instead of being a simple value, is a function. As the result, in this handler continuations are not simple values but functions. This way we can parameterise continuations not only with values returned by operations (according to their signatures), but also with our -- handler author's -- own. Let us notice, that since the result of handled computation is now a function and not a simple value -- in order for the handler's user not to notice type mismatch -- we need to execute this function with some parameter. This is a great moment to use \inl{finally} construct.

We define state effect with operations for reading and modifying it:

\lstinputlisting[lastline=3]{code_examples/state1__basics.he}

The effect and its operations are parameterised by types of stored state. Now let us define a standard handler for state effect. We will take advantage of the fact that handlers are values in Helium, and as such can be a resulting value of a function. The function will be parameterised by state's initial value:

\lstinputlisting[firstline=5, lastline=11]{code_examples/state1__basics.he}

We the computation is finished, instead of returning a value we return a function which ignores its argument (that is current state's value) and returns said value. As the result, cases for operations need to return functions. For \inl{put} we don't read state and resume computation with updated value. However, as we know resumed computation still requires one argument, which we provide from operation's parameter. In case of \inl{get} we read current state's value and resume computation by providing said value both as argument for continuation and as next state value. Finally, we have to decide what to do in the case of \inl{finally}. Since we transformed the computation from returning simple value to the one returning a function that expects value of state, we can call it with initial value of state -- provided by handler's user.

If we want to return not only result of computation but also final value of state, we just modify handler's case for \inl{return}:

\lstinputlisting[firstline=5, lastline=11]{code_examples/state2__run_state.he}

Now, after defining the side effect, its operations and handlers, we can perform computations with mutable state:

%% \lstinputlisting[firstline=7, lastline=24]{code_examples/state3__example.he}
\lstinputlisting[linerange={7-11}]{code_examples/state3__example.he}
\pagebreak
\lstinputlisting[linerange={13-22}]{code_examples/state3__example.he}

\subsection{Effect of recursion}

In some ML-like languages (such as OCaml or Helium) in order for function's identifier to be available inside its body, we need to declare the function using keyword \inl{let rec}:

\lstinputlisting[firstline=2, lastline=5]{code_examples/rec1__rec.he}

Surprisingly, thanks to possibility to define custom effects and operations, we can create recursive function that don't appear as such:

\lstinputlisting[firstline=2, lastline=13]{code_examples/rec2__effect.he}

The \inl{handle `a in ...} construct lets us give labels to effect instances created by handlers -- it is useful in ambiguous cases when we use multiple instances of the same effect or for better readability.

Using the effect of recursion, we can also define functions that are mutually recursive:

\lstinputlisting[firstline=5, lastline=23]{code_examples/rec3__mutual.he}

We keep an information which function is currently executed and when it asks to perform recursive call, we execute the other function, and finally give back the return value.

% 
% Może jeszcze współbieżność kooperatywna?
%   (26.08.2020) może jednak nie :)
% Byłaby naturalnym przedłużeniem wzajemnej rekursji
% ale zawierałaby trochę szumu w postaci kolejkowania zadań.
%

\subsection{Multiple effects at once -- failure and nondeterminism}

As the last example in this chapter, we will see how easy it is to compose effects in Helium. Let us define effects of nondeterminism and failure and very simple handlers for them:

\lstinputlisting[linerange={1-13}]{code_examples/fail_and_amb.he}

%% oraz bardzo proste uchwyty dla tych efektów:

%% \lstinputlisting[firstline=7, lastline=15]{code_examples/fail_and_amb.he}

We now define a function that checks if given formula with three free variables is satisfiable:

\lstinputlisting[linerange={17-23}]{code_examples/fail_and_amb.he}

If formula with given valuation is not satisfied, failure effect occurs. Let us bring our attention to the order of handlers -- nondeterminism on the outside, failure on the inside. Thus, when failure occurs, its handler will return false and backtracking to last available choice will be performed by handler of nondeterminism. As the result, value of\inl{is\_sat f} is equal to false if and only if failure occurred for each valuation. Now let us consider a function that check if formula is a tautology:

\lstinputlisting[linerange={25-31}]{code_examples/fail_and_amb.he}

Here, handler of failure is on the outside -- occurrence of failure during nondeterministic valuation means that there is a valuation that makes the formula not satisfied, an thus the formula is not a tautology. We can now write a nice function that will tell us if \inl{formula1} is satisfiable and if it is a tautology:

\lstinputlisting[linerange={33-42}]{code_examples/fail_and_amb.he}

Without hassle we wrote a program, that uses multiple effects at the same time, despite neither of them knowing about the other one. Combining effects is very easy, and order in which we handle them let us make simple and readable definitions of program behaviours in the case when some effectful operation is encountered.

Thanks to Helium, we got a better look at algebraic effects and handlers, we saw examples of handlers and their implementations, and solutions to sample problems. We are now ready to start implementing build systems using effects and handlers -- which we will do in the next chapter.

\undef\inl
