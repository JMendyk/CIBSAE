
\chapter{Efekty algebraiczne i uchwyty w~praktyce}

\section{Języki programowania z efektami algebraicznymi}

Zainteresowanie efektami algebraicznymi oraz uchwytami doprowadziło do powstania w ostatnich latach wielu bibliotek dla języków popularnych w środowisku akademickim i pasjonatów języków funkcyjnych -- Haskella (extensible-effects\footnote{\url{https://hackage.haskell.org/package/extensible-effects}},
fused-effects\footnote{\url{https://hackage.haskell.org/package/fused-effects}},
polysemy\footnote{\url{http://hackage.haskell.org/package/polysemy}}), Scali
(Effekt\footnote{\url{https://github.com/b-studios/scala-effekt}},
atnos-org/eff\footnote{\url{https://github.com/atnos-org/eff}})
i Idris (Effects \footnote{\url{https://www.idris-lang.org/docs/current/effects_doc/}}).

Związana z językiem OCaml jest inicjatywa ocaml-multicore\footnote{\url{https://github.com/ocaml-multicore/ocaml-multicore/wiki}}, której celem jest stworzenie implementacji OCamla ze wsparciem dla współbieżności oraz współdzielonej pamięci, a cel ten jest realizowane przez wykorzystanie konceptu efektów i uchwytów.

Badania nad efektami i uchwytami przyczyniły się także do powstania kilku eksperymentalnych języków programowania w których efekty i uchwyty są obywatelami pierwszej kategorii. Do języków tych należą:
\begin{itemize}
\item Eff\footnote{\url{https://www.eff-lang.org/}} -- powstający z inicjatywy Andreja Bauera and Matija Pretnara język o ML-podobnej składni,
\item Frank\footnote{\url{https://github.com/frank-lang/frank}} \cite{DBLP:journals/corr/LindleyMM16} -- pod przewodnictwem Sama Lindley'a, Conora McBride'a oraz Craiga McLaughlin'a projektowany z tęsknoty do ML'a, a jednocześnie upodobania do Haskell-owej dyscypliny,
\item Koka\footnote{\url{https://github.com/koka-lang/koka}} -- kierowany przez Daana Leijena z Microsoft projekt badawczy; Koka ma składnię inspirowaną JavaScriptem,
\item Helium\footnote{\url{https://bitbucket.org/pl-uwr/helium/src/master/}} \cite{biernacki2019abstracting} -- powstały w Instytucie Informatyki Uniwersytetu Wrocławskiego, z ML-podobnym systemem modułów i lekkimi naleciałościami z Haskella.
\end{itemize}

\section{Helium}

\lstset{style=helium}

Używając właśnie języka Helium, w tym rozdziale zobaczymy jak w praktyce wygląda programowanie z efektami algebraicznymi oraz uchwytami, zaś w następnym spróbujemy zaimplementować wyniki uzyskane w ,,Build systems {\`a} la carte'' \cite{mokhov2018build, mokhov2020build}. Po raz pierwszy Helium pojawia się w \cite{biernacki2019abstracting}, służąc za narzędzie do eksperymentowania i umożliwienia konstrukcji bardziej skomplikowanych przykładów oraz projektów w celu przetestowania efektów i uchwytów w praktyce.

Rozważmy przykład prostego programu napisanego w Helium, w którym definiujemy pomocniczą funkcję \textit{is\_negative} ustalającą czy liczba jest ujemna oraz {\textit question}, która pyta użytkownika o liczbę i informuje czy liczba ta jest ujemna:

\lstinputlisting{code_examples/syntax.he}

Sygnatura funkcji \textit{is\_negative} wyznaczona przez system typów Helium, to jak łatwo się domyśleć \(\mathit{Int \rightarrow Bool}\). Gdy jednak zapytamy środowisko uruchomieniowe o typ funkcji \textit{question} otrzymamy interesującą sygnaturę \(\mathit{Unit \rightarrow[IO] Unit}\). W Helium, informacje o efektach występujących w trakcie obliczania funkcji są umieszczona w sygnaturach funkcji w kwadratowych nawiasach. W przypadku funkcji \textit{question}, jej obliczenie powoduje wystąpienie efektu ubocznego związanego z mechanizmem wejścia/wyjścia. 

\begin{lstlisting}
printStr: String ->[IO] Unit
readInt: Unit ->[IO] Int
\end{lstlisting}

System inferencji typów wiedząc, że operacje we/wy są zadeklarowane z powyższymi sygnaturami wnioskuje, że skoro wystąpienia tychże operacji w kodzie \textit{question} nie są obsługiwane przez uchwyt, to efekt \textit{IO} wyjdzie poza tą funkcję.

Efekty \textit{IO} oraz \textit{RE} (runtime error) są szczególne, gdyż są dla nich zadeklarowane globalne uchwyty w bibliotece standardowej, jeśli efekt nie zostanie obsłużony i dotrze do poziomu środowiska uruchomieniowego, to ono zajmie się jego obsługą. W przypadku efektów wejścia/wyjścia środowisko skorzysta ze standardowego wejścia/wyjścia, zaś w przypadku wystąpienia efektu \textit{RE}, obliczenie zostanie przerwane ze stosownym komunikatem błędu.

\section{Przykłady implementacji uchwytów}

\subsection{Błąd}

Zaimplementujemy kilka efektów ubocznych, zaczynając od efektu błędu, wraz z uchwytami dla nich. zaczniemy od efektu błędu. W Helium, efekt oraz powodujące go operacje definiuje się następująco:

\lstinputlisting{code_examples/error1__signature.he}

Stwórzmy funkcję podobną do \textit{question}, z tym że nie będzie ona lubić wartości ujemnych:

\lstinputlisting[firstline=7, lastline=18]{code_examples/error2__inline_abort.he}

Zdefiniowaliśmy efekt uboczny \textit{Error} wraz z operacją \textit{error}, która go powoduje. Operacja ta jest parametryzowana wartością typu \textit{Unit} oraz jej (możliwy) wynik to także wartość z \textit{Unit}. Definiujemy też funkcję \textit{main} w której wywołujemy \textit{no\_negatives\_question}, jednakże obliczenie wykonujemy w uchwycie w którym definiujemy co ma się wydarzyć, gdy w czasie obliczenia wystąpi efekt błędu spowodowany operacją \textit{error}. W tym przypadku mówimy, że skutkuje ono wypisaniem wiadomości na standardowe wyjście. Nie wznawiamy obliczenia, stąd wystąpienie błędu skutkuje zakończeniem nadzorowanego obliczenia. Jeśli uruchomimy teraz program i podamy ujemną liczbę, zakończy się on komunikatem zdefiniowanym w uchwycie, a tekst ,,Question finished'' nie zostanie wypisany -- zgodnie z oczekiwaniami, obliczenie \textit{no\_negatives\_question} nie zostało kontynuowane po wystąpieniu błędu.

Jeśli pewnego uchwytu zamierzamy używać wiele razy, możemy przypisać mu identyfikator -- uchwyty są wartościami w Helium:

\lstinputlisting[firstline=7, lastline=10]{code_examples/error3__reused_handler.he}

i zmodyfikować funkcję \textit{main} by z niego korzystać:

\lstinputlisting[firstline=12, lastline=13]{code_examples/error3__reused_handler.he}

Na potrzeby przykładu, możemy rozważyć spokojniejszy uchwyt dla wystąpień \textit{error}, który wypisze ostrzeżenie o wystąpieniu błędu ale będzie kontynuował obliczenie:

\lstinputlisting[firstline=7, lastline=10]{code_examples/error4__warn_not_abort.he}

Jeśli skorzystamy z tego uchwytu w programie, po wyświetleniu ostrzeżenia obliczenie \textit{no\_negatives\_question} zostanie wznowione i na ekranie zobaczymy komunikat ,,Question finished''. Specjalna funkcja \textit{resume}, dostępna w uchwycie reprezentuje kontynuację obliczenia, które zostało przerwane wystąpieniem operacji powodującej efekt uboczny.

\subsection{Niedeterminizm}

Powróćmy do problemu, który w rozdziale drugim był inspiracją do rozważania niedeterminizmu -- sprawdzanie czy formuła jest spełnialna oraz czy jest tautologią. Przedstawiliśmy wtedy uchwyty dla obu tych problemów w naszej notacji. Implementacja efektu niedeterminizmu, operacji \textit{amb} oraz uchwytów wraz z wykorzystaniem ich wygląda następująco:

\lstinputlisting[lastline=22]{code_examples/nondet1__simple.he}

Będziemy sprawdzać czy formuła wyrażona za pomocą funkcji \textit{formula1} jest spełnialna. W tym celu w funkcji \textit{main}, wewnątrz uchwytu, niedeterministycznie ustalamy wartości zmiennych \textit{x}, \textit{y}, \textit{z} po czym obliczamy wartość funkcji \textit{formula1}. Wartość obsługiwanego wyrażenia, którą przypisujemy do zmiennej \textit{ret}, jest następnie wykorzystana do wypisania komunikatu. Ponadto -- w celu demonstracji możliwości języka -- w uchwytach zamiast kontynuować obliczenie używając \textit{resume}, przypisujemy kontynuacji nazwę \textit{r}.

W Helium, uchwyty mogą posiadać przypadki nie tylko dla operacji związanych z jakimś efektem ale także dwa specjalne przypadki: \textit{return} oraz \textit{finally}. Pierwszy jest wykonywany, gdy obliczenie pod kontrolą uchwytu kończy się zwracając wynik, przypadek jako argument otrzymuje wynik obliczenia. Zaś \textit{finally} otrzymuje jako argument obliczenie obsługiwane przez uchwyt i jest uruchamiane na początku działania uchwytu. Domyślnie przypadki te są zaimplementowane jako:

\begin{lstlisting}
handler
| return x => x
| finally f => f
end
\end{lstlisting}

Możemy je jednak sprytnie wykorzystać. Przykładowo, zamiast tylko sprawdzać czy formuła jest spełnialna, możemy sprawdzić przy ilu wartościowaniach jest prawdziwa:

\lstinputlisting[firstline=5, lastline=16]{code_examples/nondet2__count_sats.he}

Gdy obliczenie się kończy, zamiast zwracać czy formuła jest spełniona zwracamy 1 albo 0, w zależności czy formuła przy aktualnym wartościowaniu jest spełniona. Gdy obsługujemy niedeterministyczny wybór, kontynuujemy obliczenie dla obu możliwych wartości boolowskich po czym dodajemy wyniki. Wykorzystując \textit{finally} możemy włączyć komunikat o liczbie wartościowań do uchwytu:

\lstinputlisting[firstline=5, lastline=17]{code_examples/nondet3__count_and_write_sats.he}

Tutaj wykorzystanie \textit{finally} jest lekkim nadużyciem, jak jednak za chwilę zobaczymy, konstrukcja ta jest bardzo przydatna.

\subsection{Modyfikowalny stan}

Rozważmy następujący przypadek dla \textit{return} w uchwycie:

\begin{lstlisting}
handler
(* ... *)
| return x => fn s => x
end
\end{lstlisting}

Wartość obliczenia, zamiast być jego wynikiem, jest funkcją. Co za tym idzie, w tym uchwycie kontynuacje nie będą funkcjami zwracającymi wartości lecz funkcje. W ten sposób możemy parametryzować dalsze obliczenia nie tylko wartościami zwracanymi przez operacje (zgodnie z ich sygnaturą) ale także wymyślonymi przez nas -- autorów uchwytu. Zauważmy jednak, że parametr ten nie jest widoczny w obsługiwanym obliczeniu, a jedynie w uchwycie. Co więcej, skoro wyniki obsługiwanego obliczenia jest teraz funkcją, a nie wartością to by użytkownik uchwytu nie zauważył niezgodności typów musimy funkcję tą uruchomić z jakimś parametrem -- tutaj właśnie przychodzi naturalny moment na wykorzystanie konstrukcji \textit{finally}.

Definiujemy efekt stanu z operacją jego odczytu oraz modyfikacji:

\lstinputlisting[lastline=3]{code_examples/state1__basics.he}

Efekt jak i operacje są polimorficzne ze względu na typ wartości stanu. Zdefiniujemy teraz standardowy uchwyt dla efektu stanu. Skorzystamy z faktu, że uchwyty są w Helium wartościami, stąd w szczególności mogą być wynikiem funkcji. Funkcja ta będzie u nas parametryzowana wartością początkową stanu:

\lstinputlisting[firstline=5, lastline=11]{code_examples/state1__basics.he}

Gdy obliczenie się kończy, zamiast wartość, zwracamy funkcję która ignoruje argument, a zwraca właściwy wynik obliczenia. Ten argument będzie bieżącą wartością stanu. W konsekwencji przypadki dla operacji też muszą być funkcjami. Dla \textit{put} nie musimy odczytywać aktualnej wartości stanu, stąd wartość tą ignorujemy. Obliczenie wznawiamy z wartością jednostkową. Jak jednak wiemy, wynikiem będzie nie zwykła wartość lecz funkcja, której u nas dajemy wartość stanu. Stąd podajemy jej nową wartość stanu, którą parametryzowana była operacja \textit{put}. W przypadku \textit{get} postępujemy podobnie -- jednak tym razem odczytamy argument funkcji i przekażemy go do kontynuacji. Niezmiennie kontynuacja zwraca funkcję, której przekażemy aktualną wartość stanu. Pozostaje rozstrzygnąć co zrobić w przypadku \textit{finally}. Skoro jednak przerobiliśmy obliczenie ze zwracającego wartość do takiego, które zwraca funkcję oczekującą wartości stanu, to możemy podać mu wartość początkową -- określoną przez użytkownika uchwytu.

Jeśli chcemy aby obliczenie zwracało nie tylko wartość wynikową ale także końcowy stan, wystarczy że zmodyfikujemy przypadek dla \textit{return}:

\lstinputlisting[firstline=5, lastline=11]{code_examples/state2__run_state.he}

Dzięki zdefiniowanemu efektowi ubocznemu, operacjom oraz uchwytom możemy teraz łatwo wykonywać obliczenia ze stanem:

\lstinputlisting[firstline=7, lastline=24]{code_examples/state3__example.he}

\subsection{Efekt rekursji}

W niektórych językach ML-podobnych (jak na przykład OCaml czy Helium) chcąc by w ciele definicji funkcji był widoczny jej identyfikator, trzeba zadeklarować ją używając słów kluczowych \textit{let rec}:

\lstinputlisting[firstline=2, lastline=5]{code_examples/rec1__rec.he}

Co ciekawe, dzięki własnym efektom i operacjom możemy tworzyć funkcje rekurencyjne, które nie używają jawnie rekursji:

\lstinputlisting[firstline=2, lastline=13]{code_examples/rec2__effect.he}

Konstrukcja \textit{\textbf{handle} `a \textbf{in}} służy doprecyzowaniu który efekt ma być obsłużony przez uchwyt -- jest przydatna w przypadku niejednoznaczności gdy używamy wielu instancji tego samego efektu lub dla ułatwienia rozumienia kodu.

Możemy w ten sposób definiować także funkcje wzajemnie rekurencyjne:

\lstinputlisting[firstline=5, lastline=23]{code_examples/rec3__mutual.he}

Utrzymujemy informację, która funkcja jest aktualnie wykonywana i gdy prosi o wywołanie rekurencyjne uruchamiamy obliczanie drugiej funkcji po czym wynik przekazujemy do kontynuacji.

% 
% Może jeszcze współbieżność kooperatywna?
% Byłaby naturalnym przedłużeniem wzajemnej rekursji
% ale zawierałaby trochę szumu w postaci kolejkowania zadań.
% 

\subsection{Wiele efektów na raz -- porażka i niedeterminizm}

Na koniec rozdziału, zobaczymy jak łatwo w Helium komponuje się efekty. Definiujemy efekty niedeterminizmu oraz porażki:

\lstinputlisting[firstline=1, lastline=5]{code_examples/fail_and_amb.he}

oraz bardzo proste uchwyty dla tych efektów:

\lstinputlisting[firstline=7, lastline=15]{code_examples/fail_and_amb.he}

Definiujemy teraz funkcję sprawdzającą czy otrzymana formuła z trzema zmiennymi wolnymi jest spełnialna:

\lstinputlisting[firstline=19, lastline=25]{code_examples/fail_and_amb.he}

Jeśli formuła przy ustalonym wartościowaniu nie jest spełniona powoduje efekt porażki. Zwróćmy uwagę w jakiej kolejności są umieszczone uchwyty -- niedeterminizmu na zewnątrz, zaś porażki wewnątrz. W ten sposób gdy wystąpi porażka, jej uchwyt zwróci fałsz, w wyniki czego nastąpi powrót do ostatniego punktu niedeterminizmu w którym jest jeszcze wybór. Dzięki temu wartość \textit{is\_sat f} jest równa fałszowi tylko gdy przy każdym wartościowaniu nastąpi porażka. Zobaczmy teraz funkcję sprawdzającą czy otrzymana formuła jest tautologią:

\lstinputlisting[firstline=27, lastline=33]{code_examples/fail_and_amb.he}

Tutaj uchwyt dla porażki znajduje się na zewnątrz -- wystąpienie porażki oznacza, że istnieje wartościowanie przy którym formuła nie jest prawdziwa, a w konsekwencji nie może być tautologią. Możemy teraz napisać zgrabną funkcję, która wypisze nam czy \textit{formula1} jest spełnialna oraz czy jest tautologią:

\lstinputlisting[firstline=35, lastline=46]{code_examples/fail_and_amb.he}

Z łatwością napisaliśmy program, który korzysta z wielu efektów ubocznych jednocześnie, mimo że żaden z nich (ani uchwyty) nie wiedzą o istnieniu drugiego. Łączenie efektów jest bardzo proste, a kolejność w jakiej umieszczamy uchwyty umożliwia nam łatwe i czytelne definiowanie zachowania programu w przypadku wystąpienie któregokolwiek z efektów.

Dzięki językowi Helium, przyjrzeliśmy się z bliska efektom algebraicznym oraz uchwytom, zobaczyliśmy przykłady implementacji uchwytów oraz rozwiązań prostych problemów. Jesteśmy gotowi do podjęcia próby zaimplementowania systemów kompilacji z użyciem efektów i uchwytów -- czego dokonamy w następnym rozdziale.
