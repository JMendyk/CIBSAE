
\chapter{O efektach algebraicznych teoretycznie}

Wprowadzimy notację służącą opisowi prostych obliczeń, która pomoże nam -- bez zanurzania się głęboko w ich rodowód matematyczny -- zrozumieć jak prostym, a jednocześnie fascynującym tworem są efekty algebraiczne i uchwyty. Przedstawiona notacja jest intencjonalnie nieformalna, gdyż ma w dostępny sposób przedstawić abstrakcyjny opis obliczeń z efektami bez prezentowania konkretnego języka programowania.

Następnie przyjrzymy się, jak możemy zapisać popularne przykłady efektów ubocznych używając naszej notacji. Na koniec, czytelnikowi zostaną polecone zasoby do dalszej lektury, które rozszerzają opis z tego rozdziału.

\section{Notacja}

\newcommand{\return}[1]{\mathbf{return}\ #1}
\newcommand{\op}[3]{#1(#2, #3)}
\newcommand{\opi}[3]{\op{op_{#1}}{#2}{#3}}
\newcommand{\handle}[2]{\mathbf{handle}\ #1\ \mathbf{with}\ #2}
\newcommand{\hcase}[3]{#1\ #2\ \Rightarrow\ #3}
\newcommand{\fun}[2]{\lambda #1.\ #2}
\newcommand{\eval}[1]{\llbracket\, #1\, \rrbracket}
\newcommand{\cond}[3]{\mathbf{if}\ #1\ \mathbf{then}\ #2\ \mathbf{else}\ #3}

Będziemy rozważać obliczenia nad wartościami następujących trzech typów:
\begin{itemize}
\item boolowskim \(B\) -- z wartościami \(T\) i \(F\) oraz standardowymi spójnikami logicznymi,
\item liczb całkowitych \(\mathbb{Z}\) -- wraz z ich relacją równości oraz podstawowymi działaniami arytmetycznymi,
\item typem jednostkowym \(U\) -- zamieszkałym przez pojedynczą wartość \(u\),
\item oraz pary tychże typów.
\end{itemize}

% 
% Przemyślenia:
% 1. Pozbyć się "return v"
% 2. Dodać przypadek "return x" do zbioru w uchwycie, zdefiniować jego działanie
%    Można by wtedy rozszerzyć przykłady o zmianę wartości wynikowej
% 

Nasz model składać się będzie z wyrażeń:
\begin{itemize}
\item \(\return{v}\) -- gdzie \(v\) jest wyrażeniem boolowskim lub arytmetycznym,
\item \(\cond{v_1 = v_2}{e_t}{e_f}\) -- wyrażenie warunkowe, gdzie \(v_1 = v_2\) jest pytaniem o równość wartości dwóch wyrażeń arytmetycznych,
\item abstrakcyjnych operacji oznaczanych \(\{op_i\}_{i \in I}\) -- powodujących wystąpienie efektów ubocznych -- których działanie nie jest nam znane, zaś ich sygnatury to \(op_i: A \rightarrow (B \rightarrow C) \rightarrow D\), gdzie \(A\), \(B\), \(C\) oraz \(D\) to pewne typy w naszym modelu. Wyrażenie~\(\opi{i}{n}{\kappa}\) opisuje operację z argumentem \(n\) oraz dalszą częścią obliczenia \(\kappa\) parametryzowaną wynikiem operacji, które \textit{może (nie musi)} zostać wykonane po wykonaniu operacji,
\item uchwytów, czyli wyrażeń postaci \(\handle{e}{\{\ \hcase{op_i}{n\ \kappa}{h_i}\ \}_{i \in I}}\), gdzie \(e\) to inne wyrażenie; uchwyt definiuje działanie (dotychczas abstrakcyjnych) operacji. 
\end{itemize}

Przykładowymi obliczeniami w naszej notacji są więc:
\begin{equation}
\begin{gathered}
  \return{0},\quad\return{2 + 2},\quad \opi{1}{2}{\fun{x}{\return{x + 1}}} \\
  \handle{\opi{1}{2}{\fun{x}{\return{x + 1}}}}{\{\ \hcase{op_1}{n\ \kappa}{\kappa\ (2 \cdot n)} \ \}}
\end{gathered}
\end{equation}

Dla czytelności, pisząc w uchwycie zbiór który nie przebiega wszystkich operacji, przyjmujemy że uchwyt nie definiuje działania operacji; równoważnie, zbiór wzbogacamy o element: \(\hcase{op_i}{n\ \kappa}{op_i(n, \kappa)}\).

Obliczanie wartości wyrażenia przebiega następująco:
\begin{itemize}
\item \(\eval{\return v} = v\) -- wartością \(\mathbf{return}\) jest wartość wyrażenia arytmetycznego,
\item \(\eval{(\fun{x}{e})\ y} = \eval{e \left[\sfrac{x}{\eval{y}}\right]}\) -- aplikacja argumentu do funkcji,
\item
  \(\begin{aligned}[t]
    \eval{\cond{v_1 = v_2}{e_t}{e_f}} = \left\{\begin{matrix}
    \eval{e_t} & \text{gdy }\eval{v_1} = \eval{v_2} \\ 
    \eval{e_f} & \text{wpp}
    \end{matrix}\right.
  \end{aligned}\)
\item \(\eval{\handle{\return v}{H}} = \eval{\return v}\) -- uchwyt nie wpływa na wartość obliczenia, które nie zawiera efektów ubocznych,
\item \(\eval{\handle{\opi{i}{a}{\fun{x}{e}}}{H}} = \eval{h_i \left[\sfrac{n}{\eval{a}},\, \sfrac{\kappa}{\fun{x}{\handle{e}{H}}}\right]} \), \\gdzie \(H~=~\{\ \hcase{op_i}{n\ \kappa}{h_i} \ \}\).
  
\end{itemize}

Zobaczmy jak zatem wygląda obliczenie ostatniego z powyższych przykładów:
\begin{equation}\begin{split}
  \eval{\handle{\opi{1}{2}{\fun{x}{\return{x + 1}}}}{\{\ \hcase{op_1}{n\ \kappa}{\kappa\ (2 \cdot n)} \ \}}} &= \\
  \eval{\handle{(\fun{x}{\return{x+1}}) (2 \cdot 2)}{\{\ \hcase{op_1}{n\ \kappa}{\kappa\ (2 \cdot n)} \ \}}} &= \\
  \eval{\handle{\return{4+1}}{\{\ \hcase{op_1}{n\ \kappa}{\kappa\ (2 \cdot n)} \ \}}} &= \\
  \eval{\return{4 + 1}} &= 5
\end{split}\end{equation}


\section{Równania, efekt porażki i modyfikowalny stan}

Do tego momentu, nie przyjmowaliśmy żadnych założeń na temat operacji powodujących efekty uboczne. Uchwyty mogły w związku z tym działać w sposób całkowicie dowolny. Ograniczymy się w tej dowolności i nałożymy warunki na uchwyty wybranych operacji. Przykładowo, ustalmy że dla operacji \(op_r\), uchwyty muszą być takie aby następujący warunek był spełniony:
\begin{align}
  \forall n\ \forall e.\ \eval{\handle{op_r(n, \fun{x}{e})}{H}} = n
\end{align}

Zauważmy, że istnieje tylko jeden naturalny uchwyt spełniający tej warunek, jest nim \(H = \{\ \hcase{op_r}{n\ \kappa}{n} \ \}\). Co więcej, jego działanie łudząco przypomina konstrukcję wyjątków w popularnych językach programowania:

\begin{lstlisting}
  try {
    raise 5;
    // ...
  } catch (int n) {
    return n;
  }
\end{lstlisting}

Podobieństwo to jest w pełni zamierzone. Okazuje się że nasz język z jedną operacją oraz równaniem ma już moc wystarczającą do opisu konstrukcji, która w większości popularnych języków nie może zaistnieć z woli programisty, a zamiast tego musi być dostarczona przez twórcę języka.

Rozważmy kolejny przykład. Dla poprawienia czytelności, zrezygnujemy z oznaczeń \(op_i\) na operacje powodujące efekty, zamiast tego nadamy im znaczące nazwy: \(get\) oraz \(put\). Operacje te mają sygnatury \(get: U \rightarrow (\mathbb{Z} \rightarrow \mathbb{Z}) \rightarrow \mathbb{Z}\), \(put: \mathbb{Z} \rightarrow (U \rightarrow \mathbb{Z}) \rightarrow \mathbb{Z}\). Spróbujemy wyrazić działanie tych dwóch operacji by otrzymać modyfikowalną komórkę pamięci. Ustalamy równania:

\begin{itemize}
\item \(\forall e.\ \eval{get(u, \fun{\_}{get(u, \fun{x}{e})})} = \eval{get(u, \fun{x}{e})}\)

  kolejne odczyty z komórki bez jej modyfikowania dają takie same wyniki,
\item \(\forall e.\ \eval{get(u, \fun{n}{put(n, \fun{u}{e})})} = \eval{e}\)

  umieszczenie w komórce wartości, która już tam się znajduje, nie wpływa na wynik obliczenia,
\item \(\forall n.\ \forall f.\ \eval{put(n, \fun{u}{get(u, \fun{x}{f\ x})})} = \eval{f\ n}\)

  obliczenie które odczytuje wartość z komórki daje taki sam wyniki, jak gdyby miało wartość komórki podaną wprost jako argument,
\item \(\forall n_1.\ \forall n_2.\ \forall e.\ \eval{put(n_1, \fun{u}{put(n_2, \fun{u}{e})})} = \eval{put(n_2, \fun{u}{e})}\)

  komórka zachowuje się, jak gdyby pamiętała jedynie najnowszą włożoną do niej wartość.
\end{itemize}

Zauważmy, że choć nakładamy warunki na zewnętrzne skutki działania operacji \(get\) oraz \(put\), to w żaden sposób nie ograniczyliśmy swobody autora w implementacji uchwytów dla tych operacji.

% W rozdziale 4 przyglądniemy się kilku przykładom uchwytów realizujących te operacje.
% ^^^ Chyba jednak nie. Chociaż można dodać przykład, który agreguje historyczne wartości komórki i zwraca je w parze z wynikiem obliczenia.

\section{Poszukiwanie sukcesu}

Kolejnym rodzajem efektu ubocznego, który rozważymy w tym rozdziale jest niedeterminizm. Chcielibyśmy wyrażać obliczenia, w których pewne parametry mogą przyjmować wiele wartości, a ich dobór ma zostać dokonany tak by spełnić pewien określony warunek. Przykładowo, mamy trzy zmienne \(x,\, y\) oraz \(z\) i chcemy napisać program sprawdzający czy formuła \(\phi(x, y, z)\) jest spełnialna. W tym celu zdefiniujemy operację \(amb: U \rightarrow (B \rightarrow B) \rightarrow B\) związaną z efektem niedeterminizmu. Napiszmy obliczenie rozwiązujące nasz problem:
\begin{equation}\begin{split}
  \handle{
    &\op{amb}{u}{\fun{x}{
        \op{amb}{u}{\fun{y}{
            \op{amb}{u}{\fun{z}{
                \phi(x, y, z)
            }}
        }}
    }}\\
  }{ \{ \ &\hcase{amb}{u\ \kappa}{ \kappa\ (T) \ \mathbf{or} \ \kappa\ (F) } \ \} }
\end{split}\end{equation}

Gdy definiowaliśmy efekt wyjątku, obliczenie nie było kontynuowane. W przypadku niedeterminizmu kontynuujemy obliczenie dwukrotnie -- podstawiając za niedeterministycznie określoną zmienną wartości raz prawdy, raz fałszu -- w czytelny sposób sprawdzamy wszystkie możliwe wartościowania, a w konsekwencji określamy czy formuła jest spełnialna.

Możemy zauważyć, że gdybyśmy chcieli zamiast sprawdzania spełnialności, weryfikować czy formuła jest tautologią, wystarczy zmienić tylko jedno słowo -- zastąpić spójnik \(\mathbf{or}\) spójnikiem \(\mathbf{and}\) otrzymując nowy uchwyt:
\begin{equation}\begin{split}
  \handle{
    &\op{amb}{u}{\fun{x}{
        \op{amb}{u}{\fun{y}{
            \op{amb}{u}{\fun{z}{
                \phi(x, y, z)
            }}
        }}
    }}\\
  }{ \{ \ &\hcase{amb}{u\ \kappa}{ \kappa\ (T) \ \mathbf{and} \ \kappa\ (F) } \ \} }
\end{split}\end{equation}

Przedstawiona konstrukcja efektów, operacji i uchwytów tworzy dualny mechanizm w którym operacje są producentami efektów, a uchwyty ich konsumentami. Zabierając źródłom efektów ubocznych ich konkretne znaczenia semantyczne, lub nakładając na nie jedynie proste warunki wyrażone równaniami, otrzymaliśmy niezwykle silne narzędzie umożliwiające proste, deklaratywne oraz -- co najważniejsze, w kontraście do popularnych języków programowania -- samodzielne konstruowanie zaawansowanych efektów ubocznych.

\section{Dalsza lektura}

Rozdział ten miał na celu w lekki sposób wprowadzić idee, definicje i konstrukcje związane z efektami algebraicznymi i uchwytami, które będą fundamentem do zrozumienia ich wykorzystania w praktycznych przykładach oraz implementacji systemów kompilacji w dalszych rzodziałach. Czytelnicy zainteresowani głębszym poznaniem historii oraz rodowodu efektów algebraicznych i uchwytów mogą zapoznać się z następującymi materiałami:

\begin{itemize}
\item ,,An Introduction to Algebraic Effects and Handlers'' autorstwa Matija Pretnara \cite{pretnar2015introduction},
\item notatki oraz seria wykładów Andreja Bauera pt. ,,What is algebraic about algebraic effects and handlers?'' \cite{bauer2018algebraic} dostępne w formie tekstowej oraz nagrań wideo w serwisie YouTube,
\item prace Plotkina i Powera \cite{plotkin2001semantics, plotkin2002computational} oraz Plotkina i Pretnara \cite{plotkin2013handling} -- jeśli czytelnik chce poznać jedne z pierwszych wyników prowadzących do efektów algebraicznych oraz wykorzystania uchwytów,
\item społeczność skupiona wokół tematu efektów algebraicznych agreguje zasoby z nimi związane w repozytorium \cite{effectsbibliography} w serwisie GitHub.

\end{itemize}
