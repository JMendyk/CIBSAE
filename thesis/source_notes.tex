
\newcommand{\file}[1]{\texttt{#1}}

\chapter{Omówienie załączonego kodu źródłowego}

\section{Podział implementacji na pliki}

\begin{itemize}
\item \file{Common.he} oraz \file{Signature.he} -- definicje podstawowych typów oraz sygnatur,
\item \file{Store.he} -- efekt zasobu oraz uchwyty dla niego,
\item \file{Schedulers.he} -- implementacje planistów,
\item \file{Rebuilders.he} -- implementacje rekompilatorów,
\item \file{Traces.he} -- funkcje do interakcji ze śladami,
\item \file{Track.he} -- implementacje funkcji \helinl{run} oraz \helinl{track},
\item \file{Systems.he} -- przykłady definicji omawianych systemów,
\item \file{Spreadsheet.he} -- implementacja obiektów związanych z arkuszami kalkulacyjnymi,
\item \file{scratch.he} -- demonstracja wykorzystania systemów w celu kompilacji przykładowych zadań,
\item \file{Utils.he} -- implementacje funkcji pomocniczych,
\item \file{PatchedWriter.he} -- rozszerzenie modułu \helinl{Writer} z biblioteki standardowej,
\item \file{State.he} -- implementacja efektu modyfikowalnego stanu wraz z uchwytami,
\item \file{Logger.he} -- moduł umożliwiający śledzenie działania programu,
\item \file{SchedulersWithLogging.he} -- implementacje planistów wzbogacone o wyświetlanie informacji diagnostycznych
\end{itemize}

\section{Implementacje śladów}

Dokładny opis śladów w rozdziale 5 został pominięty, gdyż nie wydają się one być ciekawą częścią implementacji systemów kompilacji. Jednak dla porządku ich implementacje są przedstawione poniżej.

\subsection{Typ śladów}

Dla śladów definiujemy typ analogiczny to tego z oryginalnej implementacji w Haskellu.

\begin{minipage}{\textwidth}
  \lstinputlisting[style=Haleff-long, linerange={13-14}]{../Traces.he}
\end{minipage}

Warto zwrócić uwagę, że kompilacja zadań w czasie działania funkcji weryfikujących lub konstruujących ślady odbywa się w kontekście specjalnego uchwytu, który zwraca nie wartości lecz ich skróty.

\subsection{Ślady weryfikujące}

\begin{minipage}{\textwidth}
  \lstinputlisting[style=Haleff-long, linerange={18-28}]{../Traces.he}
\end{minipage}

\subsection{Ślady konstruktywne}

\begin{minipage}{\textwidth}
  \lstinputlisting[style=Haleff-long, linerange={37-49}]{../Traces.he}
\end{minipage}

W funkcjach \helinl{constructCT} oraz \helinl{deepDependencies} wykorzystywany jest efekt \helinl{Writer}. W \BSaLC{} wykorzystywane były typ \haskinl{Maybe} i funkcja \haskinl{catMaybes} oraz list comprehensions. Autor zachęcony brakiem infrastruktury wokół \helinl{Some} (Heliumowego odpowiednika \haskinl{Maybe}) postanowił wypróbować moduł \helinl{Writer} i zobaczyć jaki da to efekt.

\subsection{Głębokie ślady konstruktywne}

\begin{minipage}{\textwidth}
  \lstinputlisting[style=Haleff-long, linerange={53-73}]{../Traces.he}
\end{minipage}
