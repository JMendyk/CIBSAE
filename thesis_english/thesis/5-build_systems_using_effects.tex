
\chapter{Build systems using algebraic effects and handlers}

\chaptermark{Systems using effects and handlers}
\label{chapter-bsue}

In this chapter we will repeat implementation of build systems as presented in \BSaLC\cite{mokhov2018build}, however we will with Helium using algebraic effects and handlers. For starters we will come up with counterparts for Haskell abstract types related to build systems, and later we will implement all rebuilders and all but one schedulers. Finally, we will see what characterises the absent scheduler and learn about other implementation attempts inspired by work of Mokhov et al.

\section{Concept and types}

Let us go back to components of Haskell implementation and figure out its counterparts in Helium.

\subsection{Store}

\begin{lstlisting}[style=haskell-style]
data Store i k v
initialise :: i -> (k -> v) -> Store i k v
getInfo :: Store i k v -> i
putInfo :: i -> Store i k v -> Store i k v
getValue :: k -> Store i k v -> v
putValue :: Eq k => k -> v -> Store i k v -> Store i k v
\end{lstlisting}

Authors of \BSaLC\cite{mokhov2018build} represented the store as a type with read and write functions for persistent information and task results. However, each time it was used was inside some mutable state. Thus, we can simplify our implementation by merging the store with mutable state by making \haskinl{Store} an effect, and turning its functions into effectful operations.

\lstinputlisting[language=Haleff, firstline=5, lastline=9]{../src/Store.he}

Similarly as in original implementation of \haskinl{Store}, \helinl{StoreEff} is parameterised by types of persistent information as well as keys and values of tasks. Equations for its operations are analogous to the ones of mutable state but with a fixed key for operations related to result values. Additionally, we define a handler called \helinl{funStoreHandler}, in which the key--value dictionary is represented as a function -- like in examples from \BSaLC{}.

\lstinputlisting[language=Haleff, firstline=38, lastline=52]{../src/Store.he}

The implementation is similar to mutable state from chapter~4. For sanity, initial persistent information and return value dictionary is wrapped in \helinl{FunStoreType I K V} type.

Since Helium, like other ML-like languages, does not have typeclasses known in Haskell, we define couple of signatures corresponding to typeclasses used in original implementation. For the case of \helinl{funStoreHandler} module with signature \helinl{Comparable K} is used for comparing keys that identify tasks. One can notice, that an alternative approach could be to represent typeclasses as effects and related functions as effectful operations.

\begin{minipage}[t]{.45\textwidth}

  \lstinputlisting[language=Haleff, firstline=4, lastline=16]{../src/Signatures.he}

\end{minipage}\hfill
\begin{minipage}[t]{.45\textwidth}

  \lstinputlisting[language=Haleff, firstline=18, lastline=27]{../src/Signatures.he}

\end{minipage}

\subsection{Mutable state}

We saw an implementation of mutable state effect in chapter 4 and we use it here. Names for handlers of state, depending on the return value, are assigned as in corresponding functions in Haskell -- \haskinl{runState}, \haskinl{evalState}, \haskinl{execState}. We also define simple helper functions \helinl{gets} and \helinl{modify}, which use provided transformation for reading and modifying state, accordingly. Moreover, we define slightly more complicated function \helinl{embedState}. 

\lstinputlisting[language=Haleff, linerange={6-7,33-37}, float=h, title={Definitions of \helinl{gets}, \helinl{modify} and \helinl{embedState}}]{../src/State.he}

The \helinl{embedState} function creates a handler for mutable state effect, in which operations -- instead of being done by handler itself -- are passed to provided \helinl{getter} and \helinl{setter} functions, that cause some unknown side effect while being executed. We will use such embedding of mutable state in some other effect for implementing schedulers, which will pass persistent information to rebuilders as embedded mutable state.

\begin{lstlisting}[language=Haleff, float=h, title={Usage example of \helinl{embedState}}]
handle `store in
    (* ... *)
    handle `state in
        (* ... *)
    with embedState (getInfo `store) (putInfo `store)
    (* ... *)
with (* ... *)
\end{lstlisting}

\subsection{Task and build effect}

In \BSaLC{} task was a function that took operation responsible for building requested tasks, and the return result was wrapped in some type \haskinl{f} that belonged to typeclass \haskinl{c}.

\begin{lstlisting}[style=haskell-style]
newtype Task c k v = Task (forall f. c f => (k -> f v) -> f v)
type Tasks c k v = k -> Maybe (Task c k v)
\end{lstlisting}

We can notice however, that building a task is an obvious case of a side effect of build system execution, thus in our implementation instead of providing a function usually called \haskinl{fetch}, we will define a \helinl{BuildEff} effect which will occur during task building. The effect will have one associated operation called \helinl{fetch}.

\lstinputlisting[language=Haleff, linerange={11-15}]{../src/Common.he}

The task will be a function taking an instance of \helinl{BuildEff} and will be polymorphic in types of keys, values and possible side effects different from build effect (this will be useful while implementing rebuilders). Let us notice, that compared to original implementation, definition of task does not include information analogous to typeclass \haskinl{c}, whose member \haskinl{f} wrapped the result value -- we will go back to the subject of this difference later in the chapter.

\subsection{Build, scheduler, rebuilder}

What is left is to define three types mentioned in the above heading.

\begin{lstlisting}[style=haskell-style]
type Build c i k v = Tasks c k v -> k -> Store i k v -> Store i k v
type Scheduler c i ir k v = Rebuilder c ir k v -> Build c i k v
type Rebuilder c ir k v = k -> v -> Task c k v -> Task (MonadState ir) k v
\end{lstlisting}

Building, like in Haskell, takes a set of tasks and the key to build. However, in our implementation instead of also taking a store, our build has a related side effect.

Our planners will also have signatures similar to its Haskell equivalents, but of course enriched with store's side effect, and with module that defines previously mentioned basic operations on keys and values.

\lstinputlisting[language=Haleff, linerange={17-19}]{../src/Common.he}

The \helinl{Rebuilder} type looks almost like the one from original implementation. However, instead of returning a task with modified constraint, the task has new possible side effect that can occur while the task is built.

\section{An example: system busy}

since we finished defining how abstraction of build systems from Haskell relates to our implementation in Helium, we can write our version of \haskinl{busy} build system.

\lstinputlisting[language=Haleff, linerange={16-28}, float=h, title={The \haskinl{busy} build system}]{../src/Schedulers.he}

The core of the implementation, as in Haskell, is definition of handler (there: function) for \helinl{fetch}. Its body is almost a plain translation of original implementation, with a slight difference that instead of resuming the computation implicitly -- by returning a value -- it is continued explicitly be calling \helinl{resume} inside handler's body.

\section{Implementation of traces}

As in \BSaLC{}, implementation of functions that operate on traces are not particularly interesting -- in our cases they almost fully correspond to original ones with exception of using \helinl{Writer} effect instead of Haskell's infrastructure built around \haskinl{Maybe} type and list comprehensions. Implementation with some remarks is available in appendix A and in the source code.

\section{Running and tracking tasks}

While implementing schedulers and rebuilder we will want to execute tasks and track its dependencies. To do so, just like authors of \BSaLC{} did, we will define a simple function \helinl{run} and slightly more interesting \helinl{track}.

\lstinputlisting[language=Haleff, linerange={12-13,20-29}]{../src/Track.he}

The \helinl{track} function takes a label \helinl{`b}, that corresponds to the handler of an instance of build effect, and a task which should be supervised by said handler. The \helinl{track} function is responsible for tracking dependencies of a given task. To do so, an additional handler \helinl{hTrack} is created which will handle the given task. In the case of task requiring a dependency, \helinl{hTrack} will intercept the call to \helinl{fetch} and delegate it to the handler label with \helinl{`b}, to finally note that a call to \helinl{fetch} took place.

Implementation of \helinl{track} is an interesting example of constructing an effect proxy between proper computation and some specialised handler.

\pagebreak

\section{Implementing build systems}

\subsection{Excel}

\lstinputlisting[style=Haleff-long, linerange={13-14}]{../src/Systems.he}

Right at the beginning we see that the promise given by authors of \BSaLC{} was fulfilled -- build systems are created by applying rebuilders to schedulers.

The \helinl{dirtyBitRebuilder} modifies the task, so when executed it will check if key is marked as dirty. If that is the case, the task will be built. Otherwise, the provided value can be returned, since that would be result of building of the initial task.

\lstinputlisting[style=Haleff-long, linerange={36-39}]{../src/Rebuilders.he}

In the restarting handler we maintain a chain, which should approximate the order of building tasks which minimises the number of restarts. The execution begins with reuse of chain from last build and its modified version will be stored in an instance of mutable state label \helinl{`chain}. Moreover, in \helinl{`done} state we accumulate tasks which has already been built in this build run, so we do not have to build them again. Then we define a handler for mutable state corresponding to persistent information using the \helinl{embedState} function introduced earlier in the chapter.

\lstinputlisting[style=Haleff-long, linerange={56-71}]{../src/Schedulers.he}
\vspace{-1.25em}
\begin{lstlisting}[style=Haleff-long]
    let rec restartingHandler = (* ... *)
        and loop () = (* ... *)
    in
    let resultChain = loop () in
        modifyInfo (mapSnd (fn _ => resultChain))
\end{lstlisting}

The core part of this scheduler's implementation consists of build effect handler \helinl{restartingHandler} and function \helinl{loop}. This function executes tasks in the order determined by the chain from previous build, modifies the tasks using rebuilder and finally executes them. Concurrently, new chain is constructed and will be returned by \helinl{loop}.

\lstinputlisting[style=Haleff-long, linerange={72-99}]{../src/Schedulers.he}

During task's build process calls to \helinl{fetch} are intercepted by the handler and checked if requested task has already been built. If that is not the case, the chain is modified so that requested task is in front of the one being currently built. In the handler we use \helinl{return} case for noting that the task was finished and for calling \helinl{loop}.

\subsection{Shake}

\lstinputlisting[style=Haleff-long, linerange={16-17}]{../src/Systems.he}

In the Shake build system, rebuilder uses verifying traces. With \helinl{verifyVT} the rebuilder checks if the task is ,,fresh''. If so, it does not need to be built again. Otherwise, the task is built while being supervised by \helinl{track} which accumulates task's dependencies and creates new traces that will be recorded later by \helinl{recordVT}.

\pagebreak

\lstinputlisting[style=Haleff-long, linerange={43-49}]{../src/Rebuilders.he}

The implementation of suspending scheduler is significantly shorter. We have only to mutable values: the first one (\helinl{`done}) for remembering already built tasks and the other one for embedding persistent information for the needs of the rebuilder -- just like in restarting scheduler.

\lstinputlisting[style=Haleff-long, linerange={30-54}]{../src/Schedulers.he}

The \helinl{suspendingHandler} is incredibly simple -- it just calls \helinl{build} function and resumes the computation with result read from the store. In \helinl{build}, task is checked for being non-trivial (not an input) and if it has not already been built. In such case, new task is constructed with the help from rebuilder and then executed. Otherwise, the task is up to date and there is not need to build it.

\subsection{CloudShake}

\lstinputlisting[style=Haleff-long, linerange={19-20}]{../src/Systems.he}

\lstinputlisting[style=Haleff-long, linerange={59-69}]{../src/Rebuilders.he}

In thee case of constructive traces, the rebuilder checks if provided value is in the set of known values. Otherwise any known value can be returned or -- if no values are known -- the task is built. As with the rebuilder using verifying traces, dependencies are tracked while task is being built.

%% \lstinputlisting[style=Haleff-long, linerange={30-54}]{../src/Schedulers.he}

\subsection{Nix}

\lstinputlisting[style=Haleff-long, linerange={22-23}]{../src/Systems.he}

The rebuilder that uses deep constructive traces is similar to its predecessors. However, as the name suggests, it checks on which input tasks the current one depends.

\lstinputlisting[style=Haleff-long, linerange={73-84}]{../src/Rebuilders.he}

%% \lstinputlisting[style=Haleff-long, linerange={30-54}]{../src/Schedulers.he}

\section{Absent topological scheduler}

As we have seen in chapter~3, restarting and suspending schedulers can deal with both dynamic and static dependencies. That is not the case for the topological scheduler, which only supports tasks with static dependencies, which are modelled in \BSaLC{} as members of \haskinl{Applicative} typeclass.

It seems, that we have no away of stopping inspection of values returned from calls to \helinl{fetch}. We could try to wrap them in a type not known to the task's author, which seems to be a step towards the original implementation. That way, we are moving away from algebraic effects and handlers which are the topic of with paper, thus we will not study the subject of modelling static dependencies.

\section{Existing implementations in other languages}

The results presented above are the first -- to author's knowledge -- attempt to implement build systems inspired by \BSaLC{} using algebraic effects and handlers. Having said that, in \BSaLCTP{} Mokhov et al. mention two attempts of implementing build systems in popular programming languages: Rust\cite{translation_rust} and Kotlin\cite{translation_kotlin}. However, in both cases limitations of used languages lead to loss of precision and cleanliness compared to Haskell implementation.

While lack of topological scheduler does space us away from original implementation, the rest of schedulers and rebuilder that we implemented -- ignoring syntactic differences between languages -- are of similar quality and readability as their originals.
