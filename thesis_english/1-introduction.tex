\chapter{Introduction}

\section{Troubles with computational effects}

Computer programs -- thanks to their ability to interact with external resources such as storage, computer networks or human users -- can do way more than just perform predefined computations. However, this causes their runtime behaviour to depend on the external world in which these resources live and makes the programs' not just sequences of pure computations but also accompanying side effects.

Moreover, computational effects make it substantially harder to understand and reason about programs' behaviour and their correctness which limits their modularity and leads to more frequent introduction of mistakes and bugs by the authors. To avoid such issues, it is often attempted to split the programs into pure and impure parts while minimising the size of the latter. Despite these efforts, it is not trivial to tell if certain module of a program performs only pure computations and we often have to resort to trusting the author that it is true indeed.

\section{Dealing with computational effects}

One of the methods of solving that problem is inserting information about side effects into the type system. We can then use mechanisms of type inference and verification for automatic identification of functions that are not pure -- making it easy for a programmer to tell, from the function's signature, which effects might appear during function's runtime. Well known example of such solution are monads in Haskell programming language. Unfortunately, concurrent use of two independent resources represented by different monads is non-trivial and requires additional structures, such as monad transformers, which bring in additional challenges -- the program of modularity has only been shifted into other space.

On the other hand, there is a new attempt to deal with side effects with the help of type systems called algebraic effects and handlers. From the surface, they seem similar to exceptions known in many programming languages or system calls in operating systems. However, due to the split between definitions of effectful operations and their semantics, and an interesting use of continuations, they give the programmer ease of thinking and reasoning about the programs using them. And in contrast to monads, multiple algebraic effects can be freely used at once.

\section{Build systems}

To find a fine example of computer programs, whose main task is interaction with external resources, you can look no further than at build systems. There the user provides a set of mutually-depended tasks with information how to execute each of them by using results of other tasks, and the systems is responsible for their correct ordering and execution. Furthermore, we expect the build system to track changes in inputs and -- when asked to update the results -- rerun only the tasks whose output values will change. An examples of such systems are Make and -- which might seem surprising -- office applications for editing spreadsheets (such as popular Excel).

In recent publications titled \BSaLC{} \cite{mokhov2018build, mokhov2020build}, authors introduce a method of categorising build systems by taking into account how they determine order of task execution and in which way tasks are identified as requiring a rebuild. Presented categorisation leads the authors to introduction of a framework for creating build systems with expected properties which happens to be easily implementable in Haskell. What is more, it turns out that Applicative and Monad type classes correspond to the possible level of complexity of dependencies between tasks.

\section{About this paper}

This work aims to introduce the reader -- who has experience with Haskell and basics of functional programming languages -- to the innovative solution which are algebraic effects and handlers, and to demonstrate -- by following Mokhov et al -- an implementation of build systems using algebraic effects and handlers in a Helium programming language. As the result, it is possible to compare those two implementations, and to see how programming with algebraic effects and handlers looks like.

In the second chapter, we introduce simple and informal model of calculations that uses algebraic effects and handlers. Some examples of representing standard computational effects in our model are described.

Chapter three serves as an introduction to \BSaLC{}, discussion of observations made be the authors and description of abstraction of build systems and their consequences. Contents of the source publication is presented on a level sufficient to understand implementation of build systems with algebraic effects and handlers in chapter five. At the same time, the reader is encouraged to get acquainted with full contents of the work by Mokhov et al because it is an interesting and easy to read publication.

The fourth chapter begins with enumeration of existing programming languages and libraries that enable programming with algebraic effects and handlers. Next, the reader is introduced to the Helium programming language and presented with sample problems and their solutions that use effects and handlers. Moreover, use of multiple effects at the same time is demonstrated -- which is significantly easier than with monads in Haskell.

Finally, the crowning fifth chapter contains implementations of schedulers, rebuilders and build systems inspired by the results of \BSaLC{}. However, this time in a programming language with algebraic effects and handlers. Presented are the differences between abstract types from which the implementations are derived, and how use of effects and handlers influence the final form. Showcased result is missing one of the schedulers, thus and explanation is provided why it is so.
