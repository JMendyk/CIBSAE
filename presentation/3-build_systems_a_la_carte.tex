
\section{O systemach kompilacji}

\subsection{Przykłady systemów}

\begin{frame}[fragile]
  \frametitle{Make}
  \begin{columns}[t]
    \column{.5\textwidth}
    \begin{itemize}
    \item bardzo popularny
    \item szeroko dostępny
    \item względnie starym
    \item pliki \textit{makefile'e}
    \end{itemize}

    \column{.5\textwidth}
    \begin{lstlisting}[language=make, basicstyle=\scriptsize\ttfamily]
  util.o: util.h util.c
      gcc -c util.c

  main.o: util.h main.c
      gcc -c main.c

  main.exe: util.o main.o
      gcc util.o main.o -o main.exe
    \end{lstlisting}
  \end{columns}
  \begin{itemize}
  \item statyczne zależności: kompilacja przebiega tak samo, niezależnie od wyników podzadań
  \item naturalna kolejność kompilacji: porządek topologiczny -- każde zadanie skompilowane co najwyżej raz
  \end{itemize}
\end{frame}

\begin{frame}
  \begin{definition}[Minimalność]
    System kompilacji jest minimalny, gdy w trakcie budowania każde zadanie jest wykonane co najwyżej raz i tylko gdy w przechodnim domknięciu zadań, od których zależy, istnieje takie zadanie wejściowe, które zmieniło swoją wartość od czasu ostatniego budowania.
  \end{definition}
  \pause
  \begin{example}
    Make jest minimalny
  \end{example}
  \pause
  \begin{definition}[Wejście/zadanie wejściowe]
     Jeśli zadanie nie ma zdefiniowanego sposobu zbudowania, na przykład \textit{util.h} mówimy, że jest wejściem lub zadaniem wejściowym w tej konfiguracji.
  \end{definition}
\end{frame}

\begin{frame}
  \frametitle{Excel}

  Excel to system kompilacji:
  \begin{itemize}
  \item komórki -- zadania wejściowe
  \item formuły -- definicja sposobu budowania
  \end{itemize}

  \vspace{1em}
  \begin{tabular}{ l c r }
  A1: 10 & B1: INDIRECT(``A"\,\&\,C1) & C1: 1 \\
  A2: 20 & &
  \end{tabular}

  \vspace{1em}
  Dynamiczne zależności -- bez sprawdzenia C1 nie da się określić od jakich zadań zależy B1.
\end{frame}

\subsection{Abstrakcyjnie o systemach kompilacji}

\begin{frame}
  \frametitle{Nomenklatura}

  \begin{itemize}
  \item Zasób (Store) -- modyfikowany przez system (Excel -- komórki, Make -- system plików)
  %% \item Opis zadań, instrukcje konstrukcji wyników -- dostarczone przez użytkownika
  \item Trwała informacja -- pamięć systemu między uruchomieniami
  \end{itemize}

  \vspace{1.25em}
  \textbf{System kompilacji}
  \vspace{.75em}
  
  \textbf{Cel}: zmodyfikowanie stanu zasobu w takich sposób, by wartość związana ze wskazanym przez użytkownika kluczem stała się aktualna.
  
  \textbf{Otrzymuje}: definicje zadań, zasób na którym działa oraz klucz, który ma zostać zaktualizowany. Po zakończeniu działania, wartość w Store związana ze wskazanym kluczem ma być aktualna.
\end{frame}
  
\subsection{Reprezentacja}

\lstset{style=haskell-style}

\begin{frame}[fragile]
  \frametitle{Zadania i zasób}

  \begin{lstlisting}
    newtype Task c k v = Task (forall f. c f => (k -> f v) -> f v)
    type Tasks c k v = k -> Maybe (Task c k v)
  \end{lstlisting}

  \begin{lstlisting}
    data Store i k v
    initialise :: i -> (k -> v) -> Store i k v
    getInfo :: Store i k v -> i
    putInfo :: i -> Store i k v -> Store i k v
    getValue :: k -> Store i k v -> v
    putValue :: Eq k => k -> v -> Store i k v -> Store i k v
  \end{lstlisting}

  \begin{example}
  \begin{lstlisting}
    sprsh1 :: Tasks Applicative String Integer
    sprsh1 "B1" = Just $ Task $ \fetch -> ((+)  <$> fetch "A1" <*> fetch "A2")
    sprsh1 "B2" = Just $ Task $ \fetch -> ((*2) <$> fetch "B1")
    sprsh1 _    = Nothing
  \end{lstlisting}
  \vspace{-1em}
  \end{example}
  
\end{frame}

\begin{frame}[fragile]
  \frametitle{System kompilacji}
  \begin{lstlisting}
    type Build c i k v = Tasks c k v -> k -> Store i k v -> Store i k v
  \end{lstlisting}

  \begin{example}
  \begin{lstlisting}
    busy :: Eq k => Build Applicative () k v
    busy tasks key store = execState (fetch key) store
      where
        fetch :: k -> State (Store () k v) v
        fetch k = case tasks k of
            Nothing   -> gets (getValue k)
            Just task -> do v <- run task fetch
                            modify (putValue k v)
                            return v
  \end{lstlisting}
  \vspace{-1em}
  \end{example}
\end{frame}

\begin{frame}[fragile]
  \frametitle{Rodzaje zależności}

  Klasy typów to rodzaje zależności!

  \vspace{1em}
  \begin{lstlisting}
    type Build c i k v = Tasks c k v -> k -> Store i k v -> Store i k v
    newtype Task c k v = Task (forall f. c f => (k -> f v) -> f v)
  \end{lstlisting}

  \begin{itemize}
  \item \haskinl{f = Functor} -- sekwencyjne obliczenia
  \item \haskinl{f = Applicative} -- statyczne zależności
  \item \haskinl{f = Monad} -- dynamiczne zależności
  \item \haskinl{f = Selective} -- opcje wyboru znane statycznie
  \end{itemize}
\end{frame}

\begin{frame}[fragile]
  \frametitle{Określanie zależności}

  \begin{lstlisting}
    dependencies :: Task Applicative k v -> [k]
    dependencies task = getConst $ run task (\k -> Const [k]) where
      run :: c f => Task c k v -> (k -> f v) -> f v
      run (Task task) fetch = task fetch
  \end{lstlisting}

  \begin{lstlisting}
    track :: Monad m => Task Monad k v -> (k -> m v) -> m (v, [(k, v)])
    track task fetch = runWriterT $ run task trackingFetch
      where
        trackingFetch :: k -> WriterT [(k, v)] m v
        trackingFetch k = do v <- lift (fetch k); tell [(k, v)]; return v
  \end{lstlisting}
\end{frame}

\begin{frame}[fragile]
  \frametitle{Klasyfikacja}
  \begin{center}
    System kompilacji = planista + rekompilator
  \end{center}
  
  \begin{lstlisting}
    type Scheduler c i ir k v = Rebuilder c ir k v -> Build c i k v
    type Rebuilder c ir k v = k -> v -> Task c k v -> Task (MonadState ir) k v
  \end{lstlisting}

  %% \begin{columns}[t]
  %%   \column{.5\textwidth}
  %%   \begin{itemize}
  %%   \item topologiczny
  %%   \item restartujący
  %%   \item wstrzymujący
  %%   \end{itemize}

  %%   \column{.5\textwidth}
  %%   \begin{itemize}
  %%   \item brudny bit
  %%   \item ślady weryfikujące
  %%   \item ślady konstruktywne
  %%   \item głębokie ślady konstruktywne
  %%   \end{itemize}
  %% \end{columns}

  {\small
  \vspace{-0.5em}
  \begin{center}
  \begin{tabular}{r | c c c}
    \hline
                                 & \multicolumn{3}{c}{Planista} \\
    Rekompilator                 & Topologiczny & Restartujący & Wstrzymujący \\
    \hline
    Brudny bit                   & Make         & Excel        & - \\
    Ślady weryfikujące           & Ninja        & -            & Shake \\
    Ślady konstruktywne          & CloudBuild   & Bazel        & -\\
    Głębokie ślady konstrukcyjne & Buck         & -            & Nix \\
    \hline
  \end{tabular}
  \end{center}}
\end{frame}
