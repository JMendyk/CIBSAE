\documentclass[polish, 13pt]{beamer}

\usepackage[T1]{fontenc}
\usepackage[polish]{babel}
\usepackage[utf8]{inputenc}
\usepackage[justification=centering]{caption}
\usepackage{subcaption}
\usepackage{booktabs}
\usepackage{array}
\usepackage{amsmath}
\usepackage{datetime}

%% Pomocnicze makra związane z tematem pracy

\newcommand{\BSaLC}{,,Build systems {\`a} la carte''}
\newcommand{\BSaLCTP}{,,Build systems {\`a} la carte: Theory and practice''}

%% \usetheme{Warsaw}
\usetheme{CambridgeUS}

\makeatother
\setbeamertemplate{footline}
{
  \leavevmode%
  \hbox{%
  \begin{beamercolorbox}[wd=.2\paperwidth,ht=2.25ex,dp=1ex,center]{author in head/foot}%
    \usebeamerfont{author in head/foot}\insertshortauthor
  \end{beamercolorbox}%
  \begin{beamercolorbox}[wd=.6\paperwidth,ht=2.25ex,dp=1ex,center]{title in head/foot}%
    \usebeamerfont{title in head/foot}\insertshorttitle%\hspace*{3em}
  \end{beamercolorbox}%
  \begin{beamercolorbox}[wd=.2\paperwidth,ht=2.25ex,dp=1ex,right]{author in head/foot}%
    \usebeamerfont{title in head/foot}
    \ddmmyyyydate\insertshortdate\hspace*{1em}
    \insertframenumber{} / \inserttotalframenumber\hspace*{1ex}
  \end{beamercolorbox}}%
  \vskip0pt%
}
\makeatletter
%% \setbeamertemplate{navigation symbols}{}

\title[Systemy kompilacji z użyciem efektów algebraicznych i uchwytów]{Kwalifikacja i implementacja systemów kompilacji z użyciem efektów algebraicznych}

\author{Jakub Mendyk}
\date{\today}
\institute[]{Instytut Informatyki Uniwersytetu Wrocławskiego}

%% \AtBeginSection[]{
%%   \begin{frame}
%%   \vfill
%%   \centering
%%   \begin{beamercolorbox}[sep=8pt,center,shadow=true,rounded=true]{title}
%%     \usebeamerfont{title}\insertsectionhead\par%
%%   \end{beamercolorbox}
%%   \vfill
%%   \end{frame}
%% }

\begin{document}

\begin{frame}
\titlepage
\end{frame}

\begin{frame}
\frametitle{Plan prezentacji}
\tableofcontents
\end{frame}

\section{Wstęp pracy}

\subsection{Problemy z efektami ubocznymi}

\newcommand\pro{\item[\textcolor{example text.fg}{$+$}]}
\newcommand\con{\item[\alert{$-$}]}

\begin{frame}
  \frametitle{Efekty uboczne}
  %% \begin{columns}
  %% \column{0.45\textwidth}
  \begin{exampleblock}{Zalety}
    \begin{itemize}
      \pro komunikacja z innymi systemami
      \pro trwała pamięć -- system plików, bazy danych
      \pro interaktywność
    \end{itemize}
  \end{exampleblock}
  %% \column{0.45\textwidth}
  \begin{alertblock}{Wady}
    \begin{itemize}
      \con zależność od świata zewnętrzengo
      \con utrudnione rozumienie, brak modularności
      \con częstsze pomyłki
    \end{itemize}
  \end{alertblock}
  %% \end{columns}
  %% \pause
  %% \vspace{.5em}
\end{frame}

\begin{frame}
  \frametitle{Efekty uboczne są problematyczne}
  \begin{exampleblock}{Pomysł}
    Rozdzielić program na część czystą oraz część mającą efekty uboczne.
  \end{exampleblock}
  \pause
  \vspace{1em}
  \begin{alertblock}{Problem}
    Musimy zaufać autorowi, że funkcja rzeczywiście nie powoduje efektów ubocznych.
  \end{alertblock}
\end{frame}

\subsection{Radzenie sobie z efektami ubocznymi}

\begin{frame}
  \frametitle{Radzenie sobie z efektami ubocznymi}
  %% \begin{block}{Problem}
    Potrzebujemy znaleźć kogoś, kto będzie pilnował czy funkcje, które twierdzą że nie mają efektów ubocznych rzeczywiście takie są.
  %% \end{block}
  \pause
  \begin{exampleblock}{Pomysł}
    Wykorzystajmy system typów -- jest dobry w sprawdzaniu czy deklaracje programisty (adnotacje typów) są zgodne ze stanem faktycznym (implementacjami funkcji). Inferencja wyręczy nas od potrzeby pisania typów w wielu przypadkach (w przeciwieństwie do np. języka C).
  \end{exampleblock}
\end{frame}

\begin{frame}
  \frametitle{Monady}
  \begin{itemize}
  \pro umożliwiają bezpieczne programowanie z efektami
  \pro informacje o efektach ubocznych w sygnaturze
  \pro efekty nie mogą ,,uciec''
  %% \con niewygodne gdy trzeba użyć kilku efektów naraz -- potrzeba transformerów monad
  \con potrzeba transformerów monad by użyć wielu efektów naraz
  \con modularność wciąż problematyczna
  \end{itemize}
\end{frame}

\begin{frame}
  \frametitle{Efekty algebraiczne i uchwyty}
  \begin{itemize}
    \pro umożliwiają bezpieczne programowanie z efektami
    \pro informacje o efektach ubocznych w sygnaturze
    \pro efekty nie mogą ,,uciec''
    %% \pro wykorzystanie systemu typów
    \pro łatwość użycia wielu efektów jednocześnie
    \pro modularność i przejrzystość
  \end{itemize}
\end{frame}

\subsection{Systemy kompilacji}

\begin{frame}
  \frametitle{Systemy kompilacji}
  \begin{itemize}
  \item interakcja z zewnętrznymi zasobami
  \item uznawane za zło konieczne, zbyt skomplikowane
  \item powszechne wykorzystanie w ,,przemyśle''
  \item rzadko obiekt zainteresowań badaczy
  \end{itemize}
\end{frame}

\begin{frame}
  \begin{figure}
    \includegraphics[width=0.8\paperwidth, height=0.8\paperheight, keepaspectratio,
      trim=45 295 45 80, clip]{../pdfs/build-systems.pdf}
    \caption*{\begin{tiny}{https://dl.acm.org/doi/pdf/10.1145/3236774}
      \end{tiny}}
  \end{figure}
\end{frame}

\end{document}
