\documentclass[polish, 13pt, usenames, dvipsnames]{beamer}

%% \usepackage[]{xcolor}

\usepackage[T1]{fontenc}
\usepackage[polish]{babel}
\usepackage[utf8]{inputenc}
\usepackage[justification=centering]{caption}
\usepackage{subcaption}
\usepackage{booktabs}
\usepackage{array}
\usepackage{datetime}
\usepackage{amsmath, amsfonts, amsthm, xfrac}
\usepackage{listingsutf8}

%% Pomocnicze makra związane z tematem pracy

%% \theoremstyle{definition} \newtheorem*{definition}{Definicja}

\lstdefinestyle{haskell-style}{
  language=Haskell,
  inputencoding=utf8,
  extendedchars=true,
  alsodigit=!\$\%&*+-./:<=>?@^_~,
  sensitive=true,
  morecomment=[l]{--},
  %% morecomment=[s]{(*}{*)},
  morestring=[b]",
  basicstyle=\scriptsize\ttfamily,
  keywordstyle=\bf\ttfamily\color[rgb]{0,.3,.7},
  commentstyle=\color[rgb]{0.133,0.545,0.133},
  stringstyle={\color[rgb]{0.75,0.49,0.07}},
  morekeywords={Applicative, Selective, State, MonadState, Store, Task, Tasks, Build, Scheduler, Rebuilder},
  upquote=true,
  breaklines=true,
  breakatwhitespace=true,
  columns=fullflexible,
  keepspaces=true,
  showstringspaces=false,
  literate=*{`}{{`}}{1}
            {=>}{$\Rightarrow{}$}{1}
            {->}{$\rightarrow{}$}{1}
}

\lstdefinestyle{haskell-inl}{
  style=haskell-style,
  basicstyle=\small\ttfamily
}

\lstdefinelanguage{Haleff}{
  morekeywords=[1]{forall, data, let, rec, and, in, val, type, Type, signature, effect, Effect, fn, if, then, else},
  morekeywords=[2]{match, handle, handler, resume, with, end, return, finally, open },
%  lineskip=-0.1ex,
  alsodigit=!\$\%&*+-./:<=>?@^_~,
  sensitive=true,
  morecomment=[l]{//},
  morecomment=[s]{(*}{*)},
  morestring=[b]",
  basicstyle=\scriptsize\ttfamily,
  keywordstyle=\bf\ttfamily\color[rgb]{0,.3,.7},
  commentstyle=\color[rgb]{0.133,0.545,0.133},
  stringstyle={\color[rgb]{0.75,0.49,0.07}},
  upquote=true,
  breaklines=true,
  breakatwhitespace=true,
  columns=fullflexible,
  keepspaces=true,
  literate=*{`}{{`}}{1}
            {'}{{'}}{1}
}

\lstdefinestyle{Haleff-inl}{
  language=Haleff,
  basicstyle=\normalsize\ttfamily
}

\lstdefinestyle{Haleff-long}{
  language=Haleff,
  basicstyle=\tiny\ttfamily
}

\newcommand{\BSaLC}{,,Build systems {\`a} la carte''}
\newcommand{\BSaLCTP}{,,Build systems {\`a} la carte: Theory and practice''}

\newcommand{\haskinl}[1]{\lstinline[style=haskell-inl]{#1}}
\newcommand{\helinl}[1]{\lstinline[style=Haleff-inl]{#1}}

%% \usetheme{Warsaw}
\usetheme{CambridgeUS}

%% Blocks style as in Warsaw theme (visible boxes, colored background, header background)
\usecolortheme{orchid}

\makeatother
\setbeamertemplate{footline}
{
  \leavevmode%
  \hbox{%
  \begin{beamercolorbox}[wd=.2\paperwidth,ht=2.25ex,dp=1ex,center]{author in head/foot}%
    \usebeamerfont{author in head/foot}\insertshortauthor
  \end{beamercolorbox}%
  \begin{beamercolorbox}[wd=.6\paperwidth,ht=2.25ex,dp=1ex,center]{title in head/foot}%
    \usebeamerfont{title in head/foot}\insertshorttitle%\hspace*{3em}
  \end{beamercolorbox}%
  \begin{beamercolorbox}[wd=.2\paperwidth,ht=2.25ex,dp=1ex,right]{author in head/foot}%
    \usebeamerfont{title in head/foot}
    \ddmmyyyydate\insertshortdate\hspace*{1em}
    \insertframenumber{} / \inserttotalframenumber\hspace*{1ex}
  \end{beamercolorbox}}%
  \vskip0pt%
}
\makeatletter
%% \setbeamertemplate{navigation symbols}{}

\title[Systemy kompilacji z użyciem efektów algebraicznych i uchwytów]{Kwalifikacja i implementacja systemów kompilacji z użyciem efektów algebraicznych}

\author{Jakub Mendyk}
\date{\today}
\institute[]{Instytut Informatyki Uniwersytetu Wrocławskiego}

\begin{document}

\begin{frame}
\titlepage
\end{frame}

\begin{frame}
\frametitle{Plan prezentacji}
\tableofcontents
\end{frame}

\chapter{Introduction}

\section{Troubles with computational effects}

Computer programs -- thanks to their ability to interact with external resources such as storage, computer networks or human users -- can do way more than just perform predefined computations. However, this causes their runtime behaviour to depend on the external world in which these resources live and makes the programs' not just sequences of pure computations but also accompanying side effects.

Moreover, computational effects make it substantially harder to understand and reason about programs' behaviour and their correctness which limits their modularity and leads to more frequent introduction of mistakes and bugs by the authors. To avoid such issues, it is often attempted to split the programs into pure and impure parts while minimising the size of the latter. Despite these efforts, it is not trivial to tell if certain module of a program performs only pure computations and we often have to resort to trusting the author that it is true indeed.

\section{Dealing with computational effects}

One of the methods of solving that problem is inserting information about side effects into the type system. We can then use mechanisms of type inference and verification for automatic identification of functions that are not pure -- making it easy for a programmer to tell, from the function's signature, which effects might appear during function's runtime. Well known example of such solution are monads in Haskell programming language. Unfortunately, concurrent use of two independent resources represented by different monads is non-trivial and requires additional structures, such as monad transformers, which bring in additional challenges -- the program of modularity has only been shifted into other space.

On the other hand, there is a new attempt to deal with side effects with the help of type systems called algebraic effects and handlers. From the surface, they seem similar to exceptions known in many programming languages or system calls in operating systems. However, due to the split between definitions of effectful operations and their semantics, and an interesting use of continuations, they give the programmer ease of thinking and reasoning about the programs using them. And in contrast to monads, multiple algebraic effects can be freely used at once.

\section{Build systems}

To find a fine example of computer programs, whose main task is interaction with external resources, you can look no further than at build systems. There the user provides a set of mutually-depended tasks with information how to execute each of them by using results of other tasks, and the systems is responsible for their correct ordering and execution. Furthermore, we expect the build system to track changes in inputs and -- when asked to update the results -- rerun only the tasks whose output values will change. An examples of such systems are Make and -- which might seem surprising -- office applications for editing spreadsheets (such as popular Excel).

In recent publications titled \BSaLC{} \cite{mokhov2018build, mokhov2020build}, authors introduce a method of categorising build systems by taking into account how they determine order of task execution and in which way tasks are identified as requiring a rebuild. Presented categorisation leads the authors to introduction of a framework for creating build systems with expected properties which happens to be easily implementable in Haskell. What is more, it turns out that Applicative and Monad type classes correspond to the possible level of complexity of dependencies between tasks.

\section{About this paper}

This work aims to introduce the reader -- who has experience with Haskell and basics of functional programming languages -- to the innovative solution which are algebraic effects and handlers, and to demonstrate -- by following Mokhov et al -- an implementation of build systems using algebraic effects and handlers in a Helium programming language. As the result, it is possible to compare those two implementations, and to see how programming with algebraic effects and handlers looks like.

In the second chapter, we introduce simple and informal model of calculations that uses algebraic effects and handlers. Some examples of representing standard computational effects in our model are described.

Chapter three serves as an introduction to \BSaLC{}, discussion of observations made be the authors and description of abstraction of build systems and their consequences. Contents of the source publication is presented on a level sufficient to understand implementation of build systems with algebraic effects and handlers in chapter five. At the same time, the reader is encouraged to get acquainted with full contents of the work by Mokhov et al because it is an interesting and easy to read publication.

The fourth chapter begins with enumeration of existing programming languages and libraries that enable programming with algebraic effects and handlers. Next, the reader is introduced to the Helium programming language and presented with sample problems and their solutions that use effects and handlers. Moreover, use of multiple effects at the same time is demonstrated -- which is significantly easier than with monads in Haskell.

Finally, the crowning fifth chapter contains implementations of schedulers, rebuilders and build systems inspired by the results of \BSaLC{}. However, this time in a programming language with algebraic effects and handlers. Presented are the differences between abstract types from which the implementations are derived, and how use of effects and handlers influence the final form. Showcased result is missing one of the schedulers, thus and explanation is provided why it is so.




\section{Efekty algebraiczne i uchwyty}

\subsection{W teorii}

\chapter{O efektach algebraicznych teoretycznie}

Wprowadzimy notację służącą opisowi prostych obliczeń, która pomoże nam -- bez zanurzania się głęboko w ich rodowód matematyczny -- zrozumieć jak prostym, a jednocześnie fascynującym tworem są efekty algebraiczne i uchwyty. Przedstawiona notacja jest intencjonalnie nieformalna, gdyż ma w dostępny sposób przedstawić abstrakcyjny opis obliczeń z efektami bez prezentowania konkretnego języka programowania.

Następnie przyjrzymy się, jak możemy zapisać popularne przykłady efektów ubocznych używając naszej notacji. Na koniec, czytelnikowi zostaną polecone zasoby do dalszej lektury, które rozszerzają opis z tego rozdziału.

\section{Notacja}

\newcommand{\return}[1]{\mathbf{return}\ #1}
\newcommand{\op}[3]{#1(#2, #3)}
\newcommand{\opi}[3]{\op{op_{#1}}{#2}{#3}}
\newcommand{\handle}[2]{\mathbf{handle}\ #1\ \mathbf{with}\ #2}
\newcommand{\hcase}[3]{#1\ #2\ \Rightarrow\ #3}
\newcommand{\fun}[2]{\lambda #1.\ #2}
\newcommand{\eval}[1]{\llbracket\, #1\, \rrbracket}
\newcommand{\cond}[3]{\mathbf{if}\ #1\ \mathbf{then}\ #2\ \mathbf{else}\ #3}

Będziemy rozważać obliczenia nad wartościami następujących trzech typów:
\begin{itemize}
\item boolowskim \(B\) -- z wartościami \(T\) i \(F\) oraz standardowymi spójnikami logicznymi,
\item liczb całkowitych \(\mathbb{Z}\) -- wraz z ich relacją równości oraz podstawowymi działaniami arytmetycznymi,
\item typem jednostkowym \(U\) -- zamieszkałym przez pojedynczą wartość \(u\),
\item oraz pary tychże typów.
\end{itemize}

% 
% Przemyślenia:
% 1. Pozbyć się "return v"
% 2. Dodać przypadek "return x" do zbioru w uchwycie, zdefiniować jego działanie
%    Można by wtedy rozszerzyć przykłady o zmianę wartości wynikowej
% 

Nasz model składać się będzie z wyrażeń:
\begin{itemize}
\item \(\return{v}\) -- gdzie \(v\) jest wyrażeniem boolowskim lub arytmetycznym,
\item \(\cond{v_1 = v_2}{e_t}{e_f}\) -- wyrażenie warunkowe, gdzie \(v_1 = v_2\) jest pytaniem o równość wartości dwóch wyrażeń arytmetycznych,
\item abstrakcyjnych operacji oznaczanych \(\{op_i\}_{i \in I}\) -- powodujących wystąpienie efektów ubocznych -- których działanie nie jest nam znane, zaś ich sygnatury to \(op_i: A \rightarrow (B \rightarrow C) \rightarrow D\), gdzie \(A\), \(B\), \(C\) oraz \(D\) to pewne typy w naszym modelu. Wyrażenie~\(\opi{i}{n}{\kappa}\) opisuje operację z argumentem \(n\) oraz dalszą częścią obliczenia \(\kappa\) parametryzowaną wynikiem operacji, które \textit{może (nie musi)} zostać wykonane po wykonaniu operacji,
\item uchwytów, czyli wyrażeń postaci \(\handle{e}{\{\ \hcase{op_i}{n\ \kappa}{h_i}\ \}_{i \in I}}\), gdzie \(e\) to inne wyrażenie; uchwyt definiuje działanie (dotychczas abstrakcyjnych) operacji. 
\end{itemize}

Przykładowymi obliczeniami w naszej notacji są więc:
\begin{equation}
\begin{gathered}
  \return{0},\quad\return{2 + 2},\quad \opi{1}{2}{\fun{x}{\return{x + 1}}} \\
  \handle{\opi{1}{2}{\fun{x}{\return{x + 1}}}}{\{\ \hcase{op_1}{n\ \kappa}{\kappa\ (2 \cdot n)} \ \}}
\end{gathered}
\end{equation}

Dla czytelności, pisząc w uchwycie zbiór który nie przebiega wszystkich operacji, przyjmujemy że uchwyt nie definiuje działania operacji; równoważnie, zbiór wzbogacamy o element: \(\hcase{op_i}{n\ \kappa}{op_i(n, \kappa)}\).

Obliczanie wartości wyrażenia przebiega następująco:
\begin{itemize}
\item \(\eval{\return v} = v\) -- wartością \(\mathbf{return}\) jest wartość wyrażenia arytmetycznego,
\item \(\eval{(\fun{x}{e})\ y} = \eval{e \left[x / \eval{y}\right]}\) -- aplikacja argumentu do funkcji,
\item
  \(\begin{aligned}[t]
    \eval{\cond{v_1 = v_2}{e_t}{e_f}} = \left\{\begin{matrix}
    \eval{e_t} & \text{gdy }\eval{v_1} = \eval{v_2} \\ 
    \eval{e_f} & \text{wpp}
    \end{matrix}\right.
  \end{aligned}\)
%% \item \(\eval{\opi{i}{a}{f}} = \opi{i}{a}{f}\) -- obliczenie z efektem ubocznym nie może poczynić postępu póki nie ma określonego działania,
\item \(\eval{\handle{\return v}{H}} = \eval{\return v}\) -- uchwyt nie wpływa na wartość obliczenia, które nie zawiera efektów ubocznych,
\item \(\eval{\handle{\opi{i}{a}{f}}{H}} = \eval{\handle{h_i \left[n / \eval{a},\, \kappa / f\right]}{H}} \), gdzie \(H~=~\{\ \hcase{op_i}{n\ \kappa}{h_i} \ \}\), a \(h_i\) nie ma wystąpień \(op_i\).
  
\end{itemize}

Zobaczmy jak zatem wygląda obliczenie ostatniego z powyższych przykładów:
\begin{equation}\begin{split}
  \eval{\handle{\opi{1}{2}{\fun{x}{\return{x + 1}}}}{\{\ \hcase{op_1}{n\ \kappa}{\kappa\ (2 \cdot n)} \ \}}} &= \\
  \eval{\handle{(\fun{x}{\return{x+1}}) (2 \cdot 2)}{\{\ \hcase{op_1}{n\ \kappa}{\kappa\ (2 \cdot n)} \ \}}} &= \\
  \eval{\handle{\return{4+1}}{\{\ \hcase{op_1}{n\ \kappa}{\kappa\ (2 \cdot n)} \ \}}} &= \\
  \eval{\return{4 + 1}} &= 5
\end{split}\end{equation}


\section{Równania, efekt porażki i modyfikowalny stan}

Do tego momentu, nie przyjmowaliśmy żadnych założeń na temat operacji powodujących efekty uboczne. Uchwyty mogły w związku z tym działać w sposób całkowicie dowolny. Ograniczymy się w tej dowolności i nałożymy warunki na uchwyty wybranych operacji. Przykładowo, ustalmy że dla operacji \(op_r\), uchwyty muszą być takie aby następujący warunek był spełniony:
\begin{align}
  \forall n\ \forall e.\ \eval{\handle{op_r(n, \fun{x}{e})}{H}} = n
\end{align}

%% \pagebreak

Zauważmy, że istnieje tylko jeden naturalny uchwyt spełniający tej warunek, jest nim \(H = \{\ \hcase{op_r}{n\ \kappa}{n} \ \}\). Co więcej, jego działanie łudząco przypomina konstrukcję wyjątków w popularnych językach programowania:

\begin{lstlisting}
  try {
    raise 5;
    // ...
  } catch (int n) {
    return n;
  }
\end{lstlisting}

Podobieństwo to jest w pełni zamierzone. Okazuje się że nasz język z jedną operacją oraz równaniem ma już moc wystarczającą do opisu konstrukcji, która w większości popularnych języków nie może zaistnieć z woli programisty, a zamiast tego musi być dostarczona przez twórcę języka.

Rozważmy kolejny przykład. Dla poprawienia czytelności, zrezygnujemy z oznaczeń \(op_i\) na operacje powodujące efekty, zamiast tego nadamy im znaczące nazwy: \(get\) oraz \(put\). Operacje te mają sygnatury \(get: U \rightarrow (\mathbb{Z} \rightarrow \mathbb{Z}) \rightarrow \mathbb{Z}\), \(put: \mathbb{Z} \rightarrow (U \rightarrow \mathbb{Z}) \rightarrow \mathbb{Z}\). Spróbujemy wyrazić działanie tych dwóch operacji by otrzymać modyfikowalną komórkę pamięci. Ustalamy równania:

\begin{itemize}
\item \(\forall e.\ \eval{get(u, \fun{\_}{get(u, \fun{x}{e})})} = \eval{get(u, \fun{x}{e})}\)

  kolejne odczyty z komórki bez jej modyfikowania dają takie same wyniki,
\item \(\forall e.\ \eval{get(u, \fun{n}{put(n, \fun{u}{e})})} = \eval{e}\)

  umieszczenie w komórce wartości, która już tam się znajduje, nie wpływa na wynik obliczenia,
\item \(\forall n.\ \forall f.\ \eval{put(n, \fun{u}{get(u, \fun{x}{f\ x})})} = \eval{f\ n}\)

  obliczenie które odczytuje wartość z komórki daje taki sam wyniki, jak gdyby miało wartość komórki podaną wprost jako argument,
\item \(\forall n_1.\ \forall n_2.\ \forall e.\ \eval{put(n_1, \fun{u}{put(n_2, \fun{u}{e})})} = \eval{put(n_2, \fun{u}{e})}\)

  komórka zachowuje się, jak gdyby pamiętała jedynie najnowszą włożoną do niej wartość.
\end{itemize}

Zauważmy, że choć nakładamy warunki na zewnętrzne skutki działania operacji \(get\) oraz \(put\), to w żaden sposób nie ograniczyliśmy swobody autora w implementacji uchwytów dla tych operacji. % W rozdziale 4 przyglądniemy się kilku przykładom uchwytów realizujących te operacje.

% Chyba jednak nie. Chociaż można dodać przykład, który agreguje historyczne wartości komórki i zwraca je w parze z wynikiem obliczenia.

\section{Poszukiwanie sukcesu}

%% sygnatura amb powinna mieć B zamiast Bool

Kolejnym rodzajem efektu ubocznego, który rozważymy w tym rozdziale jest niedeterminizm. Chcielibyśmy wyrażać obliczenia, w których pewne parametry mogą przyjmować wiele wartości, a ich dobór ma zostać dokonany tak by spełnić pewien określony warunek. Przykładowo, mamy trzy zmienne \(x,\, y\) oraz \(z\) i chcemy napisać program sprawdzający czy formuła \(\phi(x, y, z)\) jest spełnialna. W tym celu zdefiniujemy operację \(amb: U \rightarrow (Bool \rightarrow Bool) \rightarrow \mathit{Bool}\) związaną z efektem niedeterminizmu. Napiszmy obliczenie rozwiązujące nasz problem:
\begin{equation}\begin{split}
  \handle{
    &\op{amb}{u}{\fun{x}{
        \op{amb}{u}{\fun{y}{
            \op{amb}{u}{\fun{z}{
                \phi(x, y, z)
            }}
        }}
    }}\\
  }{ \{ \ &\hcase{amb}{u\ \kappa}{ \kappa\ (T) \ \mathbf{or} \ \kappa\ (F) } \ \} }
\end{split}\end{equation}

Gdy definiowaliśmy efekt wyjątku, obliczenie nie było kontynuowane. W przypadku niedeterminizmu kontynuujemy obliczenie dwukrotnie -- podstawiając za niedeterministycznie określoną zmienną wartości raz prawdy, raz fałszu -- w czytelny sposób sprawdzamy wszystkie możliwe wartościowania, a w konsekwencji określamy czy formuła jest spełnialna.

Możemy zauważyć, że gdybyśmy chcieli zamiast sprawdzania spełnialności, weryfikować czy formuła jest tautologią, wystarczy zmienić tylko jedno słowo -- zastąpić spójnik \(\mathbf{or}\) spójnikiem \(\mathbf{and}\) otrzymując nowy uchwyt:
\begin{equation}\begin{split}
  \handle{
    &\op{amb}{u}{\fun{x}{
        \op{amb}{u}{\fun{y}{
            \op{amb}{u}{\fun{z}{
                \phi(x, y, z)
            }}
        }}
    }}\\
  }{ \{ \ &\hcase{amb}{u\ \kappa}{ \kappa\ (T) \ \mathbf{and} \ \kappa\ (F) } \ \} }
\end{split}\end{equation}

Przedstawiona konstrukcja efektów, operacji i uchwytów tworzy dualny mechanizm w którym operacje są producentami efektów, a uchwyty ich konsumentami. Zabierając źródłom efektów ubocznych ich konkretne znaczenia semantyczne, lub nakładając na nie jedynie proste warunki wyrażone równaniami, otrzymaliśmy niezwykle silne narzędzie umożliwiające proste, deklaratywne oraz -- co najważniejsze, w kontraście do popularnych języków programowania -- samodzielne konstruowanie zaawansowanych efektów ubocznych.

\section{Dalsza lektura}

Rozdział ten miał na celu w lekki sposób wprowadzić idee, definicje i konstrukcje związane z efektami algebraicznymi i uchwytami, które będą fundamentem do zrozumienia ich wykorzystania w praktycznych przykładach oraz implementacji systemów kompilacji w dalszych rzodziałach. Czytelnicy zainteresowani głębszym poznaniem historii oraz rodowodu efektów algebraicznych i uchwytów mogą zapoznać się z następującymi materiałami:

\begin{itemize}
\item ,,An Introduction to Algebraic Effects and Handlers'' autorstwa Matija Pretnara \cite{pretnar2015introduction},
\item notatki oraz seria wykładów Andreja Bauera pt. ,,What is algebraic about algebraic effects and handlers?'' \cite{bauer2018algebraic} dostępne w formie tekstowej oraz nagrań wideo w serwisie YouTube,
\item prace Plotkina i Powera \cite{plotkin2001semantics, plotkin2002computational} oraz Plotkina i Pretnara \cite{plotkin2013handling} -- jeśli czytelnik chce poznać jedne z pierwszych wyników prowadzących do efektów algebraicznych oraz wykorzystania uchwytów,
\item społeczność skupiona wokół tematu efektów algebraicznych agreguje zasoby z nimi związane w repozytorium \cite{effectsbibliography} w serwisie GitHub.

\end{itemize}


\subsection{W praktyce}

\newcommand{\inl}[1]{\lstinline[style=Haleff-inl]{#1}}

\chapter{Efekty algebraiczne i uchwyty w~praktyce}

\section{Języki programowania z efektami algebraicznymi}

Zainteresowanie efektami algebraicznymi oraz uchwytami doprowadziło do powstania w ostatnich latach wielu bibliotek dla języków popularnych w środowisku akademickim i pasjonatów języków funkcyjnych -- Haskella (extensible-effects\footnote{\url{https://hackage.haskell.org/package/extensible-effects}},
fused-effects\footnote{\url{https://hackage.haskell.org/package/fused-effects}},
polysemy\footnote{\url{http://hackage.haskell.org/package/polysemy}}), Scali
(Effekt\footnote{\url{https://github.com/b-studios/scala-effekt}},
atnos-org/eff\footnote{\url{https://github.com/atnos-org/eff}})
i Idris (Effects \footnote{\url{https://www.idris-lang.org/docs/current/effects_doc/}}).

Związana z językiem OCaml jest inicjatywa ocaml-multicore\footnote{\url{https://github.com/ocaml-multicore/ocaml-multicore/wiki}}, której celem jest stworzenie implementacji OCamla ze wsparciem dla współbieżności oraz współdzielonej pamięci, a cel ten jest realizowany przez wykorzystanie konceptu efektów i uchwytów.

Badania nad efektami i uchwytami przyczyniły się także do powstania kilku eksperymentalnych języków programowania w których efekty i uchwyty są obywatelami pierwszej kategorii. Do języków tych należą:
\begin{itemize}
\item Eff\footnote{\url{https://www.eff-lang.org/}} -- powstający z inicjatywy Andreja Bauera i Matija Pretnara język o ML-podobnej składni,
\item Frank\footnote{\url{https://github.com/frank-lang/frank}} \cite{DBLP:journals/corr/LindleyMM16} -- pod przewodnictwem Sama Lindley'a, Conora McBride'a oraz Craiga McLaughlin'a, projektowany z tęsknoty do ML'a, a jednocześnie upodobania do Haskell-owej dyscypliny,
\item Koka\footnote{\url{https://github.com/koka-lang/koka}} -- kierowany przez Daana Leijena z Microsoft projekt badawczy; Koka ma składnię inspirowaną JavaScriptem,
\item Helium\footnote{\url{https://bitbucket.org/pl-uwr/helium/src/master/}} \cite{biernacki2019abstracting} -- powstały w Instytucie Informatyki Uniwersytetu Wrocławskiego, z ML-podobnym systemem modułów i lekkimi naleciałościami z Haskella.
\end{itemize}

\section{Helium}

\lstset{language=Haleff, showstringspaces=false}

Używając właśnie języka Helium zobaczymy, jak w praktyce wygląda programowanie z efektami algebraicznymi oraz uchwytami, zaś w następnym rozdziale spróbujemy zaimplementować wyniki uzyskane w ,,Build systems {\`a} la carte'' \cite{mokhov2018build, mokhov2020build}. Po raz pierwszy Helium pojawia się w \cite{biernacki2019abstracting}, służąc za narzędzie do eksperymentowania i umożliwienia konstrukcji bardziej skomplikowanych przykładów oraz projektów w celu przetestowania efektów i uchwytów w praktyce.

Rozważmy przykład prostego programu napisanego w Helium, w którym definiujemy pomocniczą funkcję \inl{is_negative} ustalającą, czy liczba jest ujemna oraz funkcję \inl{question}, która pyta użytkownika o liczbę i informuje, czy liczba ta jest ujemna:

\lstinputlisting{code_examples/syntax.he}

Sygnatura funkcji \inl{is_negative} wyznaczona przez system typów Helium -- to jak łatwo się domyślić -- \inl{Int -> Bool}. Gdy jednak zapytamy środowisko uruchomieniowe o typ funkcji \inl{question} otrzymamy interesującą sygnaturę \inl{Unit ->[IO] Unit}. W Helium informacje o efektach występujących w trakcie obliczania funkcji są umieszczone w sygnaturach funkcji w kwadratowych nawiasach. W przypadku funkcji \inl{question}, jej obliczenie powoduje wystąpienie efektu ubocznego związanego z mechanizmem wejścia/wyjścia. 

\begin{lstlisting}
printStr: String ->[IO] Unit
readInt: Unit ->[IO] Int
\end{lstlisting}

System inferencji typów wiedząc, że operacje we/wy są zadeklarowane z powyższymi sygnaturami wnioskuje, że skoro wystąpienia tychże operacji w kodzie \inl{question} nie są obsługiwane przez uchwyt, to efekt \inl{IO} wyjdzie poza tą funkcję.

Efekty \inl{IO} oraz \inl{RE} (runtime error) są szczególne, gdyż są dla nich zadeklarowane globalne uchwyty w bibliotece standardowej -- jeśli efekt nie zostanie obsłużony i dotrze do poziomu środowiska uruchomieniowego, to ono zajmie się jego obsługą. Dla efektu \inl{IO} środowisko skorzysta ze standardowego wejścia/wyjścia, zaś w przypadku wystąpienia efektu \inl{RE}, obliczenie zostanie przerwane ze stosownym komunikatem błędu.

\section{Przykłady implementacji uchwytów}

\subsection{Błąd}

Zaimplementujemy kilka efektów ubocznych, zaczynając od efektu błędu, wraz z uchwytami dla nich. W Helium efekt oraz powodujące go operacje definiuje się następująco:

\lstinputlisting{code_examples/error1__signature.he}

Stwórzmy funkcję podobną do \inl{question} z tym, że nie będzie ona ,,lubić'' wartości ujemnych:

\lstinputlisting[firstline=7, lastline=18]{code_examples/error2__inline_abort.he}

Zdefiniowaliśmy efekt uboczny \inl{Error} wraz z operacją \inl{error}, która go powoduje. Operacja ta jest parametryzowana wartością typu \inl{Unit}, a jej (możliwy) wynik to także wartość z \inl{Unit}. Definiujemy też funkcję \inl{main}, w której wywołujemy \inl{no\_negatives\_question}. Jednakże obliczenie wykonujemy w uchwycie, w którym definiujemy co ma się wydarzyć, gdy w czasie obliczenia wystąpi efekt błędu spowodowany operacją \inl{error}. W tym przypadku mówimy, że będzie to skutkowało wypisaniem wiadomości na standardowe wyjście. Nie wznawiamy obliczenia, stąd wystąpienie błędu skutkuje zakończeniem nadzorowanego obliczenia. Jeśli uruchomimy teraz program i podamy ujemną liczbę, zakończy się on komunikatem zdefiniowanym w uchwycie, a tekst ,,Question finished'' nie zostanie wypisany. Zgodnie z oczekiwaniami -- obliczenie \inl{no\_negatives\_question} nie zostało kontynuowane po wystąpieniu błędu.

Jeśli pewnego uchwytu zamierzamy używać wiele razy, możemy przypisać mu identyfikator -- w Helium uchwyty są wartościami:

\lstinputlisting[firstline=7, lastline=10]{code_examples/error3__reused_handler.he}

zmodyfikujmy funkcję \inl{main} by korzystać ze zdefiniowanego uchwytu:

\lstinputlisting[firstline=12, lastline=13]{code_examples/error3__reused_handler.he}

Na potrzeby przykładu możemy rozważyć ,,spokojniejszy'' uchwyt dla wystąpień \inl{error}, który wypisze ostrzeżenie o wystąpieniu błędu ale będzie kontynuował obliczenie:

\lstinputlisting[firstline=7, lastline=10]{code_examples/error4__warn_not_abort.he}

Jeśli skorzystamy z tego uchwytu w programie, po wyświetleniu ostrzeżenia obliczenie \inl{no\_negatives\_question} zostanie wznowione i na ekranie zobaczymy komunikat ,,Question finished''. Specjalna funkcja \inl{resume}, dostępna w uchwycie reprezentuje kontynuację obliczenia, które zostało przerwane wystąpieniem operacji powodującej efekt uboczny.

\subsection{Niedeterminizm}

Powróćmy do problemu, który w rozdziale drugim był inspiracją do rozważania niedeterminizmu -- sprawdzanie czy formuła jest spełnialna oraz czy jest tautologią. Przedstawiliśmy wtedy uchwyty dla obu tych problemów w naszej notacji. Implementacja efektu niedeterminizmu, operacji \inl{amb} oraz uchwytów wraz z wykorzystaniem ich wygląda następująco:

\lstinputlisting[lastline=22]{code_examples/nondet1__simple.he}

Będziemy sprawdzać, czy formuła wyrażona za pomocą funkcji \inl{formula1} jest spełnialna. W tym celu w \inl{main} -- wewnątrz uchwytu -- niedeterministycznie ustalamy wartości zmiennych \inl{x}, \inl{y}, \inl{z}, po czym obliczamy wartość \inl{formula1}. Wartość obsługiwanego wyrażenia, którą przypisujemy do zmiennej \inl{ret}, jest następnie wykorzystana do wypisania komunikatu. Ponadto -- w celu demonstracji możliwości języka -- w uchwytach zamiast kontynuować obliczenie używając \inl{resume}, przypisujemy kontynuacji nazwę \inl{r}.

W Helium uchwyty mogą posiadać przypadki nie tylko dla operacji związanych z jakimś efektem ale także dwa specjalne: \inl{return} oraz \inl{finally}. Pierwszy jest wykonywany, gdy obliczenie pod kontrolą uchwytu kończy się zwracając wynik. Przypadek \inl{return} jako argument otrzymuje wynik obliczenia. Zaś \inl{finally} otrzymuje jako argument obliczenie obsługiwane przez uchwyt i jest uruchamiane na początku działania uchwytu. Domyślnie przypadki te są zaimplementowane jako:

\begin{lstlisting}
handler
| return x => x
| finally f => f
end
\end{lstlisting}

Możemy je jednak sprytnie wykorzystać. Przykładowo, zamiast tylko sprawdzać czy formuła jest spełnialna, możemy sprawdzić przy ilu wartościowaniach jest prawdziwa:

\lstinputlisting[firstline=5, lastline=16]{code_examples/nondet2__count_sats.he}

Gdy obliczenie się kończy -- zamiast zwracać, czy formuła jest spełniona -- zwracamy 1 albo 0, w zależności, czy formuła przy aktualnym wartościowaniu jest spełniona. Gdy obsługujemy niedeterministyczny wybór, kontynuujemy obliczenie dla obu możliwych wartości boolowskich po czym dodajemy wyniki. Wykorzystując \inl{finally} możemy włączyć komunikat o liczbie wartościowań do uchwytu:

\lstinputlisting[firstline=5, lastline=17]{code_examples/nondet3__count_and_write_sats.he}

Tutaj wykorzystanie \inl{finally} jest lekkim nadużyciem, jak jednak za chwilę zobaczymy, konstrukcja ta jest bardzo przydatna.

\subsection{Modyfikowalny stan}

Rozważmy następujący przypadek dla \inl{return} w uchwycie:

\begin{lstlisting}
handler
(* ... *)
| return x => fn s => x
end
\end{lstlisting}

Wartość obliczenia, zamiast być jego wynikiem, jest funkcją. Co za tym idzie, w tym uchwycie kontynuacje nie będą funkcjami zwracającymi wartości, lecz funkcje. W ten sposób możemy parametryzować dalsze obliczenia nie tylko wartościami zwracanymi przez operacje (zgodnie z ich sygnaturą), ale także wymyślonymi przez nas -- autorów uchwytu. Zauważmy jednak, że parametr ten nie jest widoczny w obsługiwanym obliczeniu, a jedynie w uchwycie. Co więcej, skoro wynik obsługiwanego obliczenia jest teraz funkcją, a nie wartością, to -- by użytkownik uchwytu nie zauważył niezgodności typów -- musimy funkcję tą uruchomić z jakimś parametrem. Tutaj właśnie przychodzi naturalny moment na wykorzystanie konstrukcji \inl{finally}.

Definiujemy efekt stanu z operacją jego odczytu oraz modyfikacji:

\lstinputlisting[lastline=3]{code_examples/state1__basics.he}

Efekt, jak i operacje są polimorficzne ze względu na typ wartości stanu. Zdefiniujemy teraz standardowy uchwyt dla efektu stanu. Skorzystamy z faktu, że uchwyty są w Helium wartościami, stąd w szczególności mogą być wynikiem funkcji. Funkcja ta będzie u nas parametryzowana wartością początkową stanu:

\lstinputlisting[firstline=5, lastline=11]{code_examples/state1__basics.he}

Gdy obliczenie się kończy, zamiast wartość zwracamy funkcję, która ignoruje argument (będzie nim bieżąca wartością stanu), a zwraca właściwy wynik obliczenia. W konsekwencji przypadki dla operacji też muszą być funkcjami. Dla \inl{put} nie musimy odczytywać aktualnej wartości stanu, stąd wartość tą ignorujemy. Obliczenie wznawiamy z wartością jednostkową. Jak jednak wiemy, wynikiem nie będzie zwykła wartość, lecz funkcja, której u nas dajemy wartość stanu. Stąd podajemy jej nowy stan, którym parametryzowana była operacja \inl{put}. W przypadku \inl{get} postępujemy podobnie -- jednak tym razem odczytamy argument funkcji i przekażemy go do kontynuacji. Niezmiennie kontynuacja zwraca funkcję, której przekażemy aktualną wartość stanu. Pozostaje rozstrzygnąć, co zrobić w przypadku \inl{finally}. Skoro jednak przerobiliśmy obliczenie ze zwracającego wartość do takiego, które zwraca funkcję oczekującą wartości stanu, to możemy podać mu wartość początkową -- określoną przez użytkownika uchwytu.

Jeśli chcemy, aby obliczenie zwracało nie tylko wartość wynikową, ale także końcowy stan, wystarczy że zmodyfikujemy przypadek dla \inl{return}:

\lstinputlisting[firstline=5, lastline=11]{code_examples/state2__run_state.he}

Dzięki zdefiniowanemu efektowi ubocznemu, operacjom oraz uchwytom możemy teraz łatwo wykonywać obliczenia ze stanem:

\lstinputlisting[firstline=7, lastline=24]{code_examples/state3__example.he}

\subsection{Efekt rekursji}

W niektórych językach ML-podobnych (jak na przykład OCaml czy Helium) chcąc, by w ciele definicji funkcji był widoczny jej identyfikator, trzeba zadeklarować ją używając słów kluczowych \inl{let rec}:

\lstinputlisting[firstline=2, lastline=5]{code_examples/rec1__rec.he}

Co ciekawe, dzięki własnym efektom i operacjom możemy tworzyć funkcje rekurencyjne, które nie używają jawnie rekursji:

\lstinputlisting[firstline=2, lastline=13]{code_examples/rec2__effect.he}

Konstrukcja \inl{handle `a in ...} służy doprecyzowaniu, który efekt ma być obsłużony przez uchwyt -- jest przydatna w przypadku niejednoznaczności, gdy używamy wielu instancji tego samego efektu lub dla ułatwienia rozumienia kodu.

Korzystając z efektu rekursji, możemy także definiować funkcje wzajemnie rekurencyjne:

\lstinputlisting[firstline=5, lastline=23]{code_examples/rec3__mutual.he}

Utrzymujemy informację, która funkcja jest aktualnie wykonywana i gdy prosi o wywołanie rekurencyjne, uruchamiamy obliczanie drugiej funkcji, po czym wynik przekazujemy do kontynuacji.

% 
% Może jeszcze współbieżność kooperatywna?
%   (26.08.2020) może jednak nie :)
% Byłaby naturalnym przedłużeniem wzajemnej rekursji
% ale zawierałaby trochę szumu w postaci kolejkowania zadań.
%

\subsection{Wiele efektów naraz -- porażka i niedeterminizm}

Na koniec rozdziału zobaczymy jak łatwo w Helium komponuje się efekty. Definiujemy efekty niedeterminizmu oraz porażki:

\lstinputlisting[firstline=1, lastline=5]{code_examples/fail_and_amb.he}

oraz bardzo proste uchwyty dla tych efektów:

\lstinputlisting[firstline=7, lastline=15]{code_examples/fail_and_amb.he}

Definiujemy teraz funkcję sprawdzającą, czy otrzymana formuła z trzema zmiennymi wolnymi jest spełnialna:

\lstinputlisting[firstline=19, lastline=25]{code_examples/fail_and_amb.he}

Jeśli formuła przy ustalonym wartościowaniu nie jest spełniona, powoduje efekt porażki. Zwróćmy uwagę w jakiej kolejności są umieszczone uchwyty -- niedeterminizmu na zewnątrz, zaś porażki wewnątrz. W ten sposób, gdy wystąpi porażka, jej uchwyt zwróci fałsz, w wyniku czego nastąpi powrót do ostatniego punktu niedeterminizmu, w którym jest jeszcze wybór. Dzięki temu wartość \inl{is\_sat f} jest równa fałszowi tylko, gdy przy każdym wartościowaniu nastąpi porażka. Zobaczmy teraz funkcję sprawdzającą, czy otrzymana formuła jest tautologią:

\lstinputlisting[firstline=27, lastline=33]{code_examples/fail_and_amb.he}

Tutaj uchwyt dla porażki znajduje się na zewnątrz -- wystąpienie porażki oznacza, że istnieje wartościowanie przy którym formuła nie jest prawdziwa, a w konsekwencji nie może być tautologią. Możemy teraz napisać zgrabną funkcję, która wypisze nam czy \inl{formula1} jest spełnialna oraz czy jest tautologią:

\lstinputlisting[firstline=35, lastline=46]{code_examples/fail_and_amb.he}

Z łatwością napisaliśmy program, który korzysta z wielu efektów ubocznych jednocześnie, mimo że żaden z nich (ani uchwyty) nie wiedzą o istnieniu drugiego. Łączenie efektów jest bardzo proste, a kolejność w jakiej umieszczamy uchwyty umożliwia nam łatwe i czytelne definiowanie zachowania programu w przypadku wystąpienie któregokolwiek z efektów.

Dzięki językowi Helium przyjrzeliśmy się z bliska efektom algebraicznym oraz uchwytom, zobaczyliśmy przykłady implementacji uchwytów oraz rozwiązań prostych problemów. Jesteśmy gotowi do podjęcia próby zaimplementowania systemów kompilacji z użyciem efektów i uchwytów -- czego dokonamy w następnym rozdziale.

\undef\inl





\newcommand{\inl}[1]{\lstinline[style=haskell-inl]{#1}}

\chapter{O systemach kompilacji (i ich klasyfikacji)}
\label{chapter-bsalc}

Systemy kompilacji, choć są wykorzystywane w praktycznie wszystkich projektach programistycznych, są przez ich użytkowników na ogół zaniedbywane, traktowane jak zło konieczne, a czasem nawet wywołują lęk oraz złość. Mimo tak dużej popularności i większym~--~niż mogłoby się wydawać~--~stopniu skomplikowania, nie cieszyły się specjalnym zainteresowaniem ze strony badaczy. Przyglądnęli się im jednak bliżej Andrey Mokhov, Neil Mitchell oraz Simon Peyton Jones w artykułach \BSaLC\cite{mokhov2018build} oraz \BSaLCTP\cite{mokhov2020build}. W tym rozdziale prześledzimy ich kroki i omówimy wyniki które otrzymali autorzy, aby w dalszej części tej pracy samodzielnie zaimplementować przedstawione systemy kompilacji w języku z efektami algebraicznymi oraz uchwytami.

\section{Przykłady systemów kompilacji}

Chcąc zrozumieć głębsze i nietrywialne relacje oraz podobieństwa między systemami kompilacji, przyglądnijmy się najpierw kilku przykładom takich systemów używanych w przemyśle.

\subsection{Make}

Make jest bardzo popularnym systemem kompilacji. Konfiguruje się go przez tworzenie plików zwanych \textit{makefile'ami}, które definiują zadania, zależności między nimi oraz sposób ich zbudowania. Przykładowa konfiguracja prezentuje się następująco:

\lstset{language=make}

\begin{lstlisting}
  util.o: util.h util.c
      gcc -c util.c

  main.o: util.h main.c
      gcc -c main.c

  main.exe: util.o main.o
      gcc util.o main.o -o main.exe
\end{lstlisting}

Konfiguracja ta definiuje sposób budowania trzech zadań: \textit{util.o}, \textit{main.o} oraz \textit{main.exe}. W linii zawierającej definicję zadania zawarta jest informacja o innych zadaniach, od których definiowane zależy~--~np. dowiadujemy się że \textit{util.o} zależy od zadań (tutaj: plików) \textit{util.h} oraz \textit{util.c}, a zadanie jest realizowane przez wykonanie polecenia \textit{gcc~\nobreakdash-c~util.c}. Jeśli zadanie nie ma zdefiniowanego sposobu zbudowania, na przykład \textit{util.h} mówimy, że jest wejściem lub zadaniem wejściowym w tej konfiguracji.

Wszystkie informacje o zależnościach między zadaniami są wyrażone w tym jednym pliku \textit{makefile}. Użytkownik, chcąc zbudować zadanie \textit{main.exe}, uruchamia program używając polecenia \textit{make~main.exe}. Po uruchomieniu system określi, które zadania mają zostać zbudowane, by zrealizować otrzymane żądanie. Z racji tego, że procedura budowania zadań przebiega tak samo, niezależnie od wyników podzadań, będziemy o takim systemie mówić, że ma statyczne zależności. Dla takich systemów naturalnym porządkiem, w którym zadania powinny być budowane jest porządek topologiczny. W ten sposób każde zadanie będzie wykonane ,,na świeżych'' zależnościach. W przeciwnym razie mogłaby istnieć potrzeba zbudowania zadania jeszcze raz.

Zauważmy, że przy ponownym uruchomieniu budowania może nie być potrzeby wykonywania niektórych zadań gdyż wejścia, od których zależą nie uległy, zmianie. Ta obserwacja prowadzi nas do konceptu minimalności, którą autorzy definiują następująco:

\definition{(Minimalność)}{
  Mówimy, że system kompilacji jest minimalny, gdy w trakcie budowania każde zadanie jest wykonane co najwyżej raz i tylko gdy w przechodnim domknięciu zadań, od których zależy, istnieje takie zadanie wejściowe, które zmieniło swoją wartość od czasu ostatniego budowania.
}

Dla Make'a informacją, które zadania należy zbudować ponownie są czasy modyfikacji plików, od których zależy zadanie~--~jeśli plik wynikowy zadania jest starszy niż wejścia, to znaczy, że zadanie powinno być ponownie zbudowane.

Należy też zauważyć, że dla pewnych konfiguracji może nie istnieć porządek topologiczny z nimi związany, gdyż istnieje cykl w zależnościach między zadaniami~--~nie będziemy jednak rozważać takich przypadków.

\subsection{Excel}

Może się to wydawać zaskakujące, ale o arkuszach kalkulacyjnych (np. programie Excel) możemy myśleć jak o systemach kompilacji. Komórki, których wartości są podane wprost uznajemy za zadania wejściowe, zaś formuły dla pozostałych komórek są definicjami sposobu budowania wartości dla nich. Przy takiej interpretacji, arkusze kalkulacyjne stają się bardzo przyjemnym oraz przydatnym przykładem systemu kompilacji.

Rozważmy teraz przykład arkusza kalkulacyjnego przedstawiony przez autorów oryginalnego artykułu, by łatwiej myśleć o tym rodzaju systemu:

\begin{tabular}{ l c r }
  A1: 10 & B1: INDIRECT(``A"\,\&\,C1) & C1: 1 \\
  A2: 20 & &
\end{tabular}

Funkcja INDIRECT dynamicznie określa, z której komórki zostanie pobrana wartość, a operator \(\&\) jest składaniem napisów. Gdy C1 = 1, wartością komórki B1 będzie wartość A1, zaś gdy C1 = 2, wartość zostanie pobrana z A2. Jak widzimy, komórki których wartości są wykorzystywane do obliczenia B1 zależą od wartości C1. W tej sytuacji mówimy o dynamicznych zależnościach między komórkami (a ogólniej, w kontekście systemów kompilacji -- zadaniami). Tutaj mamy tylko jeden stopień pośredniości, bo zależności B1 są determinowane przez wejście C1. Ogólniej, stopień pośredniości może być dowolnie duży. W takiej sytuacji mechanizm z sortowaniem topologicznym wykorzystywany w Make'u nie będzie właściwy, gdyż nie możemy a priori -- bez spoglądnięcia na stany innych komórek -- ustalić właściwego porządku budowania zadań.

Porządkowanie komórek w procesie ich obliczania jest w Excelu trochę bardziej skomplikowane. Mechanizm utrzymuje komórki w ciągu (zwanym łańcuchem). W procesie budowania Excel oblicza wartości komórek zgodnie ze skonstruowanym ciągiem. W sytuacji gdy komórka A potrzebuje wyniku innej, jeszcze nie obliczonej komórki N, Excel dokonuje restartu -- przerywa obliczanie A i przesuwa N przed A w ciągu oraz wznawia obliczanie wartości zaczynając od N. Po zakończeniu budowania, otrzymany ciąg komórek ma taką własność, że ponowne budowanie przy niezmienionych wejściach odbędzie się bez restartów. Ciąg pełni funkcję aproksymacji właściwego porządku obliczania komórek. Chcąc określić, które komórki muszą być obliczone ponownie, Excel dla każdej komórki utrzymuje informację czy jest ona brudna. Komórki stają się brudne, gdy:
\begin{itemize}
\item są wejściem i ich wartość zostanie zmieniona,
\item ich formuła zostanie zmieniona,
\item zawierają w formule funkcje, które uniemożliwiają statyczne określenie zależności -- jak na przykład INDIRECT czy IF.
\end{itemize}

Łatwo zauważyć, że Excel nie jest zatem minimalnym systemem budowania, gdyż z nadmiarem przyjmuje, które komórki muszą być obliczone ponownie. Ponadto, Excel śledzi nie tylko zmiany w wartościach wejść, ale także definicjach budowania zadań (formułach), co jest rzadką własnością w systemach kompilacji. Na ogół zmiana specyfikacji zadań wymusza na użytkowniku manualne rozpoczęcie pełnego procesu budowania.

\subsection{Shake}

Shake jest systemem kompilacji, w którym zadania definiuje się pisząc programy w języku specjalnego przeznaczenia osadzonym w Haskellu. Można w nim tworzyć konfiguracje z dynamicznymi zależnościami. Jednak w przeciwieństwie do Excela, Shake ma własność minimalności.

Zamiast konstruować ciąg zadań, jak robi to Excel, Shake generuje w trakcie budowania graf zależności. Ponadto, w przypadku wystąpienia zadania zależnego od innego dotychczas nieobliczonego, wstrzymuje wykonanie aktualnego i rozpoczyna budowanie wymaganego zadania. Gdy to się uda, wraca do wstrzymanego zadania znając już potrzebny wynik, by wznowić budowanie.

Inną własnością, którą posiada Shake, jest możliwość wykonywania wczesnego odcięcia -- w sytuacji, gdy jakieś zadanie zostało obliczone ponownie, ale jego wynik się nie zmienia, nie ma potrzeby ponownego obliczania zadań, które od niego zależą. Make i Excel nie posiadają takiej optymalizacji.

\subsection{Bazel}

Ostatnim przykładem systemu kompilacji jest Bazel, który powstał w odpowiedzi na zapotrzebowanie ze strony dużych zespołów pracujących nad oprogramowaniem znacznej wielkości. W takich projektach wiele osób może niezależnie budować te same fragmenty oprogramowania, co prowadzi do marnowania zasobów obliczeniowych oraz czasu programistów.

Bazel jest chmurowym systemem budowania -- gdy użytkownik chce zbudować oprogramowanie, system komunikuje się z serwerem i sprawdza, które z zadań mają niezmienione wejścia oraz czy zostały już przez kogoś zbudowane. Bazel skopiuje wyniki takich zadań do komputera użytkownika oszczędzając mu czas. Jako że pojedynczy programista na ogół wykonuje zmiany zamknięte w zaledwie kilku modułach, wyniki wielu zadań pozostają niezmienne i jedynie niewielka część z zadań będzie musiała być ponownie zbudowana.

Bazel śledzi zmiany sprawdzając wartości funkcji skrótu plików źródłowych. Gdy skróty pliku na komputerze użytkownika oraz serwerze systemu nie są zgodne, zadanie jest uznawane ze nieaktualne i budowane od nowa. Następnie wynik oraz nowe wartości funkcji skrótu są zapisywane na serwerze, funkcjonującym dla użytkowników jako ,,pamięć podręczna'' wyników budowania zadań.

Bazel nie wspiera aktualnie dynamicznych zależności. W procesie budowania wykorzystuje mechanizm restartowania zadań, a w celu określenia, które zadania muszą być przebudowane, utrzymuje wartości i skróty wyników zadań oraz historię wykonanych komend budowania.

\subsection{Wnioski}

Przedstawione cztery systemy kompilacji pokazały nam różne stopnie dowolności dane autorowi zadań co do stopnia skomplikowania ich obliczania. Poznaliśmy mechanizmy służące budowaniu zadań i optymalizacje, które zmniejszają liczbę niepotrzebnie obliczanych zadań. Ich wykorzystanie umożliwia niektórym systemom kompilacji osiągnięcie minimalności.

\section{Abstrakcyjnie o systemach kompilacji}

Po przedstawieniu aktualnego stanu rzeczy, autorzy proponują nomenklaturę i abstrakcyjną reprezentację przestrzeni związanej z systemami kompilacji.

\subsection{Nomenklatura}

Obiektem, na którym operuje system kompilacji jest zasób (Store), który kluczom przypisuje wartości. W przypadku Excela jest to arkusz złożony z komórek, zaś w Make'u system plików. Celem systemu jest zmodyfikowanie stanu zasobu w takich sposób, by wartość związana ze wskazanym przez użytkownika kluczem stała się aktualna. System ma pamięć w formie utrzymywanych trwałych informacji na potrzeby kolejnych uruchomień. Użytkownik dostarcza opis zadań w formie instrukcji określających jak mają być skonstruowane w oparciu o wyniki innych zadań.

System kompilacji otrzymuje definicje zadań, zasób na którym działa oraz klucz, który ma zostać zaktualizowany, wraz z jego zależnościami. Po zakończeniu działania, wartość w Store związana ze wskazanym kluczem ma być aktualna.

\subsection{Zasób oraz zadania}

\lstset{style=haskell-style}

Autorzy proponują następującą abstrakcyjną reprezentację zadania oraz zadań (jako kompletu definicji tychże):

\begin{lstlisting}
newtype Task c k v = Task (forall f. c f => (k -> f v) -> f v)
type Tasks c k v = k -> Maybe (Task c k v)
\end{lstlisting}

Zadanie oblicza swoją wartość korzystając z dostarczonej funkcji służącej uzyskiwaniu wartości innych zadań. Jest ono parametryzowane typem \textit{v} zwracanej wartości, typem kluczy \textit{k}. Jak widzimy, wartość nie jest zwracana wprost, a w nieznanym nośniku \textit{f}, który spełnia jednak warunek \textit{c}. Przykładami warunków w tym kontekście będą \textit{Applicative} oraz \textit{Monad}.

Grupa zadań jest funkcją, która kluczowi być może przyporządkowuje definicję jak skonstruować zadanie identyfikowane wskazanym kluczem. Zadania wejściowe nie mają do swoich kluczy przyporządkowanych definicji, a ich wartości są pobierane ze Store'a. Przykładowo, następującą instancję arkusza kalkulacyjnego:

\begin{tabular}{ l l }
  A1: 10 & B1: A1 + A2 \\
  A2: 20 & B2: 2 * B1
\end{tabular}

możemy wyrazić w naszej abstrakcji tak:

\begin{lstlisting}
sprsh1 :: Tasks Applicative String Integer
sprsh1 "B1" = Just $ Task $ \fetch -> ((+)  <$> fetch "A1" <*> fetch "A2")
sprsh1 "B2" = Just $ Task $ \fetch -> ((*2) <$> fetch "B1")
sprsh1 _    = Nothing
\end{lstlisting}

Zasób jest abstrakcyjnym typem danych parametryzowanym typami kluczy, wartości oraz trwałej informacji wykorzystywanej przez system kompilacji:

\begin{lstlisting}
data Store i k v
initialise :: i -> (k -> v) -> Store i k v
getInfo :: Store i k v -> i
putInfo :: i -> Store i k v -> Store i k v
getValue :: k -> Store i k v -> v
putValue :: Eq k => k -> v -> Store i k v -> Store i k v
\end{lstlisting}

Autorzy definiują podstawowe operacje na zasobie do konstruowania go, pozyskiwania i aktualizacji trwałej informacji oraz wartości kluczy.

\subsection{System kompilacji}

Typ systemu kompilacji wynika wprost z jego definicji -- otrzymuje zadania, zasób oraz klucz, a po zakończeniu działania, wartość w Store związana ze wskazanym kluczem ma być aktualna:

\begin{lstlisting}
type Build c i k v = Tasks c k v -> k -> Store i k v -> Store i k v
\end{lstlisting}

Rozważmy implementację bardzo prostego systemu budowania wyrażonego z użyciem przedstawionej abstrakcji:

\begin{lstlisting}
busy :: Eq k => Build Applicative () k v
busy tasks key store = execState (fetch key) store
  where
    fetch :: k -> State (Store () k v) v
    fetch k = case tasks k of
        Nothing   -> gets (getValue k)
        Just task -> do v <- run task fetch
                        modify (putValue k v)
                        return v
\end{lstlisting}

System \textit{busy} uruchamia obliczenie zadania w kontekście modyfikowalnego stanu, służy on spamiętywaniu wartości obliczonych zadań. Gdy zadanie ma być obliczone, jeśli jest wejściowym, to odczytana zostaje jego wartość ze Store'a, w przeciwnym razie zostaje wykonana jego definicja. System ten, podobnie jak kolejne, które zobaczymy później, składa się głównie z funkcji \textit{fetch}, która determinuje jego sposób działania. System \textit{busy} nie jest oczywiście minimalny, chociaż działa poprawnie i jest punktem początkowym do konstrukcji właściwych systemów.

System taki możemy łatwo uruchomić na przykładowym zasobie. Będzie on słownikiem realizowanym przez funkcję -- w ten sposób możemy łatwo ustalić wartość domyślną dla wszystkich wejściowych pól:

\begin{lstlisting}
> store = initialise () (\key -> if key == "A1" then 10 else 20)
> result = busy sprsh1 "B2" store
> getValue "B1" result
30
> getValue "B2" result
60
\end{lstlisting}

System działa i daje poprawne wyniki. Widzimy też, że przydaje nam się skwantyfikowanie ogólne parametru \textit{f} w definicji zadania:

\begin{lstlisting}
  newtype Task c k v = Task (forall f. c f => (k -> f v) -> f v)
\end{lstlisting}

W tym przypadku \inl{c = Applicative} oraz \inl{f = State (Store () k v) v}, w ten sposób funkcja \inl{fetch} może jako efekt uboczny wykonywać operacje na modyfikowalnym stanie opakowującym Store.

\subsection{Polimorficzność zadania}

Opakowanie wartości wynikowej umożliwia wykonywanie obliczeń z efektami ubocznymi, zaś kwantyfikator ogólny daje autorowi systemu kompilacji pełną swobodę doboru struktury, która będzie właściwa do jego potrzeb. W przypadku systemu \inl{busy} jest to modyfikowalny stan, w którym przechowywany jest zasób.

Gdyby \inl{f} było w pełni dowolne, nie dałoby się nic pożytecznego z nim zrobić, stąd musi być ograniczone przez pewne \inl{c}. Co zaskakujące, to ograniczenie definiuje jak skomplikowane mogą być zależności między zadaniami. Rozważmy trzy popularne (i jedną dodatkową) klasy typów w Haskellu:
\begin{itemize}
\item \inl{Functor} -- umożliwia nakładanie funkcji na wartość, którą opakowuje. Myśląc graficznie -- pracując z funktorem, tworzymy ciąg obliczeń modyfikujących wartość.
\item \inl{Applicative} -- umożliwia scalanie wielu wartości przez nakładanie na nie funkcji. Tutaj obliczenia prezentują się jako skierowany graf acykliczny.
\item \inl{Monad} -- w tym przypadku otrzymujemy dowolny graf, który jest ponadto dynamiczny (ze względu na wartości wynikowe). W procesie obliczeń możemy wyłuskiwać wartości i podejmować w oparciu o nie decyzje.
\item \inl{Selective}\cite{mokhov2019selective} -- jest formą pośrednią między funktorami aplikatywnymi, a monadami. Możliwe jest podejmowanie decyzji w oparciu o wyniki, jednak opcje do wyboru są zdefiniowane statycznie.
\end{itemize}

Autorzy dokonują więc niezwykle ciekawego odkrycia: zadania, w których typ \inl{f} jest funktorem aplikatywnym, mogą mieć jedynie statyczne zależności, zaś dynamiczne są możliwe gdy \inl{f} jest monadą!

Tak więc, przykład z INDIRECT w Excelu -- korzystając z naszej abstrakcji -- możemy w Haskellu przedstawić następująco:

\begin{lstlisting}
sprsh3 :: Tasks Monad String Integer
sprsh3 "B1" = Just $ Task $ \fetch -> do
    c1 <- fetch "C1"
    fetch ("A" ++ show c1)
sprsh3 _ = Nothing
\end{lstlisting}

Jednocześnie widzimy, że nie moglibyśmy wyrazić go z użyciem funktora aplikatywnego, gdyż nie mielibyśmy jak wyłuskać wartości komórki z wywołania \inl{fetch "C1"}.

Autorzy czynią kolejną obserwację, że nie tylko w teorii istnieje możliwość skonstruowania grafu zależności w zadaniach o statycznych zależnościach, ale także w praktyce -- realizuje to w Haskellu zaskakująco prosta funkcja \inl{dependencies}:

\begin{lstlisting}
dependencies :: Task Applicative k v -> [k]
dependencies task = getConst $ run task (\k -> Const [k]) where
  run :: c f => Task c k v -> (k -> f v) -> f v
  run (Task task) fetch = task fetch
\end{lstlisting}

Obliczenie wykonujemy korzystając z funktora \inl{Const}, który jest funktorem aplikatywnym, gdy pracuje na monoidach -- w tym przypadku listach. Jak widzimy, nigdzie nie jest wspomniany Store, co idzie w zgodzie z intuicją, że w przypadku zależności statycznych nie jest on nam potrzebny.

Jednocześnie nie moglibyśmy w takich sposób poznać zależności zadań z monadą, czyli dynamicznymi zależnościami, gdyż typ \inl{Const} nie jest monadą. Najlepszym przybliżeniem funkcji \inl{dependencies} jest \inl{track}, która śledzi wywołania funkcji pozyskującej wartość zadania z wykorzystaniem transformera monad \inl{WriteT}:

\begin{lstlisting}
track :: Monad m => Task Monad k v -> (k -> m v) -> m (v, [(k, v)])
track task fetch = runWriterT $ run task trackingFetch
  where
    trackingFetch :: k -> WriterT [(k, v)] m v
    trackingFetch k = do v <- lift (fetch k); tell [(k, v)]; return v
\end{lstlisting}

W tym przypadku musimy już niestety pracować z zasobem. Przykładowo, możemy przetestować funkcję \inl{track} korzystając z monady \inl{IO}, a wartości wprowadzając za pomocą klawiatury:

\begin{lstlisting}
> fetchIO k = do putStr (k ++ ": "); read <$> getLine
> track (fromJust $ sprsh2 "B1") fetchIO
C1: 1
B2: 10
(10,[("C1",1),("B2",10)])
> track (fromJust $ sprsh2 "B1") fetchIO
C1: 2
A2: 20
(20,[("C1",2),("A2",20)])

\end{lstlisting}

\section{Planiści i rekompilatorzy}

Autorzy proponują konstrukcję, w której system kompilacji jest definiowany przez dwa mechanizmy:
\begin{itemize}
\item planistę (scheduler) -- który decyduje w jakiej kolejności zadania powinny być budowane oraz
\item rekompilatora (rebuilder) -- który określa czy dane zadanie powinno być ponownie zbudowane, czy raczej wystarczy odczytać jego wartość wynikową ze Store'a.
\end{itemize}

Nie robiąc tego wprost rozważaliśmy już różne przykłady schedulerów i rebuilderów. Autorzy wyszczególniają trzy rodzaje planistów:

\begin{itemize}
\item topologicznego (topological) -- który wykorzystuje fakt, że zadania mają statyczne zależności,
%% \item restartującego (restarting) -- który w przypadku napotkania w czasie obliczania zadanie na inne niezaktualizowane zadanie przerywa obliczanie bieżącego i kiedyś zacznie je od nowa,
\item restartującego (restarting) -- który gdy w czasie obliczania zadania napotka na inne, niezaktualizowane zadanie, przerywa obliczanie bieżącego i kiedyś zacznie je od nowa,
\item wstrzymującego (suspending) -- który zamiast zaczynać od nowa, wstrzymuje jedynie obliczanie zadania do czasu uzyskania żądanej wartości.
\end{itemize}

Autorzy abstrakcyjnie przedstawiają planistów i rekompilatorów jako typy:

\begin{lstlisting}
type Scheduler c i ir k v = Rebuilder c ir k v -> Build c i k v
type Rebuilder c ir k v = k -> v -> Task c k v -> Task (MonadState ir) k v
\end{lstlisting}

%% albo -- przecinki

Tak więc, system kompilacji powstaje przez scalenie jakiegoś schedulera z jakimś rebuilderem. Rebuilder otrzymując klucz zadania oraz jego aktualną wartość i sposób obliczania tworzy nowe zadanie, które w oparciu o wnioski rekompilatora albo zbuduje zadanie i zagreguje dane dla rebuildera na potrzeby kolejnych uruchomień, albo zwróci wartość ze Store'a jeśli jest ona aktualna.

W przypadku rekompilatorów różnorodność jest trochę większa, wyszczególniamy rebuildery oparte o:
\begin{itemize}
\item brudny bit -- czy to w formie dosłownego bitu dla każdej komórki, jak to ma miejsce w Excelu, czy nietrywialnie przez weryfikowanie dat modyfikacji jak w Make'u -- mechanizm jest oparty na oznaczaniu wszystkich zadań wejściowych, których wartości się zmieniły od ostatniego uruchomienia systemu.
\item ślady weryfikujące -- które w procesie budowania rejestrują wartości funkcji skrótu uzyskanych wyników zadań i pamiętają, że na przykład zadanie A, gdy miało wartość o skrócie 1 było zależne od zadania B, gdy to miało wartość o skrócie 2. W sytuacji, gdy skróty są zgodne uznaje się, że ponowne obliczenie nie jest potrzebne.
\item ślady konstruktywne -- podobne do poprzedników, jednak funkcja skrótu jest funkcją identycznościową. Innymi słowy -- spamiętujemy całe wartości wynikowe zadań.
\item głębokie ślady konstruktywne -- zamiast rejestrować wartości bezpośrednich zależności, rejestrowane są wartości zadań wejściowych od których zadanie zależy (niezależnie czy bezpośrednio czy nie). Wadą tego mechanizmu jest brak wsparcia dla niedeterministycznych zadań, które rozważają autorzy w dalszej części swojej publikacji oraz brak możliwości wykonania wczesnego odcięcia, gdyż nie spoglądamy na wartości od których zadanie zależy bezpośrednio.
\end{itemize}

Sposób skategoryzowania systemów kompilacji przedstawiony przez autorów prowadzi do podziału przestrzeni systemów na 12 komórek, z czego 8 jest zamieszkałych przez istniejące rozwiązania:

\begin{tabular}{r | c c c}
\hline
                             & \multicolumn{3}{c}{Planista} \\
Rekompilator                 & Topologiczny & Restartujący & Wstrzymujący \\
\hline
Brudny bit                   & Make         & Excel        & - \\
Ślady weryfikujące           & Ninja        & -            & Shake \\
Ślady konstruktywne          & CloudBuild   & Bazel        & -\\
Głębokie ślady konstruktywne & Buck         & -            & Nix \\
\hline
\end{tabular}

\section{Implementowanie systemów}

Mając już ustaloną klasyfikację oraz definicje abstrakcyjnych konstrukcji i typów w Haskellu, można zaimplementować planistów i rekompilatorów. Wtedy utworzenie implementacji znanych systemów kompilacji (a nawet tych, które dotychczas były tylko pustymi polami w tabeli) jest zwykłym zaaplikowaniem rebuildera do schedulera. Wszystkie implementacje przedstawione przez autorów \BSaLC{} są dostępne w tekstach artykułów \cite{mokhov2018build, mokhov2020build} oraz w repozytorium\footnote{\url{https://github.com/snowleopard/build}} w serwisie GitHub. W rozdziale 5 zobaczymy, jak implementacja takich systemów wygląda w języku z efektami algebraicznymi i uchwytami.

\undef\inl



\chapter{Systemy kompilacji z użyciem efektów algebraicznych i uchwytów}
\chaptermark{Systemy z użyciem efektów i uchwytów}
\label{chapter-bsue}

W tym rozdziale powtórzymy implementację systemów kompilacji przedstawioną w \BSaLC\cite{mokhov2018build}, jednak dokonamy jej w języku programowania Helium używając efektów i uchwytów. Na początku wymyślimy własne odpowiedniki abstrakcyjnych struktur z Haskella związanych z systemami, następnie zaimplementujemy wszystkie rekompilatory oraz wszystkich (poza jednym) planistów. Na koniec przyglądniemy się czym charakteryzuje się pominięty planista i poznamy przykłady innych implementacji systemów inspirowanych wynikami Mokhov'a i innych.

\section{Pomysł, typy i idea}

Przypomnijmy sobie reprezentacje składowych implementacji z Haskella oraz wprowadźmy ich odpowiedniki w Helium.

\subsection{Zasób (Store)}

\begin{lstlisting}[style=haskell-style]
data Store i k v
initialise :: i -> (k -> v) -> Store i k v
getInfo :: Store i k v -> i
putInfo :: i -> Store i k v -> Store i k v
getValue :: k -> Store i k v -> v
putValue :: Eq k => k -> v -> Store i k v -> Store i k v
\end{lstlisting}

Autorzy \BSaLC\cite{mokhov2018build} reprezentowali Store jako typ z operacjami odczytu i zapisu trwałej informacji dla systemu oraz wartości wynikowych. Każdorazowo jednak, zasób był przechowywany w modyfikowalnym stanie. Możemy więc uprościć implementację przez scalenie zasobu z modyfikowalnym stanem przez uczynienie \haskinl{Store} efektem, a działania na nim operacjami powodującymi ten efekt.

\lstinputlisting[language=Haleff, firstline=5, lastline=9]{../src/Store.he}

Podobnie jak \haskinl{Store} w oryginalnej implementacji, \helinl{StoreEff} jest parametryzowany typem trwałej informacji, kluczy oraz wartości wynikowych kompilacji. Równania dla niego są analogiczne jak dla zwykłego modyfikowanego stanu z dokładnością do ustalenia klucza w operacjach na wartościach wynikowych. Definiujemy ponadto uchwyt \helinl{funStoreHandler}, w którym słownik klucz--wartość zadania utrzymywane są przez funkcję -- jak w przykładach w \BSaLC{}.

\lstinputlisting[language=Haleff, firstline=38, lastline=52]{../src/Store.he}

Implementacja jest zbliżona do przykładu modyfikowalnego stanu z rozdziału 4. Dla porządku wartość początkowa trwałej informacji oraz słownika wartości jest opakowana w typ \helinl{FunStoreType I K V}.

Jako że Helium, podobnie jak inne języki używające ML-owego systemu modułów, nie posiada klas typów znanych z Haskella, definiujemy kilka sygnatur odpowiadających klasom typów użytym w oryginalnej implementacji. W przypadku \helinl{funStoreHandler} moduł o sygnaturze \helinl{Comparable K} jest używany do porównywania kluczy identyfikujących zadania. Można zauważyć, że alternatywnym rozwiązaniem byłoby reprezentowanie odpowiedników klas typów jako efekty.

\begin{minipage}[t]{.45\textwidth}

  \lstinputlisting[language=Haleff, firstline=4, lastline=16]{../src/Signatures.he}

\end{minipage}\hfill
\begin{minipage}[t]{.45\textwidth}

  \lstinputlisting[language=Haleff, firstline=18, lastline=27]{../src/Signatures.he}

\end{minipage}

\subsection{Modyfikowalny stan}

Implementację modyfikowalnego stanu zobaczyliśmy w przykładach w rozdziale 4 i wykorzystamy ją konstruując systemy kompilacji. Nazwy uchwytom dla stanu, w zależności od zwracanych wartości, nadajemy zgodnie z ich odpowiednikami w Haskellu -- \haskinl{runState}, \haskinl{evalState}, \haskinl{execState}. Definiujemy także proste funkcje \helinl{gets} i \helinl{modify}, które używając podanego przekształcenia odpowiednio odczytują i modyfikują stan, oraz nieco bardziej skomplikowaną funkcję \helinl{embedState}.

\lstinputlisting[language=Haleff, linerange={6-7,33-37}, float=h, title={Definicje \helinl{gets}, \helinl{modify} oraz \helinl{embedState}}]{../src/State.he}

Funkcja \helinl{embedState} tworzy uchwyt dla efektu modyfikowalnego stanu, w którym modyfikacje -- zamiast być wykonywane przez uchwyt -- są przekazywane podanym funkcjom \helinl{getter} oraz \helinl{setter}, które w czasie swojego działania mogą powodować jakiś efekt uboczny. Z takiego zanurzenia modyfikowalnego stanu w innym efekcie będziemy korzystać podczas implementacji planistów, którzy trwałą informację z zasobu będą przekazywać do rekompilatorów jako właśnie modyfikowalny stan.

\begin{lstlisting}[language=Haleff, float=h, title={Przykład wykorzystania \helinl{embedState}}]
handle `store in
    (* ... *)
    handle `state in
        (* ... *)
    with embedState (getInfo `store) (putInfo `store)
    (* ... *)
with (* ... *)
\end{lstlisting}

\subsection{Zadanie i efekt kompilacji}

W oryginalnej implementacji zadanie było funkcją przyjmującą procedurę kompilacji wskazanego zadania, a wynik był zwracany w jakimś typie \haskinl{f} ograniczonym przez klasę typów \haskinl{c}.

\begin{lstlisting}[style=haskell-style]
newtype Task c k v = Task (forall f. c f => (k -> f v) -> f v)
type Tasks c k v = k -> Maybe (Task c k v)
\end{lstlisting}

Możemy jednak zauważyć, że kompilacja zadania jest oczywistym efektem ubocznym działania systemu kompilacji, stąd w naszej implementacji zamiast przekazywać funkcję, która była przez autorów zazwyczaj nazywana \haskinl{fetch}, zdefiniujemy efekt \helinl{BuildEff}, który będzie występował w czasie kompilacji zadań. Z efektem tym związana będzie jedna operacja \helinl{fetch}.

\lstinputlisting[language=Haleff, linerange={14-16}]{../src/Common.he}

Zadanie będzie funkcją wymagającą informacji o instancji efektu budowania i będzie polimorficzna ze względu na typ kluczy i wartości oraz ewentualnych efektów ubocznych nie będących efektem budowania (będzie to przydatne przy implementacji rebuilderów). Zwróćmy uwagę, że definicja typu zadania nie zawiera informacji analogicznych do klasy typów \haskinl{c}, której element \haskinl{f} ,,opakowywał'' wynik w oryginalnej implementacji -- do tej różnicy powrócimy w dalszej części rozdziału.

\subsection{Kompilacja, planista, rekompilator}

Pozostaje zdefiniować trzy ostatnie typy związane ze wspomnianymi w podtytule obiektami.

\begin{lstlisting}[style=haskell-style]
type Build c i k v = Tasks c k v -> k -> Store i k v -> Store i k v
type Scheduler c i ir k v = Rebuilder c ir k v -> Build c i k v
type Rebuilder c ir k v = k -> v -> Task c k v -> Task (MonadState ir) k v
\end{lstlisting}

Kompilacja, tak jak w oryginalnej implementacji, wymagać będzie wskazania zbioru zadań oraz klucza który ma być zbudowany. Ponadto w naszej implementacji kompilacja powoduje efekt uboczny zmiany zasobu.

Nasi planiści także będą mieli sygnatury zbliżone do swoich odpowiedników z Haskella, wzbogacone oczywiście o efekt uboczny zasobu, a także moduł definiujący opisane wcześniej podstawowe działania na kluczach i wartościach.

\lstinputlisting[language=Haleff, linerange={18-20}]{../src/Common.he}

Rebuilder przypomina swój odpowiednik z oryginalnej implementacji. Jednak, zamiast zwracać zadanie ze zmienionym \textit{constraint'em}, zadanie jest wzbogacone o dodatkowy efekt stanu mogący występować w czasie kompilacji zadania.

\section{Przykład: system busy}

Skoro ustaliliśmy jak abstrakcja systemów kompilacji w Haskellu przenosi się na naszą w Helium, możemy spróbować zaimplementować prosty system budowania \haskinl{busy} przedstawiony przez autorów.

\lstinputlisting[language=Haleff, linerange={16-28}, float=h, title={System kompilacji \haskinl{busy}}]{../src/Schedulers.he}

Rdzeniem implementacji, podobnie jak oryginalnej w Haskellu, jest definicja uchwytu (tam: funkcji) dla \helinl{fetch}. Jego ciało to przetłumaczenie oryginalnej implementacji z tą różnicą, że zamiast kontynuować obliczenie niejawnie -- przez zwracanie wyniku -- jest ono kontynuowane jawnie przez wywołanie \helinl{resume} w ciele uchwytu.

\section{Implementacja śladów}

Podobnie jak w \BSaLC{}, implementacje funkcji pracujących ze śladami nie są interesujące -- w naszym przypadku odpowiadają oryginałom poza kilkoma szczegółami w postaci wykorzystania efektu \helinl{Writer} zamiast infrastruktury zbudowanej wokół typu \haskinl{Maybe} oraz list comprehensions w Haskellu. Implementacje wraz z komentarzami dostępne są w Dodatku A oraz kodzie źródłowym.

\section{Uruchamianie i śledzenie działań}

W implementacjach planistów i rekompilatorów będziemy chcieli uruchamiać zadania oraz śledzić od jakich zadań zależy aktualnie rozważane. W tym celu, podobnie jak autorzy \BSaLC{}, definiujemy prostą funkcję \helinl{run} oraz nieco ciekawszą \helinl{track}.

\lstinputlisting[language=Haleff, linerange={12-13,20-29}]{../src/Track.he}

Funkcja \helinl{track} otrzymuje etykietę \helinl{`b} uchwytu dla efektu kompilacji oraz zadanie \helinl{task}, które ma być uruchomione pod jego nadzorem, a \helinl{track} ma wyznaczyć zadania, od których \helinl{task} zależy. W tym celu konstruowany jest dodatkowy uchwyt \helinl{hTrack}, pod nadzorem którego uruchamiamy zadanie. W sytuacji, gdy uruchomione zadanie potrzebuje wyniku innego zadania, \helinl{hTrack} ,,przechwyci'' wystąpienie \helinl{fetch} i oddeleguje wystąpienie operacji do uchwytu o etykiecie \helinl{`b}, a następnie odnotuje, że miało miejsce wywołanie \helinl{fetch}.

Implementacja funkcji \helinl{track} jest ciekawym przykładem skonstruowania pośrednika (proxy) pomiędzy obliczeniem, które ma efekty uboczne, a właściwym dla niego uchwytem.

\pagebreak

\section{Implementacje systemów kompilacji}

\subsection{Excel}

\lstinputlisting[style=Haleff-long, linerange={13-14}]{../src/Systems.he}

Już na starcie widzimy, że udało nam się dopełnić obietnicy, którą postulują autorzy \BSaLC{} -- systemy kompilacji powstają przez zaaplikowanie rekompilatora do planisty.

Funkcja \helinl{dirtyBitRebuiler} modyfikuje zadanie tak, aby przy uruchomieniu sprawdzało, czy klucz zadania jest oznaczony jako brudny. Gdy tak jest, zadanie zostanie skompilowane, w przeciwnym razie można wykorzystać wartość dostarczoną do rekompilatora, gdyż to ją zwróciłoby wykonanie pierwotnego zadania. 

\lstinputlisting[style=Haleff-long, linerange={36-39}]{../src/Rebuilders.he}

W planiście restartującym utrzymujemy łańcuch, który ma aproksymować kolejność kompilacji, w której minimalizujemy liczbę restartów. Działanie rozpoczyna się od wykorzystania łańcucha z poprzedniej kompilacji, a jego wersję wykorzystywaną i modyfikowaną w czasie działania utrzymujemy w instancji stanu o etykiecie \helinl{`chain}. Ponadto, w stanie \helinl{`done} odnotowujemy, które zadania skompilowaliśmy w tej instancji procesu, aby nie musieć uruchamiać ich ponownie oraz tworzymy uchwyt, wykorzystując opisaną wcześniej funkcję \helinl{embedState} dla modyfikowalnego stanu odpowiadającego trwałej informacji systemu kompilacji.

\lstinputlisting[style=Haleff-long, linerange={56-71}]{../src/Schedulers.he}
\vspace{-1.25em}
\begin{lstlisting}[style=Haleff-long]
    let rec restartingHandler = (* ... *)
        and loop () = (* ... *)
    in
    let resultChain = loop () in
        modifyInfo (mapSnd (fn _ => resultChain))
\end{lstlisting}

Właściwa część implementacji tego planisty składa się z uchwytu efektu kompilacji \helinl{restartingHandler} oraz funkcji \helinl{loop}. Funkcja ta wykonuje zadania w kolejności zadanej przez łańcuch z poprzedniej instancji, modyfikując zadania z użyciem rekompilatora, po czym je uruchamiając. Jednocześnie konstruowany jest nowy łańcuch, który jest wartościową zwracaną przez \helinl{loop}.

\lstinputlisting[style=Haleff-long, linerange={72-99}]{../src/Schedulers.he}

W czasie kompilacji zadania wystąpienia \helinl{fetch} są przechwytywane przez uchwyt, który sprawdza, czy zadanie jest już obliczone. W przeciwnym razie modyfikuje łańcuch tak, by potrzebne zadanie znalazło się przed zadaniem aktualnie obliczanym. W uchwycie wykorzystana jest opcja dla \helinl{return}, która odnotowuje, że zadanie skończyło się kompilować, a następnie wywołuje \helinl{loop}.

\subsection{Shake}

\lstinputlisting[style=Haleff-long, linerange={16-17}]{../src/Systems.he}

W systemie Shake rebuilder wykorzystuje ślady weryfikujące. Rekompilator używając \helinl{verifyVT} sprawdza, czy zadanie jest świeże. Jeśli tak, nie musi być obliczane ponownie. W przeciwnym razie zadanie jest kompilowane pod nadzorem funkcji \helinl{track}, która akumuluje listę bezpośrednich zależności zadania, by utworzyć z nich nowe ślady do trwałego zachowania z użyciem \helinl{recordVT}.

\pagebreak

\lstinputlisting[style=Haleff-long, linerange={43-49}]{../src/Rebuilders.he}

Implementacja planisty wstrzymującego jest znacznie krótsza od restartującego. Utrzymujemy tylko dwa stany: pierwszy (\helinl{`done}) dla odnotowania już skompilowanych zadań oraz drugi dla osadzenia trwałej informacji w stanie na potrzeby działania rekompilatora -- podobnie jak w planiście restartującym.

\lstinputlisting[style=Haleff-long, linerange={30-54}]{../src/Schedulers.he}

Uchwyt \helinl{suspendingHandler} jest niezwykle prosty -- wywołuje jedynie funkcję \helinl{build}, po czym wznawia kompilację z wynikiem potrzebnego zadania uzyskanym ze Store'a. Procedura \helinl{build} sprawdza, czy zadanie jest nietrywialne (czy nie jest wejściem) oraz czy nie zostało już obliczone. Wtedy konstruowane jest nowe zadanie z użyciem rekompilatora, po czym następuje jego uruchomienie. W innych przypadkach zadanie jest aktualne i na pewno nie ma potrzeby kompilować go ponownie.

\subsection{CloudShake}

\lstinputlisting[style=Haleff-long, linerange={19-20}]{../src/Systems.he}

\lstinputlisting[style=Haleff-long, linerange={59-69}]{../src/Rebuilders.he}

W przypadku śladów konstruktywnych rebuilder sprawdza, czy podana wartość zadania jest już wśród znanych wartości. W przeciwnym razie można zwrócić dowolną znaną wartość lub -- gdy żadna wartość nie jest znana -- następuje kompilacja zadania. Podobnie jak w przypadku rekompilatora opartego o ślady weryfikujące, tutaj kompilacja też odbywa się ze śledzeniem zadań, od których kompilowane zależy.

%% \lstinputlisting[style=Haleff-long, linerange={30-54}]{../src/Schedulers.he}

\subsection{Nix}

\lstinputlisting[style=Haleff-long, linerange={22-23}]{../src/Systems.he}

Rekompilator używający głębokich śladów konstruktywnych przypomina swoich poprzedników. Jednak -- zgodnie ze swoją nazwą -- sprawdza, od których zadań wejściowych w istocie badane zadanie zależy.

\lstinputlisting[style=Haleff-long, linerange={73-84}]{../src/Rebuilders.he}

%% \lstinputlisting[style=Haleff-long, linerange={30-54}]{../src/Schedulers.he}

\section{Nieobecny planista topologiczny}

Jak zobaczyliśmy w rozdziale 3, planiści restartujący i wstrzymujący radzą sobie z zadaniami o dynamicznych jak i statycznych zależnościach. Inaczej sytuacja ma się w przypadku planisty topologicznego, który działa jedynie z zadaniami o statycznych zależnościach, które w \BSaLC{} są modelowane z wykorzystaniem klasy \haskinl{Applicative}.

Wydaje się, że nie mamy jak uniemożliwić zadaniom inspekcję wyników wywołań \helinl{fetch}. Moglibyśmy opakowywać je w nieznany twórcy zadania typ, co wydaje się powracać do oryginalnej implementacji. Oddalamy się jednak od efektów algebraicznych i uchwytów będących tematem tej pracy, stąd nie będziemy badać dokładniej tematu modelowania statycznych zależności.

\section{Istniejące podejścia do implementacji w innych językach}

Opisane wyżej wyniki są pierwszą -- według wiedzy autora -- próbą implementacji systemów kompilacji inspirowanych \BSaLC{} używając języka z efektami algebraicznymi oraz uchwytami. W \BSaLCTP{} Mokhov i inni wspominają jednak o dwóch znanych im próbach implementacji systemów kompilacji w popularnych językach programowania: Rust\cite{translation_rust} oraz Kotlin\cite{translation_kotlin}. Jak jednak zauważają, w obu przypadkach ograniczenia użytych języków doprowadziły do utracenia precyzji i schludności rozwiązań w porównaniu z oryginalną implementacją w Haskellu.

O ile brak planisty topologicznego w naszej implementacji rzeczywiście oddala nas od oryginału, o tyle planiści oraz rekompilatorzy zaimplementowani przez nas -- z dokładnością do różnic składniowych języków -- nie odbiegają jakością oraz czytelnością od swoich pierwowzorów.





\chapter{Podsumowanie i wnioski}

Celem pracy było zapoznanie i zaciekawienie czytelnika tematem efektów algebraicznych i uchwytów oraz zaprezentowanie nowej implementacji systemów kompilacji podążając krokami autorów \BSaLC{}. Implementacja w eksperymentalnym języku Helium miała zademonstrować, jak wygląda programowanie z efektami algebraicznymi i uchwytami oraz umożliwić zaobserwowanie, jak różni się ono od radzenia sobie z efektami ubocznymi przez użycie monad w języku Haskell.

Traktując \helinl{fetch} jako operację efektu ubocznego kompilacji, a nie jako argument do zadania, udało się nam wykorzystać możliwości języka z efektami i uchwytami w centralnej części implementacji systemów kompilacji.

Uzyskana -- dzięki programowaniu z efektami i uchwytami, a nie monadami -- swoboda użycia wielu efektów jednocześnie uspokoiła nasze obawy i zachęciła do eksperymentowania. Etykietowanie różnych instancji tego samego efektu umożliwiło utrzymywanie w modyfikowalnym stanie wielu wartości bez szkody dla czytelności oraz rozumieniu kodu. Dało to też możliwość tworzenia pośredników między różnymi efektami. % (\helinl{embedState}, funkcja \helinl{track}).

Reprezentacja zasobu, nad którym odbywała się kompilacja jako efektu ubocznego, nie tylko zapobiegła potrzebie każdorazowego umieszczania go w modyfikowalnym stanie, ale także lepiej oddało jego naturę bycia trwałym i zewnętrznym tworem.

Tym, co utraciliśmy, była precyzja opisu skomplikowania relacji między zadaniami. Transparentność wyników operacji z efektami, w porównaniu do ,,opakowywania'' ich instancjami funktorów aplikatywnych lub monad, uniemożliwiła łatwe reprezentowanie zadań o statycznych zależnościach.

Problem ten był jednak łatwy do zauważenia już na początku rozważań nad własną implementacją \BSaLC{}. Oprócz tego, w czasie implementowania systemów kompilacji, autor nie napotkał znacznych trudności w programowaniu z efektami algebraicznymi i uchwytami. Pozostałe były związane z ograniczonym doświadczeniem autora z językiem Helium lub eksperymentalną naturą języka i chwilowymi problemami środowiska uruchomieniowego z klarownym objaśnieniem źródła niezgodności typów.

Podsumowując, programowanie z efektami algebraicznymi i uchwytami jest możliwe, jest przyjemne i uwalnia autora od ograniczeń, które dotychczas wydawały się nie do uniknięcia. Jak zobaczyliśmy, można spróbować powtórzyć wyniki przeprowadzone w znanym funkcyjnym języku programowania i z zachwytem odkryć, że implementacja z efektami i uchwytami jest równie interesująca.


\section{}

\begin{frame}{}
  \vspace{2em}
  \centering
  \Large\emph{Dziękuję za uwagę}
\end{frame}

\end{document}
