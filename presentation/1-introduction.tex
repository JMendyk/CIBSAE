
\section{Wstęp pracy}

\subsection{Problemy z efektami ubocznymi}

\newcommand\pro{\item[\textcolor{example text.fg}{$+$}]}
\newcommand\con{\item[\alert{$-$}]}

\begin{frame}
  \frametitle{Efekty uboczne}
  \begin{exampleblock}{Zalety}
    \begin{itemize}
      \pro komunikacja z innymi systemami
      \pro trwała pamięć -- system plików, bazy danych
      \pro interaktywność
    \end{itemize}
  \end{exampleblock}
  \begin{alertblock}{Wady}
    \begin{itemize}
      \con zależność od świata zewnętrznego
      \con utrudnione rozumienie, brak modularności
      \con częstsze pomyłki
    \end{itemize}
  \end{alertblock}
\end{frame}

\begin{frame}
  \frametitle{Efekty uboczne są problematyczne}
  \begin{exampleblock}{Pomysł}
    Rozdzielić program na część czystą oraz część mającą efekty uboczne.
  \end{exampleblock}
  \pause
  \vspace{1em}
  \begin{alertblock}{Problem}
    Musimy zaufać autorowi, że funkcja rzeczywiście nie powoduje efektów ubocznych.
  \end{alertblock}
\end{frame}

\subsection{Radzenie sobie z efektami ubocznymi}

\begin{frame}
  \frametitle{Radzenie sobie z efektami ubocznymi}
    Potrzebujemy znaleźć kogoś, kto będzie pilnował czy funkcje, które twierdzą że nie mają efektów ubocznych rzeczywiście takie są.
  \pause
  \begin{exampleblock}{Pomysł}
    Wykorzystajmy system typów -- jest dobry w sprawdzaniu czy deklaracje programisty (adnotacje typów) są zgodne ze stanem faktycznym (implementacjami funkcji). Inferencja wyręczy nas od potrzeby pisania typów w wielu przypadkach (w przeciwieństwie do np. języka C).
  \end{exampleblock}
\end{frame}

\begin{frame}
  \frametitle{Monady}
  \begin{itemize}
  \pro umożliwiają bezpieczne programowanie z efektami
  \pro informacje o efektach ubocznych w sygnaturze
  \pro efekty nie mogą ,,uciec''
  \con potrzeba transformerów monad by użyć wielu efektów naraz
  \con modularność wciąż problematyczna
  \end{itemize}
\end{frame}

\begin{frame}
  \frametitle{Efekty algebraiczne i uchwyty}
  \begin{itemize}
    \pro umożliwiają bezpieczne programowanie z efektami
    \pro informacje o efektach ubocznych w sygnaturze
    \pro efekty nie mogą ,,uciec''
    \pro łatwość użycia wielu efektów jednocześnie
    \pro modularność i przejrzystość
  \end{itemize}
\end{frame}

\subsection{Systemy kompilacji}

\begin{frame}
  \frametitle{Systemy kompilacji}
  \begin{itemize}
  \item interakcja z zewnętrznymi zasobami
  \item uznawane za zło konieczne, zbyt skomplikowane
  \item powszechne wykorzystanie w ,,przemyśle''
  \item rzadko obiekt zainteresowań badaczy
  \end{itemize}
\end{frame}

\begin{frame}
  \begin{figure}
    \includegraphics[width=0.8\paperwidth, height=0.8\paperheight, keepaspectratio,
      trim=45 295 45 80, clip]{../pdfs/build-systems.pdf}
    \caption*{\begin{tiny}{https://dl.acm.org/doi/pdf/10.1145/3236774}
      \end{tiny}}
  \end{figure}
\end{frame}
