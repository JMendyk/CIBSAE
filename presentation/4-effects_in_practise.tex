
\begin{frame}
  \frametitle{}
  \begin{columns}[t]
    \column{.4\paperwidth}
    Biblioteki:
    \begin{itemize}
      \item extensible-effects (Haskell)
      \item fused-effects (Haskell)
      \item atnos-org/eff (Scala)
      \item Effects (Idris)
    \end{itemize}
    \column{.4\paperwidth}
    Języki programowania:
    \begin{itemize}
      \item Eff
      \item Frank
      \item Koka
      \item Helium
    \end{itemize}
  \end{columns}
\end{frame}

\newcommand{\inl}[1]{\lstinline[style=Haleff-inl]{#1}}
\lstset{language=Haleff, showstringspaces=false, inputpath=../thesis/code_examples}

\begin{frame}
  \frametitle{Przykład programu w Helium}
  \lstinputlisting{syntax.he}
\end{frame}

\begin{frame}
  \frametitle{Efekt błędu}
  \lstinputlisting{error1__signature.he}
  \lstinputlisting[firstline=7, lastline=18]{error2__inline_abort.he}
\end{frame}

\begin{frame}
  \frametitle{Jeden efekt, różne uchwyty}
  \lstinputlisting[firstline=7, lastline=10]{error3__reused_handler.he}
  \lstinputlisting[firstline=7, lastline=10]{error4__warn_not_abort.he}
\end{frame}

\begin{frame}
  \frametitle{Niedeterminizm}
  \lstinputlisting[linerange={1-2}]{nondet1__simple.he}
  \begin{columns}[t]
    \column{.455\paperwidth}
    \lstinputlisting[linerange={4-7}]{nondet1__simple.he}
    \column{.455\paperwidth}
    \lstinputlisting[linerange={9-12}]{nondet1__simple.he}
  \end{columns}
  \lstinputlisting[linerange={14-22}]{nondet1__simple.he}
\end{frame}

\begin{frame}
  \frametitle{Niedeterminizm 2}
  \lstinputlisting[linerange={5-16}]{nondet2__count_sats.he}
\end{frame}

\begin{frame}
  \frametitle{Modyfikowalny stan}
  \lstinputlisting[lastline=3]{state1__basics.he}
  \lstinputlisting[firstline=5, lastline=11]{state1__basics.he}
\end{frame}

\begin{frame}
  \frametitle{Rekursja}
  \lstinputlisting[firstline=2, lastline=5]{rec1__rec.he}
  \vspace{2em}
  \lstinputlisting[firstline=2, lastline=13]{rec2__effect.he}
\end{frame}

\begin{frame}
  \frametitle{Rekursja}
  \lstinputlisting[firstline=5, lastline=23]{rec3__mutual.he}
\end{frame}

\begin{frame}
  \frametitle{Wiele efektów naraz}
  \lstinputlisting[linerange={1-13}]{fail_and_amb.he}
\end{frame}

\begin{frame}
  \frametitle{Wiele efektów naraz -- sprawdzanie spełnialności}
  \lstinputlisting[linerange={17-23}]{fail_and_amb.he}
\end{frame}

\begin{frame}
  \frametitle{Wiele efektów naraz -- sprawdzanie tautologiczności}
  \lstinputlisting[linerange={25-31}]{fail_and_amb.he}
\end{frame}

\begin{frame}
  \frametitle{Wiele efektów naraz}
  Łączenie efektów jest bardzo proste, a kolejność w jakiej umieszczamy uchwyty umożliwia łatwe i czytelne definiowanie zachowania programu w przypadku wystąpienia któregokolwiek z efektów.
\end{frame}
