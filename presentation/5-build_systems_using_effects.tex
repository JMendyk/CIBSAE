
\section{Systemy z użyciem efektów i uchwytów}

\lstset{language=Haleff, showstringspaces=false, inputpath=../src}

\subsection{Reprezentacja}

\begin{frame}[fragile]
  \frametitle{Zasób (Store)}

  \lstinputlisting[language=Haleff, firstline=5, lastline=9]{Store.he}
  \lstinputlisting[language=Haleff, firstline=38, lastline=52]{Store.he}
\end{frame}

\begin{frame}[fragile]
  \frametitle{Modyfikowalny stan}
  \lstinputlisting[language=Haleff, linerange={33-37}]{State.he}

  \begin{lstlisting}[language=Haleff]
handle `store in
    (* ... *)
    handle `state in
        (* ... *)
    with embedState (getInfo `store) (putInfo `store)
    (* ... *)
with (* ... *)
\end{lstlisting}
\end{frame}

\begin{frame}[fragile]
  \frametitle{Zadanie i efekt kompilacji}

  \lstinputlisting[language=Haleff, linerange={11-15}]{Common.he}

  \vspace{1em}
  {\small (brak ograniczenia rodzaju zależności, o tym później)}
\end{frame}

\begin{frame}[fragile]
  \frametitle{Kompilacja, planista, rekompilator}
  \lstinputlisting[language=Haleff, linerange={17-19}]{Common.he}
  \lstinputlisting[language=Haleff, linerange={20-29}]{Track.he}
\end{frame}

\begin{frame}[fragile]
  \frametitle{Przykład: system busy}
  \lstinputlisting[language=Haleff, linerange={16-28}]{Schedulers.he}
\end{frame}

\subsection{Implementacje planistów}

\begin{frame}[fragile]
  \frametitle{Planista restartujący}
  \lstinputlisting[style=Haleff-long, linerange={56-71}]{Schedulers.he}
  \vspace{-1.25em}
\begin{lstlisting}[style=Haleff-long]
    let rec restartingHandler = (* ... *)
        and loop () = (* ... *)
    in
    let resultChain = loop () in
        modifyInfo (mapSnd (fn _ => resultChain))
\end{lstlisting}
  
\end{frame}

\begin{frame}
  \frametitle{Planista restartujący (cd.)}
  \lstinputlisting[style=Haleff-long, linerange={72-99}]{Schedulers.he}
\end{frame}

\begin{frame}[fragile]
  \frametitle{Planista wstrzymujący}
  \lstinputlisting[style=Haleff-long, linerange={30-54}]{Schedulers.he}
\end{frame}

\subsection{Implementacje rekompilatorów}

\begin{frame}[fragile]
  \frametitle{Rekompilatoy z brudnym bitem oraz śladami weryfikującymi}
  \lstinputlisting[style=Haleff-long, linerange={36-39}]{Rebuilders.he}
  \lstinputlisting[style=Haleff-long, linerange={43-49}]{Rebuilders.he}
\end{frame}

\begin{frame}[fragile]
  \frametitle{Rekompilatoy ze śladami konstruktywnymi}
  \lstinputlisting[style=Haleff-long, linerange={59-69}]{Rebuilders.he}
  \lstinputlisting[style=Haleff-long, linerange={73-84}]{Rebuilders.he}
\end{frame}

\subsection{}

\begin{frame}
  \frametitle{Co z planistą topologicznym?}

  Wydaje się, że nie mamy jak uniemożliwić zadaniom inspekcję wyników wywołań \helinl{fetch}.

  \vspace{1em}
  
  Opakowywanie w nieznany twórcy zadania typ -- powrót do oryginalnej implementacji -- mało satysfakcjonujące.
\end{frame}
